% !TeX spellcheck = en_US
\documentclass[10pt, english]{article}

\usepackage{geometry}
\usepackage[no-math]{fontspec}
\usepackage{mathfont}
\usepackage{babel}
\usepackage[autostyle]{csquotes}
\usepackage{graphicx}
\usepackage[x11names, table]{xcolor}
\usepackage{import}
\usepackage{tikz}
\usepackage{pgfplots}
\usepackage{pgfplotstable}
\usepackage{xifthen}
\usepackage{caption}
\usepackage[hidelinks]{hyperref}
\usepackage{booktabs}
\usepackage{tabularx}
\usepackage{seqsplit}
\usepackage{texshade}

%%% font definitions
\setsansfont{Arial}
\renewcommand{\familydefault}{\sfdefault}
\mathfont{Arial}
\newfontfamily{\signiffont}{Arial Unicode MS}
\setmonofont{Lucida Sans Typewriter}

%%% font sizes
\newcommand{\fignormal}{\scriptsize}
\newcommand{\figsmall}{\fontsize{6}{7}\selectfont}
\newcommand{\figtiny}{\tiny}
\newcommand{\figlarge}{\footnotesize}
\newcommand{\fighuge}{\small}

%%% page layout
\geometry{letterpaper, textwidth = 180mm, textheight = 225 mm, marginratio = 1:1}
\renewcommand{\textfraction}{0.01}
    
%%% header and footer
\pagestyle{empty}

%%% no indentation
\setlength\parindent{0pt}

%%% caption format
\DeclareCaptionLabelSeparator{bar}{ | }

\captionsetup{
	labelsep = bar,
	justification = justified,
	singlelinecheck = false,
	labelfont = bf,
	font = small,
	figureposition = below,
	tableposition = above
}
\captionsetup[figure]{skip = .5\baselineskip}

\newcommand{\titleend}{ }
\newcommand{\nextentry}{ }

%%% useful lengths
\newlength{\onecolumnwidth}
\setlength{\onecolumnwidth}{\textwidth} % 180mm
\newlength{\twocolumnwidth}
\setlength{\twocolumnwidth}{88mm}
\newlength{\threecolumnwidth}
\setlength{\threecolumnwidth}{57mm}
\newlength{\twothirdcolumnwidth}
\setlength{\twothirdcolumnwidth}{119mm}
\newlength{\fourcolumnwidth}
\setlength{\fourcolumnwidth}{42mm}
\newlength{\threequartercolumnwidth}
\setlength{\threequartercolumnwidth}{134mm}
\setlength{\columnsep}{4mm}

\newlength{\templength}
\newlength{\basewidth}

\newlength{\xdistance}
\newlength{\ydistance}

%%% useful commands
\renewcommand{\textprime}{\char"2032}
\renewcommand{\textDelta}{\char"0394}

\makeatletter
	\newcommand{\distance}[2]{
		\path (#1);
		\pgfgetlastxy{\xa}{\ya} 
		\path (#2);
		\pgfgetlastxy{\xb}{\yb}   
		\pgfpointdiff{\pgfpoint{\xa}{\ya}}{\pgfpoint{\xb}{\yb}}%
		\setlength{\xdistance}{\pgf@x}
		\setlength{\ydistance}{\pgf@y}
	}
	
	\def\convertto#1#2{\strip@pt\dimexpr #2*65536/\number\dimexpr 1#1}
\makeatother

\makeatletter
  \def\thisfilepath{\import@path}
\makeatother

\newcommand{\GCto}{%
	\\[-.25\baselineskip]to\\[-.2\baselineskip]%
}

%%% tikz setup
\pgfmathsetseed{928}

%%% general
\usetikzlibrary{calc, positioning, arrows.meta, arrows, bending, external, backgrounds, topaths}

\tikzexternalize[prefix=extFigures/, only named=true]

%%% layers
\pgfdeclarelayer{background}
\pgfdeclarelayer{foreground}
\pgfsetlayers{background, main, foreground}


%%% colors
\colorlet{arabidopsis}{OrangeRed1}
\colorlet{Arabidopsis}{arabidopsis}
\colorlet{maize}{Chartreuse3}
\colorlet{Maize}{maize}
\colorlet{sorghum}{SteelBlue1}
\colorlet{Sorghum}{sorghum}
\colorlet{35S enhancer}{DodgerBlue1}
\colorlet{leafCol}{Green4}
\colorlet{protoCol}{Goldenrod1}


%%% save and use png for externalization
\tikzset{
	% Defines a custom style which generates BOTH, .pdf and .png export
	% but prefers the .png on inclusion.
	png export/.style = {
		external/system call/.add = {}{%
			&& pdftocairo -png -r 300 -transp -singlefile "\image.pdf" "\image" %
		},
		/pgf/images/external info,
		/pgf/images/include external/.code = {%
			\includegraphics[width=\pgfexternalwidth,height=\pgfexternalheight]{##1.png}%
		},
	}
}

%%%%%%%%%%%%%%%%%%%%%%%%
%%% drawing commands %%%
%%%%%%%%%%%%%%%%%%%%%%%%
\makeatletter
\tikzset{
	jitter/.style = {shift = {(rand * #1, rand * #1)}},
	jitter/.default = 0.1cm,
	arrow/.style = {very thick, -Latex},
	chloroplast/.style = {Chartreuse3, fill opacity = .25 + rnd * .25, draw opacity = 0.5 + rnd * .5, rotate = rand * 360, x radius = .1 + rand * .03, y radius = .06 + rand * 0.02},
	barcodeStyle/.style = {Orchid1},
	barcode/.code = {\pgfkeysalso{barcodeStyle/.style = {#1}}},
	promoterStyle/.style = {DarkSeaGreen3},
	promoter/.code = {\pgfkeysalso{promoterStyle/.style = {#1}}},
	enhancerStyle/.style = {Dodgerblue1},
	enhancer/.code = {\pgfkeysalso{enhancerStyle/.style = {#1}}},
	promoterStrength/.style = {semithick},
	promoter strength/.is choice,
	promoter strength/very weak/.code = {\pgfkeysalso{promoterStrength/.style = {very thin}}},
	promoter strength/weak/.code = {\pgfkeysalso{promoterStrength/.style = {thin}}},
	promoter strength/normal/.code = {\pgfkeysalso{promoterStrength/.style = {semithick}}},
	promoter strength/strong/.code = {\pgfkeysalso{promoterStrength/.style = {thick}}},
	promoter strength/very strong/.code = {\pgfkeysalso{promoterStrength/.style = {very thick}}}
}
\makeatother

%%% STARR-seq construct
% use:	\STARRconstruct[promoter = ..., enhancer = ..., promoter strength = ..., barcode = ...]{<position>}
\newcommand{\STARRconstruct}[2][]{
	\begin{scope}[#1]
		\draw[dashed, thick] (#2) -- ++(.5, 0) coordinate (c1) {};
		\path (c1) ++(4, 0) coordinate (c2) {};
		\draw[dashed, thick] (c2) -- ++(.5, 0) coordinate (construct end);
		
		\fill[promoterStyle] (c1) ++(1, -.1) rectangle ++(.3, .2) ++(0, -.1) coordinate (c3);
		\draw[-{Stealth[round]}, promoterStrength] (c3) ++(.5\pgflinewidth, 0) |- ++(.4,.3);
		
		\node[draw = black, thin, anchor = west, minimum width = 1cm] (ORF) at ($(c3) + (.4, 0)$) {GFP};
		
		\begin{pgfonlayer} {background}
			\fill[barcodeStyle] (ORF.north west) ++(.15, 0) rectangle ($(ORF.south west) + (.05, 0)$);
		\end{pgfonlayer}
		
		\node[draw = black, thin, anchor = west, text depth = 0pt] (polyA) at ($(ORF.east) + (.2, 0)$) {pA site};
		
		\draw[line width = .2cm, -Triangle Cap, enhancerStyle]  (c1) ++(.2, 0) -- ++(.6, 0);
		
		\begin{pgfonlayer} {background}
			\draw[thick] (c1) -- (ORF.west) (ORF.east) -- (polyA.west) (polyA.east) -- (c2);
		\end{pgfonlayer}
	
		\end{scope}
}

\newcommand{\STARRconstructShort}[2][]{
	\begin{scope}[#1]
		\draw[dashed, thick] (#2) -- ++(.5, 0) coordinate (c1) {};
		\path (c1) ++(2.7, 0) coordinate (c2) {};
		\draw[dashed, thick] (c2) -- ++(.5, 0) coordinate (construct end);
		
		\fill[promoterStyle] (c1) ++(.2, -.1) rectangle ++(.3, .2) ++(0, -.1) coordinate (c3);
		\draw[-{Stealth[round]}, promoterStrength] (c3) ++(.5\pgflinewidth, 0) |- ++(.4,.3);
		
		\node[draw = black, thin, anchor = west, minimum width = 1cm] (ORF) at ($(c3) + (.4, 0)$) {GFP};
		
		\begin{pgfonlayer} {background}
			\fill[barcodeStyle] (ORF.north west) ++(.15, 0) rectangle ($(ORF.south west) + (.05, 0)$);
		\end{pgfonlayer}
		
		\node[draw = black, thin, anchor = west, text depth = 0pt] (polyA) at ($(ORF.east) + (.2, 0)$) {pA};
		
		\draw[line width = .2cm, -Triangle Cap, enhancerStyle]  (c1) ++(.2, 0) -- ++(.6, 0);
		
		\begin{pgfonlayer} {background}
			\draw[thick] (c1) -- (ORF.west) (ORF.east) -- (polyA.west) (polyA.east) -- (c2);
		\end{pgfonlayer}
	
		\end{scope}
}


%%% mRNA transcript
% use:	\transcript[barcode = ...]{<position>}
\newcommand{\transcript}[2][]{%
	\begin{scope}[#1]
		
		\draw[thick] (#2) -- ++(.4, 0) node[fill = white, anchor = west, minimum width = 1cm, outer sep = 0pt] (ORF) {GFP};
		
		\node[anchor = west, inner sep = .15em] (polyA) at ($(ORF.east) + (.2, 0)$) {AAAAAA};
		
		\draw[thick] (ORF.east) -- (polyA.west);
	
		\fill[barcodeStyle] (ORF.north west) ++(.15, 0) rectangle ($(ORF.south west) + (.05, 0)$);
		\draw[thin] (ORF.north west) rectangle (ORF.south east);
	\end{scope}
}


%%% infiltrated leaf
% use:	\leaf[<scale factor>]{<position>}
\newcommand{\leaf}[2][1]{%
	\coordinate (leaf center) at (#2);
	\begin{scope}[scale = #1, rotate = -45]
	
		\path (leaf center) + (-.28, 2.122) coordinate (tip);
	
		\path (tip) ++(-1.512, -2.036) coordinate (c1) ++(1.495, -1.482) coordinate (c2) ++(0.115, -0.077) coordinate (c3) ++(0.091, -0.648) coordinate (c4) ++(0.197, 0) coordinate (c5) ++(-0.088, 0.624) coordinate (c6) ++(0.144, 0.107) coordinate (c7) ++(1.58, 1.514) coordinate (c8);
		
		\filldraw[draw = Chartreuse4, fill = Chartreuse3] (c5)
			.. controls +(-0.052, 0.182) and ($(c5) + (-0.088, 0.354)$) .. (c6)
			.. controls +(0, 0.076) and ($(c6) + (0.028, 0.149)$) .. (c7)
			.. controls +(0.508, -0.162) and ($(c7) + (1.957, -0.273)$) .. (c8)
			.. controls +(-0.424, 2.007) and ($(c8) + (-1.637, 1.595)$) .. (tip)
			.. controls +(-0.255, -0.410) and ($(tip) + (-1.446, -0.729)$) .. (c1) 
			.. controls +(-0.064, -1.265) and ($(c1) + (0.359, -1.908)$) .. (c2)
			.. controls +(0.057, 0.022) and ($(c2) + (0.115, 0.012)$) .. (c3)
			.. controls +(0, -0.203) and ($(c3) + (0.025, -0.458)$) .. (c4)
		;
		
		\coordinate (stem) at ($(c4)!.5!(c5)$);
			
		\path (stem) ++(-0.079, 0.78) coordinate (c9) ++(0.014, 0.148) coordinate (c10) ++(0.017, 0.548) coordinate (c11) ++(0, 0.15) coordinate (c12) ++(-0.036, 0.942) coordinate (c13) ++(-0.015, 0.21) coordinate (c14) ++(-0.13, 1.023) coordinate (c15);
		
		\draw[Chartreuse2, thin, line cap = round] (c15) .. controls +(0.071, -0.366) and ($(c15) + (0.106, -0.707)$) .. (c14);
		\draw[Chartreuse2] (c14) .. controls +(0.005, -0.072)  and ($(c14) + (0.01, -0.142)$) .. (c13);
		\draw[Chartreuse2] (c13) .. controls +(0.02, -0.315) and ($(c13) + (0.034, -0.632)$) .. (c12);
		\draw[Chartreuse2, thick] (c12) .. controls +(0, -0.051) and ($(c12) + (0, -0.101)$) .. (c11);
		\draw[Chartreuse2, thick] (c11) .. controls +(-0.002, -0.185) and ($(c11) + (-0.006, -0.378)$) .. (c10);
		\draw[Chartreuse2, very thick] (c10) .. controls +(-0.003, -0.05) and ($(c10) + (-0.009, -0.099)$) .. (c9);
		\draw[Chartreuse2, very thick, line cap = round] (c9) .. controls +(-0.025, -0.249) and ($(c9) + (0.013, -0.542)$) .. (stem);
		
		\draw[Chartreuse2, line cap = round, thick] (stem) ++(1.462, 2.097) coordinate (c16) .. controls +(-0.2, -1.143) and ($(c16) + (-1.525, -0.933)$) .. ++(-1.527, -1.169) ++(1.064, 1.937) coordinate (c17);
		\draw[Chartreuse2, line cap = round] (c17) .. controls +(-0.312, -1.002) and ($(c17) + (-1.047, -0.998)$) .. ++(-1.047, -1.239) ++(0.4807, 1.856) coordinate (c18);
		\draw[Chartreuse2, line cap = round, thin] (c18) .. controls +(-0.242, -0.562) and ($(c18) + (-0.544, -0.553)$) .. ++(-0.531, -0.704) ++(-0.563, 0.624) coordinate (c19);
		\draw[Chartreuse2, line cap = round, thin] (c19) .. controls +(0.082, -0.422) and ($(c19) + (0.564, -0.624)$) .. ++(0.578, -0.834) ++(-0.975, 0.286) coordinate (c20);
		\draw[Chartreuse2, line cap = round] (c20) .. controls +(0.019, -0.88) and ($(c20) + (1.006, -1.128)$) .. ++(1.011, -1.378) ++(-1.487, 0.529) coordinate (c21);
		\draw[Chartreuse2, line cap = round, thick] (c21) .. controls +(0.034, -1.107) and ($(c21) + (1.484, -0.824)$) .. ++(1.456, -1.225);
		
		
		\fill[Chartreuse4, opacity = .5, decoration={random steps, segment length = #1 * 0.075cm, amplitude = #1 * 0.075cm}, decorate] (tip)  ++(-0.2, -2) circle (0.8);
		
	\end{scope}
}

% save in a box
\newsavebox{\leafbox}
\savebox{\leafbox}{%
	\tikzset{external/optimize=false}%
	\begin{tikzpicture}%
		\path[use as bounding box] (-.5175, .485) rectangle ++(1, -1);%
		\leaf[0.25]{0, 0}%
	\end{tikzpicture}%
}

\newcommand{\leafsymbol}[2][(.35, -.35)]{
	\node[inner sep = 0pt, shift = {#1}] at (#2) {\scalebox{0.5}{\usebox{\leafbox}}}; 
}


%%% protoplast
% use:	\protoplast[<scale factor>]{<position>}
\newcommand{\protoplast}[2][1]{%
	\coordinate (protoplast) at (#2);
	\begin{scope}[scale = #1]
		\shadedraw[Chartreuse4, outer color = Chartreuse2, inner color = white] (protoplast) circle (.4);
		
		\begin{scope}
			\clip (protoplast) circle (.4);	
			\foreach \i in {1, ..., 20} {
				\coordinate[jitter = .4cm] (cp\i) at (protoplast);
				\filldraw[chloroplast] (cp\i) ellipse;
			}
		\end{scope}
	\end{scope}
}

% save in a box
\newsavebox{\protobox}
\savebox{\protobox}{%
	\tikzset{external/optimize=false}%
	\begin{tikzpicture}%
		\path[use as bounding box] (-.5, .5) rectangle ++(1, -1);
		\protoplast[1.15]{0, 0}%
	\end{tikzpicture}%
}

\newcommand{\protosymbol}[2][(.35, -.35)]{
	\node[inner sep = 0pt, shift = {#1}] at (#2) {\scalebox{0.5}{\usebox{\protobox}}}; 
}


%%% sun
% use: \sun[<scale factor>]{<position>}
\newcommand{\sun}[2][1]{%
	\coordinate (sun) at (#2);
	\begin{scope}[scale = #1]
		\fill[Gold1] (sun) circle (.2);
		\foreach \x in {1, ..., 6} {
			\draw[very thick, Gold1] (sun) ++(\x * 60: .25) -- ++(\x * 60: .125);
			\draw[very thick, Gold1] (sun) ++(\x * 60 + 30: .25) -- ++(\x * 60 + 30: .15);
		}
	\end{scope}
}

% save in a box
\newsavebox{\lightbox}
\savebox{\lightbox}{%
	\tikzset{external/optimize=false}%
	\begin{tikzpicture}%
		\path[use as bounding box] (-.5, .5) rectangle ++(1, -1);
		\sun[1.15]{0, 0}%
	\end{tikzpicture}%
}

\newcommand{\lightsymbol}[2][(.35, -.35)]{
	\node[inner sep = 0pt, shift = {#1}] at (#2) {\scalebox{0.5}{\usebox{\lightbox}}}; 
}


%%% set compatibility level
\pgfplotsset{compat = 1.17}

%%% length, counters, booleans, ...
\newlength{\plotwidth}
\newlength{\plotheight}
\newlength{\plotylabelwidth}
\newlength{\plotxlabelheight}
\newlength{\groupplotsep}

\setlength{\plotylabelwidth}{2.8em}
\setlength{\plotxlabelheight}{2.8em}
\setlength{\groupplotsep}{1mm}

\newcounter{groupplot}


%%% general setup
\usepgfplotslibrary{statistics, groupplots, colormaps}

\pgfplotsset{
	table/col sep = tab,
	every axis/.append style = {
		alias = last plot,
		thin,
		scale only axis = true,
		width = \fourcolumnwidth - \plotylabelwidth,
		height = 3.5cm,
		legend style = {node font = \figsmall},
		tick label style = {node font = \figsmall},
		tick align = outside,
		label style = {node font = \fignormal},
		title style = {node font = \fignormal},
		every axis title shift = 0pt,
		max space between ticks = 20,
		major tick length = 0.1cm,
		minor tick length = 0.066cm,
		ticklabel style = {
			/pgf/number format/fixed,
		},
		every tick/.append style = {black, thin},
		tickwidth = .75mm,
		every x tick label/.append style = {align = center, inner xsep = 0pt},
		every y tick label/.append style = {inner ysep = 0pt},
		scaled ticks = false,
		mark/.default = solido,
		mark size = .5,
		grid = both,
		grid style = {very thin, gray!20},
		tick pos = lower,
		x grids/.is choice,
		x grids/true/.style = {xmajorgrids = true, xminorgrids = true},
		x grids/false/.style = {xmajorgrids = false, xminorgrids = false},
		y grids/.is choice,
		y grids/true/.style = {ymajorgrids = true, yminorgrids = true},
		y grids/false/.style = {ymajorgrids = false, yminorgrids = false},
		boxplot/whisker extend={\pgfkeysvalueof{/pgfplots/boxplot/box extend}*0.5},
		boxplot/every median/.style = thick,
		title style = {minimum width = \plotwidth, fill = gray!20, draw = black, yshift = -.5\pgflinewidth, name = title, text depth = .15\baselineskip, align = center},
		legend cell align = left,
		axis background/.style = {fill = white},
		line plot/.style = {semithick, line join = round},
		unbounded coords = jump,
		boxplot/draw direction = y,
		xticklabel style = {align = center}
	},
	x decimals/.style = {x tick label style = {/pgf/number format/.cd, fixed, zerofill, precision = #1}},
	y decimals/.style = {y tick label style = {/pgf/number format/.cd, fixed, zerofill, precision = #1}},
	zero line/.style = {execute at begin axis = {\draw[very thin, #1] (\xmin, 0) -- (\xmax, 0);}},
	zero line/.default = black
}


%%% extra markers
% round marker without border
\pgfdeclareplotmark{solido}{%
  \pgfpathcircle{\pgfpointorigin}{\pgfplotmarksize + .5\pgflinewidth}%
  \pgfusepathqfill
}


%%% commands for min and max coordinates
\def\xmin{\pgfkeysvalueof{/pgfplots/xmin}}
\def\xmax{\pgfkeysvalueof{/pgfplots/xmax}}
\def\ymin{\pgfkeysvalueof{/pgfplots/ymin}}
\def\ymax{\pgfkeysvalueof{/pgfplots/ymax}}


%%% add jitter to points of a scatter plot
\pgfplotsset{
	x jitter/.style = {
		x filter/.expression = {x + rand * #1}
	},
	y jitter/.style = {
		y filter/.expression = {y + rand * #1}
	},
	x jitter/.default = 0.25,
	y jitter/.default = 0.25
}


%%% set x axis limits and tick labels (depending on sample number)
% use:	x limits = {<number of samples>}
%				x tick table = {<table>}{<column>}
% for half violin plots use:	x tick table half = {<table>}{<column>}
\pgfplotsset{
	x limits/.style = {xmin = 1 - 0.6, xmax = #1 + 0.6},
	y limits/.style = {ymin = 1 - 0.6, ymax = #1 + 0.6},
	x tick table/.code 2 args= {\getrows{#1}},
	x tick table/.append style = {%
		x limits = {\datarows},
		xtick = {1, ..., \datarows},
		xticklabels from table = {#1}{#2}
	},
	x tick table half/.code 2 args= {\getrows{#1}},
	x tick table half/.append style = {%
		x limits = {\datarows},
		xtick = {1.5, 3.5, ..., \datarows},
		xticklabel = {\pgfmathint{\tick}\pgfplotstablegetelem{\pgfmathresult}{#2}\of{#1}\pgfplotsretval}
	},
	y tick table/.code 2 args= {\getrows{#1}},
	y tick table/.append style = {%
		y limits = {\datarows},
		ytick = {1, ..., \datarows},
		yticklabels from table = {#1}{#2}
	},
	y tick table half/.code 2 args= {\getrows{#1}},
	y tick table half/.append style = {%
		y limits = {\datarows},
		ytick = {1.5, 3.5, ..., \datarows},
		yticklabel = {\pgfmathint{\tick}\pgfplotstablegetelem{\pgfmathresult}{#2}\of{#1}\pgfplotsretval}
	},
	xymin/.style = {xmin = #1, ymin = #1},
	xymax/.style = {xmax = #1, ymax = #1},
	xytick/.style = {xtick = {#1}, ytick = {#1}}
}



%%% horizontal groupplot
% use: \begin{hgroupplot}[<axis options>]{<total width>}{<number of columns>}{<xlabel>} <plot commands> \end{mygroupplot}
\newenvironment{hgroupplot}[4][]{%
	\pgfmathsetlength{\plotwidth}{(#2 - \plotylabelwidth - ((#3 - 1) * \groupplotsep) ) / #3}%
	\ifthenelse{\isempty{#4}}{%
		\def\xlabel{}%
	}{%
		\def\xlabel{%
			\path let \p1 = (group c1r1.west), \p2 = (group c#3r1.east), \p3 = (xlabel.base) in node[anchor = base] (plot xlabel) at ($(\x1, \y3)!.5!(\x2, \y3)$) {#4};%
		}%
	}
	\begin{groupplot}[
		width = \plotwidth,
		group style = {
			horizontal sep = \groupplotsep,
			y descriptions at = edge left,
			x descriptions at = edge bottom,
			columns = #3,
			rows = 1,
			every plot/.append style = {first plot, after first plot h}
		},
		xlabel = \phantom{#4},
		xlabel style = {name = xlabel},
		#1
	]
}{%
	\end{groupplot}%
	\xlabel%
}


\newenvironment{vgroupplot}[4][]{%
	\pgfmathsetlength{\plotheight}{(#2 - \plotxlabelheight - ((#3 - 1) * \groupplotsep) ) / #3}%
	\ifthenelse{\isempty{#4}}{%
		\def\ylabel{}%
	}{%
		\def\ylabel{%
			\path let \p1 = (group c1r1.north), \p2 = (group c1r#3.south), \p3 = (ylabel.base) in node[anchor = base, rotate = 90] (plot ylabel) at ($(\x3, \y1)!.5!(\x3, \y2)$) {#4};%
		}%
	}
	\setcounter{groupplot}{0}
	\begin{groupplot}[
		height = \plotheight,
		group style = {
			vertical sep = \groupplotsep,
			y descriptions at = edge left,
			x descriptions at = edge bottom,
			columns = 1,
			rows = #3,
			every plot/.append style = {no inner x ticks, first plot, after first plot v, if last plot v = #3, last plot v}
		},
		ylabel = \phantom{#4},
		ylabel style = {name = ylabel},
		#1
	]
}{%
	\end{groupplot}%
	\ylabel%
}

% required styles
\pgfplotsset{
	no inner y ticks/.style = {ymajorticks = false, yminorticks = false},
	no inner x ticks/.style = {xmajorticks = false, xminorticks = false},
	outer x ticks/.style = {xmajorticks = true, xminorticks = true},
	group position/.code = {\pgfkeysalso{first plot/.style = {#1}}},
	after first plot h/.code = {\gpgfplotsset{first plot/.style = no inner y ticks}},
	after first plot v/.code = {\gpgfplotsset{first plot/.style = {}}},
	if last plot v/.code = {\stepcounter{groupplot}\ifnum\value{groupplot}=#1\pgfkeysalso{last plot v/.style = {outer x ticks}}\fi},
	last plot v/.style = {}
}


%%% legend setup
\pgfplotsset{
	legend image code/.code = {
		\draw [mark repeat = 2,mark phase = 2, #1]
			plot coordinates {
				(0em, 0em)
				(.75em, 0em)
				(1.5em, 0em)
			};
	},
	/pgfplots/ybar legend/.style = {
		/pgfplots/legend image code/.code = {
			\draw [##1, /tikz/.cd, bar width = 2pt, yshift = -0.3em, bar shift = 0pt]
			plot coordinates {
				(0em, 0.8em)
				(\pgfplotbarwidth, 1.5em)
				(2*\pgfplotbarwidth, 0.6em)
			};
		},
	},
	/pgfplots/xbar legend/.style = {
		/pgfplots/legend image code/.code = {
			\draw [##1, /tikz/.cd, bar width = 2pt, yshift = -0.2em, bar shift = 0pt]
			plot coordinates {
				(0.8em, 0em)
				(1.5em, \pgfplotbarwidth)
				(0.6em, 2*\pgfplotbarwidth)
			};
		},
	},
	/pgfplots/xbar legend 1/.style = {
		/pgfplots/legend image code/.code = {
			\draw [##1, fill = gray, /tikz/.cd, bar width = 2pt, yshift = -0.2em, bar shift = 0pt]
			plot coordinates {
				(.8em, 0em)
				(.6em, \pgfplotbarwidth)
			};
			\draw [##1, /tikz/.cd, bar width = 2pt, yshift = -0.2em, bar shift = 0pt]
			plot coordinates {
				(1.5em, 2*\pgfplotbarwidth)
			};
		},
	},
	/pgfplots/xbar legend 2/.style = {
		/pgfplots/legend image code/.code = {
			\draw [##1, fill = gray, /tikz/.cd, bar width = 2pt, yshift = -0.2em, bar shift = 0pt]
			plot coordinates {
				(.8em, 0em)
				(.6em, 2*\pgfplotbarwidth)
			};
			\draw [##1, /tikz/.cd, bar width = 2pt, yshift = -0.2em, bar shift = 0pt]
			plot coordinates {
				(1.5em, \pgfplotbarwidth)
			};
		},
	},
	/pgfplots/xbar legend 3/.style = {
		/pgfplots/legend image code/.code = {
			\draw [##1, /tikz/.cd, bar width = 2pt, yshift = -0.2em, bar shift = 0pt]
			plot coordinates {
				(1.5em, 0em)
			};
			\draw [##1, fill = gray, /tikz/.cd, bar width = 2pt, yshift = -0.2em, bar shift = 0pt]
			plot coordinates {
				(.6em, \pgfplotbarwidth)
				(.8em, 2*\pgfplotbarwidth)
			};
		},
	},
	/pgfplots/density legend/.style = {
		/pgfplots/legend image code/.code = {
			\draw [##1, x = .25em, y = .25em, yshift = -0.3em]	plot[domain = 0:6] (\x,{4*1/exp(((\x-3)^2)/2)});
		},
	},
}


%%% helper for global pgfplot style definitions
\newcommand\gpgfplotsset[1]{%
	\begingroup%
		\globaldefs=1\relax%
		\pgfqkeys{/pgfplots}{#1}%
	\endgroup%
}


%%% get number of rows/columns in a table -> stored in `\datarows`/`\datacols`
% use: \getrows{<table>}
\newcommand{\getrows}[1]{%
	\pgfplotstablegetrowsof{#1}%
	\pgfmathsetmacro{\datarows}{\pgfplotsretval}%
}

\newcommand{\getcols}[1]{%
	\pgfplotstablegetcolsof{#1}%
	\pgfmathsetmacro{\datacols}{\pgfplotsretval}%
}

%%% variants of `\pgfplotsinvokeforeach` that iterates from 1 to the number of rows/columns in the table
% use:	\foreachtablerow{<table or file>}{<command (use `#1` to get current iterator)>}
\makeatletter
	\long\def\foreachtablerow#1#2{%
		\getrows{#1}
		\long\def\pgfplotsinvokeforeach@@##1{#2}%
		\pgfplotsforeachungrouped \pgfplotsinvokeforeach@ in {1, ..., \datarows} {%
			\expandafter\pgfplotsinvokeforeach@@\expandafter{\pgfplotsinvokeforeach@}%
		}%
	}
	
	\long\def\foreachtablecol#1#2{%
		\getcols{#1}
		\long\def\pgfplotsinvokeforeach@@##1{#2}%
		\pgfplotsforeachungrouped \pgfplotsinvokeforeach@ in {1, ..., \datacols} {%
			\expandafter\pgfplotsinvokeforeach@@\expandafter{\pgfplotsinvokeforeach@}%
		}%
	}
\makeatother


%%% command to add significance indicator to plot
% usage: \signif[pgf options]{file}{first sample}{second sample}
% useful options: `raise = ...` to raise (or lower) the indicator (connectors stay at the same height; use yshift to also raise them)
%									`shorten both = ...` to increase (or decrease) distance between plot and indicator (default = 2.5)
%									`signif cutoff = ...` to set the significance cutoff
%									`signif levels = {..., ..., ...}` to set the significance levels
%									`bar only` to not draw connectors (only the horizontal bar)
%									`draw = none` to not draw any line 
\makeatletter
	\newcommand{\signif}{%
		\@ifstar
			\signifStar%
			\signifNoStar%
	}
\makeatother

\newcommand{\signifNoStar}[4][]{%
	\addplot [no marks, shorten both = 2.5, point meta = \thisrow{p.value.#3_#4}, nodes near coords = \printsignif, nodes near coords style = {node font = \figsmall}, #1] table [x = x.#3_#4, y = y.#3_#4] {#2};
}
\newcommand{\signifStar}[4][]{%
	\addplot [no marks, shorten both = 2.5, point meta = \thisrow{p.value.#3_#4}, nodes near coords = \printsignif*, nodes near coords style = {node font = \figsmall}, #1] table [x = x.#3_#4, y = y.#3_#4] {#2};
}


%% significance for all pairs (especially usefull for half violin plots or plots with selected pvalues only)
% usage: \signif(*)[pgf options]{file}
\makeatletter
	\newcommand{\signifall}{%
		\@ifstar
			\signifallStar%
			\signifallNoStar%
	}
\makeatother

\newcommand{\signifallStar}[2][]{%
	\getcols{#2}%
	\pgfmathsetmacro{\samplen}{\datacols / 3}%
	\pgfplotsinvokeforeach{1, ..., \samplen}
	{%
		\pgfmathsetmacro{\xcol}{int(##1 - 1)}%
		\pgfmathsetmacro{\ycol}{int(\xcol + \samplen)}%
		\pgfmathsetmacro{\pcol}{int(\ycol + \samplen)}%
		\addplot [no marks, shorten both = 2.5, point meta = \thisrowno{\pcol}, nodes near coords = \printsignif*, nodes near coords style = {node font = \figsmall}, #1] table [x index = \xcol, y index = \ycol] {#2};%
	}
}

\newcommand{\signifallNoStar}[2][]{%
	\getcols{#2}%
	\pgfmathsetmacro{\samplen}{\datacols / 3}%
	\pgfplotsinvokeforeach{1, ..., \samplen}
	{%
		\pgfmathsetmacro{\xcol}{int(##1 - 1)}%
		\pgfmathsetmacro{\ycol}{int(\xcol + \samplen)}%
		\pgfmathsetmacro{\pcol}{int(\ycol + \samplen)}%
		\addplot [no marks, shorten both = 2.5, point meta = \thisrowno{\pcol}, nodes near coords = \printsignif, nodes near coords style = {node font = \figsmall}, #1] table [x index = \xcol, y index = \ycol] {#2};%
	}
}


%% significance for simple comparisons (compared to zero distribution)
% use:	\signifallsimple(*)[<pgfplots options>]{<table>}{<column with x coords>}{<column with p-values>}
\makeatletter
	\newcommand{\signifallsimple}{%
		\@ifstar
			\signifallsimpleStar%
			\signifallsimpleNoStar%
	}
\makeatother

\newcommand{\signifallsimpleStar}[4][]{%
	\addplot [no marks, draw = none, point meta = \thisrow{#4}, nodes near coords = \vphantom{A}\printsignif*, nodes near coords style = {node font = \figsmall, anchor = north}, #1] table [x = #3, y expr = \pgfkeysvalueof{/pgfplots/ymax}] {#2};%
}

\newcommand{\signifallsimpleNoStar}[4][]{%
	\addplot [no marks, draw = none, point meta = \thisrow{#4}, nodes near coords = \printsignif, nodes near coords style = {node font = \figsmall, anchor = north}, #1] table [x = #3, y expr = \pgfkeysvalueof{/pgfplots/ymax}] {#2};%
}


%% use point meta data to print p-values (the * version prints significance levels)
% use as: point meta = \thisrow{<p-value column>}, nodes near coords = \printsignif<*>[<cutoff (one value)/signif. levels (three values separated by ",")>] 
\makeatletter
	\newcommand{\printsignif}{%
		\@ifstar
			\printsignifStar%
			\printsignifNoStar%
	}
\makeatother

\newcommand{\printsignifStar}[1][\signiflevelone,\signifleveltwo,\signiflevelthree]{%
	\pgfmathfloatifflags{\pgfplotspointmeta}{3}{}{\pgfmathfloattosci{\pgfplotspointmeta}\tolevel#1,{\pgfmathresult}}%
}
\newcommand{\printsignifNoStar}[1][\signifcutoff]{%
	\pgfmathfloatifflags{\pgfplotspointmeta}{3}{}{\pgfmathfloattosci{\pgfplotspointmeta}\ifsignif#1,{\pgfmathresult}}%
}

% abbreviation for non-significant p-values and symbol for significance level
\newcommand{\notsignif}{ns}
\newcommand{\issignif}{{\signiffont \symbol{"2217}}}

%% default significance cutoff/levels
% 1 to \signifcutoff or \signiflevelone: ns
% \signiflevelone to \signifleveltwo: *
% \signifleveltwo to \signiflevelthree: **
% \signiflevelthree to 0: ***
\def\signifcutoff{0.01}
\def\signiflevelone{0.01}
\def\signifleveltwo{0.001}
\def\signiflevelthree{0.0001}

% convert p-value to significance level
\def\tolevel#1,#2,#3,#4{%
	\pgfmathparse{#4 <= #3 ? "\issignif\issignif\issignif" : (#4 <= #2 ? "\issignif\issignif" : (#4 <= #1 ? "\issignif" : "\notsignif"))}\pgfmathresult%
}

% print \notsignif for non-significant p-values
\def\ifsignif#1,#2{%
	\pgfmathparse{#2 <= #1 ? "\pgfmathprintnumber{#2}" : "\notsignif"}\pgfmathresult%
}

% useful styles
\makeatletter
\tikzset{
	bar only/.style = jump mark mid,
	shorten both/.style = {
		shorten < = #1,
		shorten > = #1
	},
	raise/.code = {
		\pgfkeysalso{yshift = #1}
		\pgfmathaddtolength\pgf@shorten@start@additional{-#1}
		\pgfmathaddtolength\pgf@shorten@end@additional{-#1}
	},
	signif cutoff/.code = {
		\def\signifcutoff{#1}
	},
	signif levels/.code args = {#1,#2,#3}{
		\def\signiflevelone{#1}
		\def\signifleveltwo{#2}
		\def\signiflevelthree{#3}
	}
}
\makeatother


%%% style to prepare boxplot from summarized data in a table
% table layout: one row per sample; required columns and content: lw, lower whisker; lq, lower quartile; med, median; uq, upper quartile; uw, upper whisker
% usage: boxplot from table = {table macro}{row number (0-based)} 
\makeatletter
\pgfplotsset{
    boxplot prepared from table/.code={
        \def\tikz@plot@handler{\pgfplotsplothandlerboxplotprepared}%
        \pgfplotsset{
            /pgfplots/boxplot prepared from table/.cd,
            #1,
        }
    },
    /pgfplots/boxplot prepared from table/.cd,
        table/.code={
        	\pgfplotstable@isloadedtable{#1}%
        		{\pgfplotstablecopy{#1}\to\boxplot@datatable}%
        		{\pgfplotstableread{#1}\boxplot@datatable}%
        },
        row/.initial=1,
        make style readable from table/.style={
            #1/.code={
            		\pgfmathint{\pgfkeysvalueof{/pgfplots/boxplot prepared from table/row} - 1}
                \pgfplotstablegetelem{\pgfmathresult}{##1}\of\boxplot@datatable
                \pgfplotsset{boxplot/#1/.expand once={\pgfplotsretval}}
            }
        },
        make style readable from table=lower whisker,
        make style readable from table=upper whisker,
        make style readable from table=lower quartile,
        make style readable from table=upper quartile,
        make style readable from table=median
}
\makeatother

\pgfplotsset{
	boxplot from table/.style 2 args = {%
		boxplot prepared from table = {%
			table = #1,
			row = #2,
			lower whisker = lw,
			lower quartile = lq,
			median = med,
			upper quartile = uq,
			upper whisker = uw
		}
	}
}


%%% draw a combined violin and box plot
% use `violin shade (inverse) = 0` to not shade the violin plots
% use: \violinbox[<pgfplots options for violin and box plots>]{<table for boxplot>}{<file for violin plot>}
\newcommand{\violinbox}[3][]{%
	\foreachtablerow{#2}{%
		\violinplot[save row = {##1}, violin shade, #1]{#3}{##1};%
		\boxplot[save row = {##1}, boxplot/box extend = {\pgfkeysvalueof{/pgfplots/violin extend} * 0.2}, boxplot/whisker extend = 0, #1]{#2}{##1};%
	}%
}

%%% draw a combined half violin and box plot
% use: \halfviolinbox[<pgfplots options for violin and box plots>]{<table for boxplot>}{<file for violin plot>}
\newcommand{\halfviolinbox}[3][]{%
	\foreachtablerow{#2}{%
		\ifthenelse{\isodd{##1}}{%
			\halfviolinplotleft[save row = {##1}, #1]{#3}{##1};%
			\boxplot[save row = {##1}, boxplot/box extend = {\pgfkeysvalueof{/pgfplots/violin extend} * 0.3}, boxplot/whisker extend = 0, boxplot/draw position = ##1 + 0.45, boxplot/draw relative anchor = 1, #1]{#2}{##1};%
		}{%
			\halfviolinplotright[save row = {##1}, #1]{#3}{##1};%
			\boxplot[save row = {##1}, boxplot/box extend = {\pgfkeysvalueof{/pgfplots/violin extend} * 0.3}, boxplot/whisker extend = 0, boxplot/draw position = ##1 - 0.45, boxplot/draw relative anchor = 0, #1]{#2}{##1};%
		}
		
	}%
}

%% draw a violin plot
% use: \violinplot[<pgfplots options>]{<file>}{<sample>}
\newcommand{\violinplot}[3][]{%
	\addplot [black, fill = viocol, fill opacity = .5, #1, viostyle] table [x expr = \thisrow{x.#3} * \pgfkeysvalueof{/pgfplots/violin extend} + #3, y = y.#3] {#2} -- cycle;%
}

%% draw a half violin plot
% use: \violinplot[<pgfplots options>]{<file>}{<sample>}
\newcommand{\halfviolinplotright}[3][]{%
	\addplot [black, fill = viocolright, fill opacity = .5, #1, viostyle] table [x expr = \thisrow{x.#3} * 2 * \pgfkeysvalueof{/pgfplots/violin extend} + #3 - 0.45, y = y.#3] {#2} -- cycle;%
}
\newcommand{\halfviolinplotleft}[3][]{%
	\addplot [black, fill = viocolleft, fill opacity = .5, #1, viostyle] table [x expr = \thisrow{x.#3} * -2 * \pgfkeysvalueof{/pgfplots/violin extend} + #3 + 0.45, y = y.#3] {#2} -- cycle;%
}

%% draw a box plot
% use: \boxplot[<pgfplots options>]{<table>}{<row>}
% does not add outliers! use `\outliers[<pgfplots options>]{<table>}{<sample>}
\newcommand{\boxplot}[3][]{%
	\addplot [black, fill = boxcol, boxplot from table = {#2}{#3}, boxplot/draw position = #3, mark = solido, mark options = black, #1, boxstyle] coordinates {};%
}

\newcommand{\outliers}[3][]{%
	\addplot [black, only marks, mark = solido, #1] table [x expr = #3, y = outlier.#3] {#2};%
}

%%% draw boxplots from table
% use:	\boxplots[<outlier options>]{<boxplot options>}{<table>}{<outlier file>} or
%				\boxplots*[<outlier options>]{<boxplot options>}{<table>} to not draw outliers
\makeatletter
	\newcommand{\boxplots}{%
		\@ifstar
			\boxplotsStar%
			\boxplotsNoStar%
	}
\makeatother

\newcommand{\boxplotsStar}[3][]{%
	\foreachtablerow{#3}{%
		\boxplot[save row = {##1}, #2]{#3}{##1};%
	}%
}

\newcommand{\boxplotsNoStar}[4][]{%
	\foreachtablerow{#3}{%
		\boxplot[save row = {##1}, #2]{#3}{##1};%
		\outliers[#1]{#4}{##1};%
	}%
}


% default colors for violin and box plots
\colorlet{viocol}{gray}
\colorlet{viocolleft}{viocol!50!black}
\colorlet{viocolright}{viocol}
\colorlet{vioshade}{black}
\colorlet{boxcol}{white}
\colorlet{boxshade}{black}

% useful styles for violin and box plots
\pgfplotsset{
	save row/.code = {\def\currrow{#1}},
	viostyle/.style = {},
	violin extend/.initial = 0.9,
	violin color/.code = {\colorlet{viocol}{#1}},
	violin color left/.code = {\colorlet{viocolleft}{#1}},
	violin color right/.code = {\colorlet{viocolright}{#1}},
	violin color half/.code = {\colorlet{viocolleft}{#1!50!black}\colorlet{viocolright}{#1}},
	violin colors half/.code n args = {3}{
		\pgfmathparse{\currrow - 1}
		\ifcase\pgfmathresult
			\colorlet{viocolleft}{#1!50!black}
		\or
			\colorlet{viocolright}{#1}
		\or
			\colorlet{viocolleft}{#2!50!black}
		\or
			\colorlet{viocolright}{#2}
		\or
			\colorlet{viocolleft}{#3!50!black}
		\or
			\colorlet{viocolright}{#3}
		\fi
	},
	violin color half inverse/.code = {\colorlet{viocolleft}{#1}\colorlet{viocolright}{#1!50!black}},
	violin shade color/.code = {\colorlet{vioshade}{#1}},
	violin shade inverse/.code = {\pgfmathparse{100 - (#1 * (\currrow - 1) / (\datarows - 1))}\pgfkeysalso{viostyle/.estyle = {fill = viocol!\pgfmathresult!vioshade}}},
	violin shade inverse/.default = 50,
	violin shade/.code = {\pgfmathparse{100 - (#1 * (1 - (\currrow - 1) / (\datarows - 1)))}\pgfkeysalso{viostyle/.estyle = {fill = viocol!\pgfmathresult!vioshade}}},
	violin shade/.default = 50,
	boxstyle/.style = {},
	box color/.code = {\colorlet{boxcol}{#1}},
	box shade color/.code = {\colorlet{boxshade}{#1}},
	box shade inverse/.code = {\pgfmathparse{100 - (#1 * (\currrow - 1) / (\datarows - 1))}\pgfkeysalso{boxstyle/.estyle = {fill = boxcol!\pgfmathresult!boxshade}}},
	box shade inverse/.default = 50,
	box shade/.code = {\pgfmathparse{100 - (#1 * (1 - (\currrow - 1) / (\datarows - 1)))}\pgfkeysalso{boxstyle/.estyle = {fill = boxcol!\pgfmathresult!boxshade}}},
	box shade/.default = 50,
	sample color/.style = {visualization depends on = value \thisrow{#1} \as \samplecolor}
}


%%% display sample size
% use:	\samplesize[<pgfplots options>]{<table>}{<column with x coords>}{<column with sample size>}
% to shift every second sample size use:	\samplesize[scatter, no marks, visualization depends on = {mod(x, 2) * .5\baselineskip \as \shift}, scatter/@pre marker code/.append style = {/tikz/yshift = \shift}]{<table>}{<column with x coords>}{<column with sample size>}
\newcommand{\samplesize}[4][]{%
	\addplot [black, draw = none, point meta = \thisrow{#4}, sample size, #1] table [x = #3, y expr = \pgfkeysvalueof{/pgfplots/ymin}] {#2};%
}

%% for half violin plots
% use:	\samplesizehalf[<pgfplots options>]{<table>}{<column with x coords>}{<column with sample size>}
% colored by viocolleft and viocolright; adjust with `violin color half = ...`, `violin color left = ...`, and/or `violin color right = ...`
\newcommand{\samplesizehalf}[4][]{%
	\addplot [draw = none, point meta = \thisrow{#4}, sample size, visualization depends on = \thisrow{#3} \as \id, nodes near coords = \twosamplesize, #1] table [x expr = \thisrow{#3} - 0.5, y expr = \pgfkeysvalueof{/pgfplots/ymin}] {#2};
}

% stacked display of two sample sizes
\newcommand{\twosamplesize}{%
	\ifthenelse{\isodd{\id}}{%
		\pgfmathfloattosci{\pgfplotspointmeta}%
		\xdef\lastn{\pgfmathresult}%
	}{%
		\ifcsname samplecolor\endcsname%
			\pgfplotsset{violin color half = \samplecolor}%
		\fi%
		\textcolor{viocolleft}{\pgfmathprintnumber{\lastn}}\\[-.25\baselineskip]%
		\ifcsname samplecolor\endcsname%
			\pgfplotsset{violin color half = \samplecolor}%
		\fi%
		\textcolor{viocolright}{\pgfmathprintnumber{\pgfplotspointmeta}}%
	}%
}

% default style for sample size nodes
\pgfplotsset{
	sample size/.style = {
		nodes near coords,
		nodes near coords style = {
			node font = \figtiny,
			align = center,
			/pgf/number format/1000 sep = {}
		}
	}
}


%%% logo plots
% use:	\logoplot[axis options]{file}
% specify size and position in axis options (width, heigth, at, anchor, ...)
% the width of the letters can be changed with option: letter width = ...
% base colors are changed with option: base colors = {A = ..., C = ..., ...}
% axis styles are changed with opiton: logo axis = none (no axes) / default (normal pgfplot axes) / IC (y axis for information content; no x axis; the default setting) / IC and position (y axis for information content and x axis for position)

\newcommand{\logoplot}[2][]{
	\begin{axis}[
		logo axis,
		#1,
		logo plot
	]
		\addlogoplot{#2};
	\end{axis}
}

\newcommand{\addlogoplot}[1]{%
	\pgfplotsinvokeforeach{1, ..., 4}{%
		\addplot[%
			scatter,
			scatter src = explicit symbolic,
			only marks,
			mark size = \pgfkeysvalueof{/pgfplots/width} / (\xmax - 0.5) * \pgfkeysvalueof{/pgfplots/letter width},
			visualization depends on = \thisrow{IC_##1} * \pgfkeysvalueof{/pgfplots/height} / \ymax \as\baseht
		] table[x = pos,y = IC_##1, meta = base_##1] {#1};%
	}%
}

\colorlet{baseAcol}{Green4}
\colorlet{baseCcol}{Blue2}
\colorlet{baseGcol}{Goldenrod1}
\colorlet{baseTcol}{Red2}
	
\pgfdeclareplotmark{baseA}{%
	\pgftransformxscale{\pgfplotmarksize}
	\pgftransformyscale{\baseht}
	\pgfpathmoveto{\pgfpoint{-.5}{-1}}%
	\pgfpathlineto{\pgfpoint{-.1}{0}}%
	\pgfpathlineto{\pgfpoint{.1}{0}}%
	\pgfpathlineto{\pgfpoint{.5}{-1}}%
	\pgfpathlineto{\pgfpoint{.3}{-1}}%
	\pgfpathlineto{\pgfpoint{.2}{-.75}}%
	\pgfpathlineto{\pgfpoint{-.2}{-.75}}%
	\pgfpathlineto{\pgfpoint{-.3}{-1}}%
	\pgfpathclose
	\pgfpathmoveto{\pgfpoint{.14}{-.6}}%
	\pgfpathlineto{\pgfpoint{0}{-.25}}%
	\pgfpathlineto{\pgfpoint{-.14}{-.6}}%
	\pgfpathclose
	\pgfusepathqfill
}

\pgfdeclareplotmark{baseC}{%
	\pgftransformxscale{\pgfplotmarksize}
	\pgftransformyscale{\baseht}
	\pgfpathmoveto{\pgfpoint{.5}{-.7}}%
	\pgfpathcurveto{\pgfpoint{.4}{-.9}}{\pgfpoint{.3}{-1}}{\pgfpoint{0}{-1}}%
	\pgfpathcurveto{\pgfpoint{-.3}{-1}}{\pgfpoint{-.5}{-.825}}{\pgfpoint{-.5}{-.5}}%
	\pgfpathcurveto{\pgfpoint{-.5}{-.175}}{\pgfpoint{-.3}{0}}{\pgfpoint{0}{0}}%
	\pgfpathcurveto{\pgfpoint{.3}{0}}{\pgfpoint{.4}{-.1}}{\pgfpoint{.5}{-.3}}%
	\pgfpathlineto{\pgfpoint{.3}{-.38}}%
	\pgfpathcurveto{\pgfpoint{.225}{-.25}}{\pgfpoint{.2}{-.15}}{\pgfpoint{0}{-.15}}%
	\pgfpathcurveto{\pgfpoint{-.2}{-.15}}{\pgfpoint{-.3}{-.3}}{\pgfpoint{-.3}{-.5}}%
	\pgfpathcurveto{\pgfpoint{-.3}{-.7}}{\pgfpoint{-.2}{-.85}}{\pgfpoint{0}{-.85}}%
	\pgfpathcurveto{\pgfpoint{.2}{-.85}}{\pgfpoint{.225}{-.75}}{\pgfpoint{.3}{-.62}}%
	\pgfpathclose
	\pgfusepathqfill
}

\pgfdeclareplotmark{baseG}{%
	\pgftransformxscale{\pgfplotmarksize}
	\pgftransformyscale{\baseht}
	\pgfpathmoveto{\pgfpoint{.5}{-.7}}%
	\pgfpathcurveto{\pgfpoint{.4}{-.9}}{\pgfpoint{.3}{-1}}{\pgfpoint{0}{-1}}%
	\pgfpathcurveto{\pgfpoint{-.3}{-1}}{\pgfpoint{-.5}{-.825}}{\pgfpoint{-.5}{-.5}}%
	\pgfpathcurveto{\pgfpoint{-.5}{-.175}}{\pgfpoint{-.3}{0}}{\pgfpoint{0}{0}}%
	\pgfpathcurveto{\pgfpoint{.3}{0}}{\pgfpoint{.4}{-.1}}{\pgfpoint{.5}{-.3}}%
	\pgfpathlineto{\pgfpoint{.3}{-.38}}%
	\pgfpathcurveto{\pgfpoint{.225}{-.25}}{\pgfpoint{.2}{-.15}}{\pgfpoint{0}{-.15}}%
	\pgfpathcurveto{\pgfpoint{-.2}{-.15}}{\pgfpoint{-.3}{-.3}}{\pgfpoint{-.3}{-.5}}%
	\pgfpathcurveto{\pgfpoint{-.3}{-.7}}{\pgfpoint{-.2}{-.85}}{\pgfpoint{0}{-.85}}%
	\pgfpathcurveto{\pgfpoint{.2}{-.85}}{\pgfpoint{.225}{-.75}}{\pgfpoint{.3}{-.62}}%
	\pgfpathclose
	\pgfpathmoveto{\pgfpoint{.5}{-.55}}%
	\pgfpathlineto{\pgfpoint{.5}{-1}}%
	\pgfpathlineto{\pgfpoint{.3}{-1}}%
	\pgfpathlineto{\pgfpoint{.3}{-.7}}%
	\pgfpathlineto{\pgfpoint{0}{-.7}}%
	\pgfpathlineto{\pgfpoint{0}{-.55}}%
	\pgfpathclose
	\pgfusepathqfill
}

\pgfdeclareplotmark{baseT}{%
	\pgftransformxscale{\pgfplotmarksize}
	\pgftransformyscale{\baseht}
	\pgfpathmoveto{\pgfpoint{-.1}{-1}}%
	\pgfpathlineto{\pgfpoint{-.1}{-.15}}%
	\pgfpathlineto{\pgfpoint{-.5}{-.15}}%
	\pgfpathlineto{\pgfpoint{-.5}{0}}%
	\pgfpathlineto{\pgfpoint{.5}{0}}%
	\pgfpathlineto{\pgfpoint{.5}{-.15}}%
	\pgfpathlineto{\pgfpoint{.1}{-.15}}%
	\pgfpathlineto{\pgfpoint{.1}{-1}}%
	\pgfpathclose
	\pgfusepathqfill
}

\pgfplotsset{
	base colors/.code = {
		\pgfkeys{
			/logo plot/.cd,
			#1
		}
	},
	/logo plot/A/.code = {\colorlet{baseAcol}{#1}},
	/logo plot/C/.code = {\colorlet{baseCcol}{#1}},
	/logo plot/G/.code = {\colorlet{baseGcol}{#1}},
	/logo plot/T/.code = {\colorlet{baseTcol}{#1}},
	letter width/.initial = 1,
	logo y axis/.style = {},
	logo x axis/.style = {},
	show IC/.style = {
		logo y axis/.style = {
			ytick = {0, 1, 2},
			ylabel = IC (bits),
			axis y line = left,
			y axis line style = {line cap = round, -},
			ytick align = outside,
			axis y line shift = \pgfkeysvalueof{/pgfplots/major tick length}	
		}
	},
	show pos/.style = {
		logo x axis/.style = {
			axis x line = bottom,
			x axis line style = {draw = none},
			xtick style = {draw = none},
			xtick align = inside
		}
	},
	logo axis/.is choice,
	logo axis/default/.style = {
		logo y axis/.style = {},
		logo x axis/.style = {}
	},
	logo axis/none/.style = {
		logo y axis/.style = {axis y line = none},
		logo x axis/.style = {axis x line = none}
	},
	logo axis/IC/.style = {
		logo x axis/.style = {axis x line = none},
		show IC
	},
	logo axis/IC and position/.style = {
		show IC,
		show pos
	},
	logo axis/.default = IC,
	logo plot/.style = {
		logo y axis,
		logo x axis,
		stack plots = y,
		ymin = 0,
		ymax = 2,
		grid = none,
		enlarge x limits = {abs = .5},
		scatter/classes = {
			A={mark = baseA, baseAcol},
			C={mark = baseC, baseCcol},
			G={mark = baseG, baseGcol},
			T={mark = baseT, baseTcol}
		}
	}
}

%%% table layout
\renewcommand{\arraystretch}{1.33}

\setlength{\aboverulesep}{0pt}
\setlength{\belowrulesep}{0pt}

\newcounter{tblerows}
\expandafter\let\csname c@tblerows\endcsname\rownum % to restore proper behaviour of \rowcolors in tabularx environments

\newcolumntype{L}[1]{>{\raggedright\let\newline\\\arraybackslash\hspace{0pt}}p{#1}}
\newcolumntype{C}[1]{>{\centering\let\newline\\\arraybackslash\hspace{0pt}}p{#1}}
\newcolumntype{R}[1]{>{\raggedleft\let\newline\\\arraybackslash\hspace{0pt}}p{#1}}


%%% declare figure types
\newif\ifmain
\newif\ifsupp

\newcounter{fig}
\newcounter{sfig}
\newcounter{stab}


\newenvironment{fig}{%
	\begin{figure}%
		\stepcounter{fig}%
		\pdfbookmark{\figurename\ \thefig}{figure\thefig}
		\tikzsetnextfilename{figure\thefig}%
		\fignormal%
		\centering%
}{%
		\begin{tikzpicture}[remember picture,overlay]
			\node[anchor=south east, shift={(-2.5cm,1.5cm)}] at (current page.south east) {\normalsize Jores \textit{et al.}, Fig. \thefig};
		\end{tikzpicture}
	\end{figure}%
	\clearpage
}

\newenvironment{sfig}{%
	\begin{figure}[h]%
		\stepcounter{sfig}%
		\setcounter{subfig}{0}%
		\pdfbookmark{\figurename\ \thesfig}{supp_figure\thesfig}
		\tikzsetnextfilename{supp_fig\thesfig}%
		\fignormal%
		\centering%
}{%		
		\begin{tikzpicture}[remember picture,overlay]
			\node[anchor=south east, shift={(-2.5cm,1.5cm)}] at (current page.south east) {\normalsize Jores \textit{et al.}, Supp. Fig. \thesfig};
		\end{tikzpicture}
	\end{figure}%
	\clearpage
}

\newenvironment{stab}{%
	\begin{table}[h]%
		\stepcounter{stab}%
		\pdfbookmark{\tablename\ \thestab}{supp_table\thestab}
		\small%
}{%		
		\begin{tikzpicture}[remember picture,overlay]
			\node[anchor=south east, shift={(-2.5cm,1.5cm)}] at (current page.south east) {\normalsize Jores \textit{et al.}, Supp Table \thestab};
		\end{tikzpicture}
	\end{table}%
	\clearpage
}

%%% subfigure labels
\newif\ifsubfigupper
\subfigupperfalse

\tikzset{
	subfig label/.style = {anchor = north west, inner sep = 0pt, font = \normalsize\bfseries}
}

\newcounter{subfig}[figure]


\newcommand{\subfiglabel}[2][]{
	\node[anchor = north west, inner sep = 0pt, font = \normalsize\bfseries, #1] at (#2) {\strut\stepcounter{subfig}\ifsubfigupper\Alph{subfig}\else\alph{subfig}\fi};
}

\newcommand{\plainsubfigref}[1]{\textbf{\ifsubfigupper\uppercase{#1}\else\lowercase{#1}\fi}}
\newcommand{\subfigref}[1]{\textbf{\plainsubfigref{#1}},}
\newcommand{\subfigrange}[2]{\textbf{\plainsubfigref{#1}}-\textbf{\plainsubfigref{#2}},}
\newcommand{\parensubfig}[2][]{(\textbf{#1\plainsubfigref{#2}})}

%%%% no subfigure labels (for use in presentation)
%\renewcommand{\subfiglabel}[2][]{
%	\node[anchor = north west, inner sep = 0pt, font = \normalsize\bfseries, #1] at (#2) {\strut\stepcounter{subfig}\phantom{\ifsubfigupper\Alph{subfig}\else\alph{subfig}\fi}};
%}


%%% what to include
\maintrue
\supptrue

%%%%%%%%%%%%%%%%%%%%%%%%%%%%%%%%%%%%
%%%   ||    preamble end    ||   %%%r
%%%--\\//------------------\\//--%%%
%%%   \/   begin document   \/   %%%
%%%%%%%%%%%%%%%%%%%%%%%%%%%%%%%%%%%%

\begin{document}
	\sffamily
	\frenchspacing
	
	\renewcommand{\figurename}{Fig.}
	
	%%% Main start
	
	\ifmain
		
		\begin{fig}
			\tikzset{png export}%
			%\tikzset{external/export next = false}

\begin{tikzpicture}
	
	%%% scheme of the assay
	\coordinate (scheme) at (0, 0);

	%%% array
	
	\coordinate (array) at ($(scheme) + (.2, -1)$);
	
	\draw[thin, fill = gray!50] (array) -- ++(3, 0) -- ++(0.6, 0.3) coordinate (array right) -- ++(-3, 0) -- cycle;
	
	\foreach \i/\col [evaluate = \i as \j using 7 - \i] in {1/SeaGreen4, 2/DodgerBlue1, 3/MediumOrchid2, 4/Aquamarine2, 5/Chocolate1, 6/RoyalBlue2} {
		\foreach \k in {1, 2, 3, 4} {
			\draw[\col!75!black, rounded corners = 0.05cm] (array) ++(\i * 0.5 - 0.1 + rand * 0.06, 0.075 + rand * 0.03) -- ++(0, 0.1) -- ++(-0.05, 0.1) -- ++(0.1, 0.2) -- ++(-0.05, 0.1) -- ++(0, 0.1);
			\draw[\col!75!white, rounded corners = 0.05cm] (array) ++(\j * 0.5 + 0.2 + rand * 0.06, 0.225 + rand * 0.03)  -- ++(0, 0.1) -- ++(-0.05, 0.1) -- ++(0.1, 0.2) -- ++(-0.05, 0.1) -- ++(0, 0.1);
		}
	}
	
	\node[anchor = west, align = center, xshift = .2em] at (array right) {array-synthesis\\of plant\\core promoters};


	%%% constructs
	% constructs without enhancer
	\coordinate (construct1) at ($(array) + (-.1, -1.35)$);
	\coordinate (construct2) at ($(construct1) + (.25, -.75)$);
	
	\STARRconstruct[barcode = Orchid1, promoter = Aquamarine2!80!black, promoter strength = weak, enhancer = {draw = none}]{construct1};
	\STARRconstruct[barcode = MediumPurple2, promoter = RoyalBlue2!75!black, promoter strength = strong, enhancer = {draw = none}]{construct2};
	
	\node[yshift = -.75cm, align = center] (or) at (construct2 -| .5\threecolumnwidth, 0) {STARR-seq constructs without (above)\\or with \textcolor{35S enhancer}{35S enhancer} (below)};
	
	% constructs with enhancer
	\coordinate[yshift = -.85cm] (construct3) at (construct1 |- or);
	\coordinate (construct4) at ($(construct3) + (.25, -.75)$);
	
	\STARRconstruct[barcode = HotPink1, promoter = SeaGreen4!75!white, promoter strength = very strong, enhancer = 35S enhancer]{construct3};
	\STARRconstruct[barcode = SlateBlue3, promoter = Chocolate1!75!white, promoter strength = normal, enhancer = 35S enhancer]{construct4};
	
	\coordinate (scheme south) at ($(construct4) + (0, -.25)$);
	
	
	%%% infiltration
	\node[anchor = north, align = center, xshift = -.8cm] (agroinfiltration) at (.5\textwidth, 0) {\textit{Agrobacterium}-mediated\\transformation of\\tobacco leaves};
	
	\coordinate[yshift = -1cm] (leaf) at (agroinfiltration.south);
	
	\leaf[0.5]{leaf};

	
	%%% protoplasts
	\node[anchor = south, align = center, text depth = 0pt] (electroporation) at (agroinfiltration |- ORF.south) {electroporation of\\maize protoplasts};
	
	\coordinate[yshift = .9cm] (protoplasts) at (electroporation.north);
	
	\protoplast{$(protoplasts) + (-.3, .3)$};
	\protoplast{$(protoplasts) + (.1, -.4)$};
	\protoplast{$(protoplasts) + (.2, .4)$};
	\protoplast{$(protoplasts) + (.5, -.15)$};
	\protoplast{$(protoplasts) + (-.5, -.2)$};
	
	
	%%% mRNA
	\node[anchor = east, align = center, xshift = -.25cm] (mRNA) at (agroinfiltration.east -| \textwidth - \fourcolumnwidth - \columnsep, 0) {mRNA extraction \&\\barcode sequencing};
	
	\coordinate[yshift = .5cm] (transcripts no) at (leaf -| mRNA.west);
	\coordinate[yshift = .25cm] (transcripts with) at (or -| transcripts no);
	
	% transcripts no enhancer
	\transcript[barcode = MediumPurple2]{transcripts no};
	\transcript[barcode = Orchid1]{$(transcripts no) + (-.5, -.33)$};
	\transcript[barcode = MediumPurple2]{$(transcripts no) + (-.17, -.67)$};
	
	% transcripts with enhancer
	\transcript[barcode = HotPink1]{transcripts with};
	\transcript[barcode = HotPink1]{$(transcripts with) + (-.4, -.33)$};
	\transcript[barcode = HotPink1]{$(transcripts with) + (-.25, -.67)$};
	\transcript[barcode = SlateBlue3]{$(transcripts with) + (-.15, -1)$};
	\transcript[barcode = HotPink1]{$(transcripts with) + (-.45, -1.33)$};
	
	
	%%% arrows
	\draw[arrow] (array) ++(1.55, -.25) -- ++(0, -.75);
	
	\coordinate (scheme center) at ($(scheme)!.5!(scheme south)$);
	\coordinate (arrow 1) at ($(construct end)!.5!(leaf) + (.125 - .55, 0)$);
	\coordinate (arrow 2) at ($(leaf) + (1.5, 0)$);
	
	\draw[arrow] (scheme center -| arrow 1) ++(-.375, 0) -- ++(.75, 0);
	\draw[arrow] (scheme center -| arrow 2) ++(-.375, 0) -- ++(.75, 0);
	
	
	%%% controls
	\coordinate (controls) at (scheme -| \textwidth - \fourcolumnwidth, 0);
	
	\begin{hgroupplot}[%
		ylabel = \phantom{$\log_2$(enrichment)},
		ylabel style = {name = plot ylabel},
		ymin = -2.25,
		ymax = 6,
		ytick = {-10, -9, ..., 10},
		group position = {anchor = above north west, at = {(controls)}, xshift = \plotylabelwidth, yshift = -.75cm},
		group/every plot/.append style = {
			x grids = false,
			x limits = 6,
			xtick = {1, ..., 6},
			xticklabels = {$-$, $+$, $-$, $+$, $-$, $+$},
			xticklabel style = {name = xticklabel, inner sep = .1em},
			typeset ticklabels with strut 
		}
	]{\fourcolumnwidth}{2}{}

			
		\nextgroupplot[
			title = tobacco\\leaves
		]
		
		
		\foreach \sample/\col in {%
			At_noENH/arabidopsis,
			At_withENH/arabidopsis,
			Zm_noENH/maize,
			Zm_withENH/maize,
			Sb_noENH/sorghum,
			Sb_withENH/sorghum%
		} {
			\edef\thisplot{
				\noexpand\addplot[boxplot, fill = \col, fill opacity = 0.5, mark = solido, mark options = {black}] table[y = \sample] {rawData/enrichment_controls_leaf.tsv};
			}
			\thisplot
		}
		
		\pgfplotsinvokeforeach{1, ..., 6}{
			\coordinate (l#1) at (#1, 0);
		}
		
		\nextgroupplot[
			title = maize\\protoplasts
		]
		
		
		\foreach \sample/\col in {%
			At_noENH/arabidopsis,
			At_withENH/arabidopsis,
			Zm_noENH/maize,
			Zm_withENH/maize,
			Sb_noENH/sorghum,
			Sb_withENH/sorghum%
		} {
			\edef\thisplot{
				\noexpand\addplot[boxplot, fill = \col, fill opacity = 0.5, mark = solido, mark options = {black}] table[y = \sample] {rawData/enrichment_controls_proto.tsv};
			}
			\thisplot
		}
		
		\pgfplotsinvokeforeach{1, ..., 6}{
			\coordinate (p#1) at (#1, 0);
		}
			
	\end{hgroupplot}
	
	\node[anchor = base west, rotate = 90] at (plot ylabel.base |- group c1r1.south) {promoter strength [$\log_2$(enrichment)]};
	
 	\node[anchor = base east, node font = \figsmall] at (group c1r1.west |- xticklabel.base) {\strut enhancer};
 	
 	\distance{l1}{l2}
 	\pgfmathsetlength{\templength}{0.5 * \pgfkeysvalueof{/pgfplots/boxplot/box extend} * \xdistance}
 	\draw[serif cm-serif cm] (l1.south west |- xticklabel.south) ++(-\templength, 0) -- ++(2\templength + \xdistance, 0) node[pos = .5, anchor = north, node font = \figsmall] (label At) {At};
 	\draw[serif cm-serif cm] (l3.south west |- xticklabel.south) ++(-\templength, 0) -- ++(2\templength + \xdistance, 0) node[pos = .5, anchor = north, node font = \figsmall] {Zm};
 	\draw[serif cm-serif cm] (l5.south west |- xticklabel.south) ++(-\templength, 0) -- ++(2\templength + \xdistance, 0) node[pos = .5, anchor = north, node font = \figsmall] {Sb};
 	
 	\draw[serif cm-serif cm] (p1.south west |- xticklabel.south) ++(-\templength, 0) -- ++(2\templength + \xdistance, 0) node[pos = .5, anchor = north, node font = \figsmall] {At};
 	\draw[serif cm-serif cm] (p3.south west |- xticklabel.south) ++(-\templength, 0) -- ++(2\templength + \xdistance, 0) node[pos = .5, anchor = north, node font = \figsmall] {Zm};
 	\draw[serif cm-serif cm] (p5.south west |- xticklabel.south) ++(-\templength, 0) -- ++(2\templength + \xdistance, 0) node[pos = .5, anchor = north, node font = \figsmall] {Sb};
 	
 	\leafsymbol[(0, .35)]{group c1r1.above north};
 	\protosymbol[(0, .35)]{group c2r1.above north};
	
	
	%%% correlation plot (maize library in tobacco)
	\coordinate[yshift = -\columnsep] (corLeaf) at (scheme |- scheme south);
	
	\begin{axis}[
		width = \threecolumnwidth - \plotylabelwidth,
		at = {(corLeaf)},
		anchor = north west,
		xshift = \plotylabelwidth,
		xymin = -7.95,
		xymax = 6.75,
		xytick = {-10, -8, ..., 10},
		xlabel = {promoter strength (rep 1)},
		ylabel = {promoter strength (rep 2)},
		xlabel style = {name = plot xlabel},
		scatter/classes = {
			noENH={black},
			withENH={35S enhancer}
		}
	]
	
		\addplot [
			scatter,
			scatter src = explicit symbolic,
			only marks,
			mark = solido,
			mark size = 0.25,
			fill opacity = 0.1
		] table[x = rep1,y = rep2, meta = sample.name] {rawData/enrichment_correlation_Zm_dark.tsv};
		
		\addplot [
			scatter,
			scatter src = explicit symbolic,
			only marks,
			mark = text,
			text mark as node = true,
			text mark style = {align = left, node font = \figsmall, anchor = north west, yshift = -2\baselineskip * \coordindex},
			text mark = {\spearman\\[-.25\baselineskip]\rsquare},
			visualization depends on = value \thisrow{spearman}\as\spearman,
			visualization depends on = value \thisrow{rsquare}\as\rsquare
		] table [x expr = \xmin, y expr = \ymax, meta = sample.name] {rawData/enrichment_correlation_Zm_dark_stats.tsv};
		
		\node[anchor = south east, thin, draw, fill = white, align = center, node font = \figsmall] at (rel axis cs: 0.97, 0.03) {enhancer\\\tikz\path[fill = black] (-.3em -1pt, 0) (0, 0) circle (1pt) node[anchor = west] {$-$\vphantom{A}}; ~ \tikz\path[fill = DodgerBlue1] (0, 0) circle (1pt) node[anchor = west] {$+$\vphantom{A}};};
		
	\end{axis}
	
	\leafsymbol[(-.3, -.3)]{last plot.south west};
	

	%%% correlation plot (maize library in protoplasts)
	\coordinate (corProto) at (corLeaf -| \twothirdcolumnwidth - \threecolumnwidth, 0);
	
	\begin{axis}[
		width = \threecolumnwidth - \plotylabelwidth,
		at = {(corProto)},
		anchor = north west,
		xshift = \plotylabelwidth,
		xymin = -7.95,
		xymax = 6.75,
		xytick = {-10, -8, ..., 10},
		xlabel = {promoter strength (rep 1)},
		ylabel = {promoter strength (rep 2)},
		scatter/classes = {
			noENH={black},
			withENH={35S enhancer}
		}
	]
	
		\addplot [
			scatter,
			scatter src = explicit symbolic,
			only marks,
			mark = solido,
			mark size = 0.25,
			fill opacity = 0.1
		] table[x = rep1,y = rep2, meta = sample.name] {rawData/enrichment_correlation_Zm_proto.tsv};
		
		\addplot [
			scatter,
			scatter src = explicit symbolic,
			only marks,
			mark = text,
			text mark as node = true,
			text mark style = {align = left, node font = \figsmall, anchor = north west, yshift = -2\baselineskip * \coordindex},
			text mark = {\spearman\\[-.25\baselineskip]\rsquare},
			visualization depends on = value \thisrow{spearman}\as\spearman,
			visualization depends on = value \thisrow{rsquare}\as\rsquare
		] table [x expr = \xmin, y expr = \ymax, meta = sample.name] {rawData/enrichment_correlation_Zm_proto_stats.tsv};
		
		\node[anchor = south east, thin, draw, fill = white, align = center, node font = \figsmall] at (rel axis cs: 0.97, 0.03) {enhancer\\\tikz\path[fill = black] (-.3em -1pt, 0) (0, 0) circle (1pt) node[anchor = west] {$-$\vphantom{A}}; ~ \tikz\path[fill = DodgerBlue1] (0, 0) circle (1pt) node[anchor = west] {$+$\vphantom{A}};};
		
	\end{axis}
	
	\protosymbol[(-.3, -.3)]{last plot.south west};
	
	
	%%% comparison tobacco leaves and maize protoplasts (maize library)
	\coordinate (leafVproto) at (corLeaf -| \textwidth - \threecolumnwidth, 0);
	
	\begin{axis}[
		width = \threecolumnwidth - \plotylabelwidth,
		at = {(leafVproto)},
		anchor = north west,
		xshift = \plotylabelwidth,
		xymin = -7.95,
		xymax = 6.75,
		xytick = {-10, -8, ..., 10},
		ylabel = {tobacco leaves},
		xlabel = {maize protoplasts},
		xlabel style = {align = center},
		ylabel style = {align = center},
		scatter/classes = {
			noENH={black},
			withENH={35S enhancer}
		}
	]
	
		\addplot [
			scatter,
			scatter src = explicit symbolic,
			only marks,
			mark = solido,
			mark size = 0.25,
			fill opacity = 0.1
		] table[x = proto,y = leaf, meta = sample.name] {rawData/enrichment_leaf-vs-proto_Zm.tsv};
		
		\addplot [
			scatter,
			scatter src = explicit symbolic,
			only marks,
			mark = text,
			text mark as node = true,
			text mark style = {align = left, node font = \figsmall, anchor = north west, yshift = -2\baselineskip * \coordindex},
			text mark = {\spearman\\[-.25\baselineskip]\rsquare},
			visualization depends on = value \thisrow{spearman}\as\spearman,
			visualization depends on = value \thisrow{rsquare}\as\rsquare
		] table [x expr = \xmin, y expr = \ymax, meta = sample.name] {rawData/enrichment_leaf-vs-proto_Zm_stats.tsv};
		
		\node[anchor = south east, thin, draw, fill = white, align = center, node font = \figsmall] at (rel axis cs: 0.97, 0.03) {enhancer\\\tikz\path[fill = black] (-.3em -1pt, 0) (0, 0) circle (1pt) node[anchor = west] {$-$\vphantom{A}}; ~ \tikz\path[fill = DodgerBlue1] (0, 0) circle (1pt) node[anchor = west] {$+$\vphantom{A}};};
		
	\end{axis}
	
	\leafsymbol[(-.4, 0)]{last plot.south west};
	\protosymbol[(0, -.4)]{last plot.south west};
	\node[node font = \figsmall, shift = {(-.4, -.4)}] at (last plot.south west) {vs.};
	
	
	%%% enrichment by enhancer (tobacco)
	\coordinate[yshift = -\columnsep] (enrEnhLeaf) at (plot xlabel.south -| scheme);
	
	\leafsymbol{enrEnhLeaf};
	
	\begin{hgroupplot}[%
		ylabel = promoter strength,
		ymin = -9,
		ymax = 6,
		ytick = {-10, -8, ..., 10},
		group position = {anchor = above north west, at = {(enrEnhLeaf)}, xshift = \plotylabelwidth},
		group/every plot/.append style = {x grids = false, typeset ticklabels with strut},
		xticklabel style = {inner sep = .1em}
	]{\threecolumnwidth}{3}{35S enhancer}

			
		\nextgroupplot[
			title = Arabidopsis,
			x tick table = {rawData/enrichment_enh_At_leaf_boxplot.tsv}{LaTeX.label}
		]
			
		% violin and box plot
		\violinbox[violin color = arabidopsis]{rawData/enrichment_enh_At_leaf_boxplot.tsv}{rawData/enrichment_enh_At_leaf_violin.tsv}
			
		% add sample size
		\samplesize{rawData/enrichment_enh_At_leaf_boxplot.tsv}{id}{n}
		
		
		\nextgroupplot[
			title = Maize,
			x tick table = {rawData/enrichment_enh_Zm_leaf_boxplot.tsv}{LaTeX.label}
		]
		
		% violin and box plot
		\violinbox[violin color = maize]{rawData/enrichment_enh_Zm_leaf_boxplot.tsv}{rawData/enrichment_enh_Zm_leaf_violin.tsv}
			
		% add sample size
		\samplesize{rawData/enrichment_enh_Zm_leaf_boxplot.tsv}{id}{n}
		
		
		\nextgroupplot[
			title = Sorghum,
			x tick table = {rawData/enrichment_enh_Sb_leaf_boxplot.tsv}{LaTeX.label}
		]
			
		% violin and box plot
		\violinbox[violin color = sorghum]{rawData/enrichment_enh_Sb_leaf_boxplot.tsv}{rawData/enrichment_enh_Sb_leaf_violin.tsv}
		
		% add sample size
		\samplesize{rawData/enrichment_enh_Sb_leaf_boxplot.tsv}{id}{n}
			
	\end{hgroupplot}
	
	
	%%% enrichment by enhancer (protoplasts)
	\coordinate (enrEnhProto) at (\twothirdcolumnwidth - \threecolumnwidth, 0 |- enrEnhLeaf);
	
	\protosymbol{enrEnhProto};
	
	\begin{hgroupplot}[%
		ylabel = promoter strength,
		ymin = -9,
		ymax = 6,
		ytick = {-10, -8, ..., 10},
		group position = {anchor = above north west, at = {(enrEnhProto)}, xshift = \plotylabelwidth},
		group/every plot/.append style = {x grids = false, typeset ticklabels with strut},
		xticklabel style = {inner sep = .1em}
	]{\threecolumnwidth}{3}{35S enhancer}

			
		\nextgroupplot[
			title = Arabidopsis,
			x tick table = {rawData/enrichment_enh_At_proto_boxplot.tsv}{LaTeX.label}
		]
			
		% violin and box plot
		\violinbox[violin color = arabidopsis]{rawData/enrichment_enh_At_proto_boxplot.tsv}{rawData/enrichment_enh_At_proto_violin.tsv}
			
		% add sample size
		\samplesize{rawData/enrichment_enh_At_proto_boxplot.tsv}{id}{n}
		
		
		\nextgroupplot[
			title = Maize,
			x tick table = {rawData/enrichment_enh_Zm_proto_boxplot.tsv}{LaTeX.label}
		]
		
		% violin and box plot
		\violinbox[violin color = maize]{rawData/enrichment_enh_Zm_proto_boxplot.tsv}{rawData/enrichment_enh_Zm_proto_violin.tsv}
			
		% add sample size
		\samplesize{rawData/enrichment_enh_Zm_proto_boxplot.tsv}{id}{n}
		
		
		\nextgroupplot[
			title = Sorghum,
			x tick table = {rawData/enrichment_enh_Sb_proto_boxplot.tsv}{LaTeX.label}
		]
			
		% violin and box plot
		\violinbox[violin color = sorghum]{rawData/enrichment_enh_Sb_proto_boxplot.tsv}{rawData/enrichment_enh_Sb_proto_violin.tsv}
		
		% add sample size
		\samplesize{rawData/enrichment_enh_Sb_proto_boxplot.tsv}{id}{n}
			
	\end{hgroupplot}


	%%% GO-terms
	\coordinate (GOterms) at (\textwidth - \threecolumnwidth, 0 |- enrEnhLeaf);
	
	\distance{group c1r1.south}{group c1r1.above north}

	\begin{vgroupplot}[%
		width = \fourcolumnwidth - \plotylabelwidth,
		xlabel = {$-\log_{10}(p\ \textsf{value})$},
		xmin = 0,
		xmax = 27.5,
		group position = {at = {(GOterms)}, anchor = north east, xshift = \threecolumnwidth},
		xbar = 0pt,
		y dir = reverse,
		legend columns = 3,
		legend pos = south east,
		group/every plot/.append style = {y grids = false, bar width = 0.25}
	]{\ydistance + \plotxlabelheight}{2}{}
	
		\nextgroupplot[
			y tick table = {rawData/GO_enrichment_leaf.tsv}{term_name},
		]
			
			\addlegendimage{xbar legend 1, fill = arabidopsis, fill opacity = .5}
			\addlegendimage{xbar legend 2, fill = maize, fill opacity = .5}
			\addlegendimage{xbar legend 3, fill = sorghum, fill opacity = .5}
			
			\legend{At\phantom{A}, Zm\phantom{A}, Sb}
			
			\addplot [draw = black, fill = gray, fill opacity = .5, forget plot] table [x = ns_At, y expr = \lineno] {rawData/GO_enrichment_leaf.tsv};
			\addplot [draw = black, fill = arabidopsis, fill opacity = .5] table [x = p_At, y expr = \lineno] {rawData/GO_enrichment_leaf.tsv};
			
			\addplot [draw = black, fill = gray, fill opacity = .5, forget plot] table [x = ns_Zm, y expr = \lineno] {rawData/GO_enrichment_leaf.tsv};
			\addplot [draw = black, fill = maize, fill opacity = .5] table [x = p_Zm, y expr = \lineno] {rawData/GO_enrichment_leaf.tsv};
			
			\addplot [draw = black, fill = gray, fill opacity = .5, forget plot] table [x = ns_Sb, y expr = \lineno] {rawData/GO_enrichment_leaf.tsv};
			\addplot [draw = black, fill = sorghum, fill opacity = .5] table [x = p_Sb, y expr = \lineno] {rawData/GO_enrichment_leaf.tsv};
			
			\addplot [sharp plot, red, update limits = false] coordinates {(3, 0) (3, 6)};
			
			
		\nextgroupplot[
			y tick table = {rawData/GO_enrichment_proto.tsv}{term_name},
		]
			
			\addplot [draw = black, fill = gray, fill opacity = .5, forget plot] table [x = ns_At, y expr = \lineno] {rawData/GO_enrichment_proto.tsv};
			\addplot [draw = black, fill = arabidopsis, fill opacity = .5] table [x = p_At, y expr = \lineno] {rawData/GO_enrichment_proto.tsv};
			
			\addplot [draw = black, fill = gray, fill opacity = .5, forget plot] table [x = ns_Zm, y expr = \lineno] {rawData/GO_enrichment_proto.tsv};
			\addplot [draw = black, fill = maize, fill opacity = .5] table [x = p_Zm, y expr = \lineno] {rawData/GO_enrichment_proto.tsv};
			
			\addplot [draw = black, fill = gray, fill opacity = .5, forget plot] table [x = ns_Zm, y expr = \lineno] {rawData/GO_enrichment_proto.tsv};
			\addplot [draw = black, fill = sorghum, fill opacity = .5] table [x = p_Sb, y expr = \lineno] {rawData/GO_enrichment_proto.tsv};
			
			\addplot [sharp plot, red, update limits = false] coordinates {(1.3, 0) (1.3, 6)};
	
	\end{vgroupplot}
	
	\leafsymbol[(-.35, -.35)]{group c1r1.north east};
	\protosymbol[(-.35, .35)]{group c1r2.south east};
	
	
	
	%%% enrichment by type (leaf)
	\coordinate[yshift = -\columnsep] (enrTypeLeaf) at (scheme |- plot xlabel.south);
	
	\leafsymbol{enrTypeLeaf};
	
	\begin{hgroupplot}[%
		ylabel = promoter strength,
		ymin = -9,
		ymax = 8,
		ytick = {-10, -8, ..., 10},
		xticklabel style = {rotate = 45, align = right, anchor = north east},
		group position = {anchor = above north west, at = {(enrTypeLeaf)}, xshift = \plotylabelwidth},
		group/every plot/.append style = {x grids = false}
	]{\twocolumnwidth}{3}{}

		\nextgroupplot[
			title = Arabidopsis,
			x tick table = {rawData/enrichment_type_At_leaf_boxplot.tsv}{LaTeX.label}
		]
			
		% violin and box plot
		\violinbox[violin color = arabidopsis]{rawData/enrichment_type_At_leaf_boxplot.tsv}{rawData/enrichment_type_At_leaf_violin.tsv}
			
		% add sample size
		\samplesize{rawData/enrichment_type_At_leaf_boxplot.tsv}{id}{n}
		
		% add significance level
		\signif*{rawData/enrichment_type_At_leaf_pvalues.tsv}{1}{2}
		\signif*{rawData/enrichment_type_At_leaf_pvalues.tsv}{2}{3}
		
		
		\nextgroupplot[
			title = Maize,
			x tick table = {rawData/enrichment_type_Zm_leaf_boxplot.tsv}{LaTeX.label}
		]
		
		% violin and box plot
		\violinbox[violin color = maize]{rawData/enrichment_type_Zm_leaf_boxplot.tsv}{rawData/enrichment_type_Zm_leaf_violin.tsv}
			
		% add sample size
		\samplesize{rawData/enrichment_type_Zm_leaf_boxplot.tsv}{id}{n}
		
		% add significance level
		\signif*{rawData/enrichment_type_Zm_leaf_pvalues.tsv}{1}{2}
		\signif*{rawData/enrichment_type_Zm_leaf_pvalues.tsv}{2}{3}
		
		
		\nextgroupplot[
			title = Sorghum,
			x tick table = {rawData/enrichment_type_Sb_leaf_boxplot.tsv}{LaTeX.label}
		]
			
		% violin and box plot
		\violinbox[violin color = sorghum]{rawData/enrichment_type_Sb_leaf_boxplot.tsv}{rawData/enrichment_type_Sb_leaf_violin.tsv}
		
		% add sample size
		\samplesize{rawData/enrichment_type_Sb_leaf_boxplot.tsv}{id}{n}
		
		% add significance level
		\signif*{rawData/enrichment_type_Sb_leaf_pvalues.tsv}{1}{2}
		\signif*{rawData/enrichment_type_Sb_leaf_pvalues.tsv}{2}{3}
			
	\end{hgroupplot}
	
	
	%%% enrichment by type (proto)
	\coordinate (enrTypeProto) at (enrTypeLeaf -| \textwidth - \twocolumnwidth, 0);
	
	\protosymbol{enrTypeProto}; 
	
	\begin{hgroupplot}[%
		ylabel = promoter strength,
		ymin = -9,
		ymax = 8,
		ytick = {-10, -8, ..., 10},
		xticklabel style = {rotate = 45, align = right, anchor = north east},
		group position = {anchor = above north west, at = {(enrTypeProto)}, xshift = \plotylabelwidth},
		group/every plot/.append style = {x grids = false}
	]{\twocolumnwidth}{3}{}

		\nextgroupplot[
			title = Arabidopsis,
			x tick table = {rawData/enrichment_type_At_proto_boxplot.tsv}{LaTeX.label}
		]
			
		% violin and box plot
		\violinbox[violin color = arabidopsis]{rawData/enrichment_type_At_proto_boxplot.tsv}{rawData/enrichment_type_At_proto_violin.tsv}
			
		% add sample size
		\samplesize{rawData/enrichment_type_At_proto_boxplot.tsv}{id}{n}
		
		% add significance level
		\signif*{rawData/enrichment_type_At_proto_pvalues.tsv}{1}{2}
		\signif*{rawData/enrichment_type_At_proto_pvalues.tsv}{2}{3}
		
		
		\nextgroupplot[
			title = Maize,
			x tick table = {rawData/enrichment_type_Zm_proto_boxplot.tsv}{LaTeX.label}
		]
		
		% violin and box plot
		\violinbox[violin color = maize]{rawData/enrichment_type_Zm_proto_boxplot.tsv}{rawData/enrichment_type_Zm_proto_violin.tsv}
			
		% add sample size
		\samplesize{rawData/enrichment_type_Zm_proto_boxplot.tsv}{id}{n}
		
		% add significance level
		\signif*{rawData/enrichment_type_Zm_proto_pvalues.tsv}{1}{2}
		\signif*{rawData/enrichment_type_Zm_proto_pvalues.tsv}{2}{3}
		
		
		\nextgroupplot[
			title = Sorghum,
			x tick table = {rawData/enrichment_type_Sb_proto_boxplot.tsv}{LaTeX.label}
		]
			
		% violin and box plot
		\violinbox[violin color = sorghum]{rawData/enrichment_type_Sb_proto_boxplot.tsv}{rawData/enrichment_type_Sb_proto_violin.tsv}
		
		% add sample size
		\samplesize{rawData/enrichment_type_Sb_proto_boxplot.tsv}{id}{n}
		
		% add significance level
		\signif*{rawData/enrichment_type_Sb_proto_pvalues.tsv}{1}{2}
		\signif*{rawData/enrichment_type_Sb_proto_pvalues.tsv}{2}{3}
			
	\end{hgroupplot}
	
	
	%%% subfigure labels
	\subfiglabel{scheme}
	\subfiglabel{controls}
	\subfiglabel[yshift = .5\columnsep]{corLeaf}
	\subfiglabel[yshift = .5\columnsep]{corProto}
	\subfiglabel[yshift = .5\columnsep]{leafVproto}
	\subfiglabel[yshift = .5\columnsep]{enrEnhLeaf}
	\subfiglabel[yshift = .5\columnsep]{enrEnhProto}
	\subfiglabel[yshift = .5\columnsep]{GOterms}
	\subfiglabel[yshift = .5\columnsep]{enrTypeLeaf}
	\subfiglabel[yshift = .5\columnsep]{enrTypeProto}

\end{tikzpicture}%
		\end{fig}
		
		\begin{figure}[h]
			\caption{%
				\textbf{STARR-seq measures core promoter strength in tobacco leaves and maize protoplasts.}\titleend
				\subfigref{A} Assay scheme. The core promoters (bases $-165$ to $+5$ relative to the TSS) of all genes of Arabidopsis, maize and sorghum were array-synthesized and cloned into STARR-seq constructs to drive the expression of a barcoded GFP reporter gene. For each species, two libraries, one without and one with a 35S enhancer upstream of the promoter, were created. The libraries were subjected to STARR-seq in transiently transformed tobacco leaves and maize protoplasts.\nextentry
				\subfigref{B} Each promoter library (At, Arabiopsis; Zm, maize; Sb, sorghum) contained two internal control constructs driven by the 35S minimal promoter without ($-$) or with ($+$) an upstream 35S enhancer. The enrichment ($\log_2$) of recovered mRNA barcodes compared to DNA input was calculated with the enrichment of the enhancer-less control set to 0. In all following figures this metric is indicated as promoter strength. Each boxplot (center line, median; box limits, upper and lower quartiles; whiskers, 1.5 $\times$ interquartile range; points, outliers) represents the enrichment of all barcodes linked to the corresponding construct.\nextentry
				\subfigref{C}\subfigref{D} Correlation of two biological replicates of STARR-seq using the maize promoter libraries in tobacco leaves \parensubfig{C} or in maize protoplasts \parensubfig{D}.\nextentry
				\subfigref{E} Comparison of the strength of maize promoters in tobacco leaves and maize protoplasts.\nextentry
				\subfigref{F}\subfigref{G} Violin plots of promoter strength as measured by STARR-seq in tobacco leaves \parensubfig{F} or maize protoplasts \parensubfig{G} for libraries without ($-$) or with ($+$) the 35S enhancer upstream of the promoter. \nextentry
				\subfigref{H} Enrichment of selected GO terms for genes associated with the 1000 strongest promoters in the Arabidopsis (At), maize (Zm), and sorghum (Sb) promoter libraries without enhancer in tobacco leaves (top panel) and maize protoplasts (bottom panel). The red line marks the significance threshold (adjusted $p\ \textsf{value} \leq 0.05$). Non-significant bars are shown in gray.\nextentry
				\subfigref{I}\subfigref{J} Violinplots of promoter strength (libraries without 35S enhancer) in tobacco leaves \parensubfig{I} or maize protoplasts \parensubfig{J}. Promoters were grouped by gene type. In all figures, violinplots represent the kernel density distribution and the boxplots within represent the median (center line), upper and lower quartiles (box limits), and 1.5 $\times$ the interquartile range (whiskers) for all corresponding promoters. Numbers at the bottom of the plot indicate the number of tested promoters. Significant differences between two samples were determined using the Wilcoxon rank-sum test and are indicated: \issignif, $p \leq \signiflevelone$; \issignif\issignif, $p \leq \signifleveltwo$; \issignif\issignif\issignif, $p \leq \signiflevelthree$; \notsignif, not significant.
			}%
			\label{fig:overview}%
		\end{figure}
		\clearpage
		
		\begin{fig}
			%\tikzset{external/export next = false}

\begin{tikzpicture}

	%%% GC density
	\coordinate (density) at (0, 0);
	
	\begin{axis}[
		anchor = north west,
		at = {(density)},
		xshift = \plotylabelwidth,
		width = \twocolumnwidth - \plotylabelwidth,
		enlarge x limits = false,
		ymin = 0,
		ytick = {0, ..., 10},
		xlabel = GC content,
		ylabel = density,
		xlabel style = {name = plot xlabel},
		density legend
	]
	
	\addplot [arabidopsis, fill, fill opacity = .5] table[x = x.At ,y = y.At] {rawData/GC_density.tsv};
	\addplot [maize, fill, fill opacity = .5] table[x = x.Zm ,y = y.Zm] {rawData/GC_density.tsv};
	\addplot [sorghum, fill, fill opacity = .5] table[x = x.Sb ,y = y.Sb] {rawData/GC_density.tsv};
	
	\addplot [sharp plot, update limits = false, arabidopsis] table[x = promoters.At, y = y] {rawData/GC_average.tsv};
	\addplot [sharp plot, update limits = false, arabidopsis, dashed] table[x = genome.At, y = y] {rawData/GC_average.tsv};
	\addplot [sharp plot, update limits = false, maize] table[x = promoters.Zm, y = y] {rawData/GC_average.tsv};
	\addplot [sharp plot, update limits = false, maize, dashed] table[x = genome.Zm, y = y] {rawData/GC_average.tsv};
	\addplot [sharp plot, update limits = false, sorghum] table[x = promoters.Sb, y = y] {rawData/GC_average.tsv};
	\addplot [sharp plot, update limits = false, sorghum, dashed] table[x = genome.Sb, y = y] {rawData/GC_average.tsv};
	
	\legend{Arabidopsis, Maize, Sorghum}
	
	\end{axis}
	
	
	%%% enrichment by GC (tobacco)
	\coordinate[yshift = -\columnsep] (enrGCleaf) at (density |- plot xlabel.south);
	
	\leafsymbol{enrGCleaf};
	
	\begin{hgroupplot}[%
		ylabel = promoter strength,
		ymin = -9,
		ymax = 8,
		ytick = {-10, -8, ..., 10},
		group position = {anchor = above north west, at = {(enrGCleaf)}, xshift = \plotylabelwidth},
		group/every plot/.append style = {x grids = false}
	]{\twocolumnwidth}{3}{GC content}

			
		\nextgroupplot[
			title = Arabidopsis,
			x tick table = {rawData/enrichment_GC_At_leaf_boxplot.tsv}{LaTeX.label}
		]
			
		% violin and box plot
		\violinbox[violin color = arabidopsis]{rawData/enrichment_GC_At_leaf_boxplot.tsv}{rawData/enrichment_GC_At_leaf_violin.tsv}
			
		% add sample size
		\samplesize{rawData/enrichment_GC_At_leaf_boxplot.tsv}{id}{n}
		
		% add significance level
		\signif*{rawData/enrichment_GC_At_leaf_pvalues.tsv}{1}{2}
		\signif*{rawData/enrichment_GC_At_leaf_pvalues.tsv}{2}{3}
		\signif*{rawData/enrichment_GC_At_leaf_pvalues.tsv}{3}{4}
		\signif*{rawData/enrichment_GC_At_leaf_pvalues.tsv}{4}{5}
		
		
		\nextgroupplot[
			title = Maize,
			x tick table = {rawData/enrichment_GC_Zm_leaf_boxplot.tsv}{LaTeX.label}
		]
		
		% violin and box plot
		\violinbox[violin color = maize]{rawData/enrichment_GC_Zm_leaf_boxplot.tsv}{rawData/enrichment_GC_Zm_leaf_violin.tsv}
			
		% add sample size
		\samplesize{rawData/enrichment_GC_Zm_leaf_boxplot.tsv}{id}{n}
		
		% add significance level
		\signif*{rawData/enrichment_GC_Zm_leaf_pvalues.tsv}{1}{2}
		\signif*{rawData/enrichment_GC_Zm_leaf_pvalues.tsv}{2}{3}
		\signif*{rawData/enrichment_GC_Zm_leaf_pvalues.tsv}{3}{4}
		\signif*{rawData/enrichment_GC_Zm_leaf_pvalues.tsv}{4}{5}
		
		
		\nextgroupplot[
			title = Sorghum,
			x tick table = {rawData/enrichment_GC_Sb_leaf_boxplot.tsv}{LaTeX.label}
		]
			
		% violin and box plot
		\violinbox[violin color = sorghum]{rawData/enrichment_GC_Sb_leaf_boxplot.tsv}{rawData/enrichment_GC_Sb_leaf_violin.tsv}
		
		% add sample size
		\samplesize{rawData/enrichment_GC_Sb_leaf_boxplot.tsv}{id}{n}
		
		% add significance level
		\signif*{rawData/enrichment_GC_Sb_leaf_pvalues.tsv}{1}{2}
		\signif*{rawData/enrichment_GC_Sb_leaf_pvalues.tsv}{2}{3}
		\signif*{rawData/enrichment_GC_Sb_leaf_pvalues.tsv}{3}{4}
		\signif*{rawData/enrichment_GC_Sb_leaf_pvalues.tsv}{4}{5}
			
	\end{hgroupplot}


	%%% GC by position (tobacco)
	\coordinate (GCpos) at (density -| \textwidth - \twocolumnwidth, 0);
	
	\leafsymbol{GCpos};
	
	\begin{axis}[
		anchor = north west,
		at = {(GCpos)},
		xshift = \plotylabelwidth,
		width = \twocolumnwidth - \plotylabelwidth,
		ytick = {-1, -.9, ..., 1},
		xtick = {6, 26, ..., 166},
		xticklabel = {$\pgfmathparse{\tick < 166 ? \tick - 166 : \tick -165}\pgfmathprintnumber[print sign]{\pgfmathresult}$},
		xmin = 0,
		xmax = 170,
		ymax = 0,
		y decimals = 1,
		xlabel = position (rel. to TSS),
		ylabel = Pearson's $r$,
		xlabel style = {name = plot xlabel},
		legend pos = south east,
		legend columns = 3
	]
	
	\addplot [sharp plot, arabidopsis] table[x = pos, y = At_leaf] {rawData/GC_by_pos.tsv};
	\addplot [sharp plot, maize] table[x = pos, y = Zm_leaf] {rawData/GC_by_pos.tsv};
	\addplot [sharp plot, sorghum] table[x = pos, y = Sb_leaf] {rawData/GC_by_pos.tsv};
	
	\legend{Arabidopsis\phantom{A}, Maize\phantom{A}, Sorghum}
	
	\end{axis}


	%%% enrichment by GC (protoplasts)
	\coordinate[yshift = -\columnsep] (enrGCproto) at (GCpos |- plot xlabel.south);
	
	\protosymbol{enrGCproto};
	
	\begin{hgroupplot}[%
		ylabel = promoter strength,
		ymin = -9,
		ymax = 8,
		ytick = {-10, -8, ..., 10},
		group position = {anchor = above north west, at = {(enrGCproto)}, xshift = \plotylabelwidth},
		group/every plot/.append style = {x grids = false}
	]{\twocolumnwidth}{3}{GC content}

			
		\nextgroupplot[
			title = Arabidopsis,
			x tick table = {rawData/enrichment_GC_At_proto_boxplot.tsv}{LaTeX.label}
		]
			
		% violin and box plot
		\violinbox[violin color = arabidopsis]{rawData/enrichment_GC_At_proto_boxplot.tsv}{rawData/enrichment_GC_At_proto_violin.tsv}
			
		% add sample size
		\samplesize{rawData/enrichment_GC_At_proto_boxplot.tsv}{id}{n}
		
		% add significance level
		\signif*{rawData/enrichment_GC_At_proto_pvalues.tsv}{1}{2}
		\signif*{rawData/enrichment_GC_At_proto_pvalues.tsv}{2}{3}
		\signif*{rawData/enrichment_GC_At_proto_pvalues.tsv}{3}{4}
		\signif*{rawData/enrichment_GC_At_proto_pvalues.tsv}{4}{5}
		
		
		\nextgroupplot[
			title = Maize,
			x tick table = {rawData/enrichment_GC_Zm_proto_boxplot.tsv}{LaTeX.label}
		]
		
		% violin and box plot
		\violinbox[violin color = maize]{rawData/enrichment_GC_Zm_proto_boxplot.tsv}{rawData/enrichment_GC_Zm_proto_violin.tsv}
			
		% add sample size
		\samplesize{rawData/enrichment_GC_Zm_proto_boxplot.tsv}{id}{n}
		
		% add significance level
		\signif*{rawData/enrichment_GC_Zm_proto_pvalues.tsv}{1}{2}
		\signif*{rawData/enrichment_GC_Zm_proto_pvalues.tsv}{2}{3}
		\signif*{rawData/enrichment_GC_Zm_proto_pvalues.tsv}{3}{4}
		\signif*{rawData/enrichment_GC_Zm_proto_pvalues.tsv}{4}{5}
		
		
		\nextgroupplot[
			title = Sorghum,
			x tick table = {rawData/enrichment_GC_Sb_proto_boxplot.tsv}{LaTeX.label}
		]
			
		% violin and box plot
		\violinbox[violin color = sorghum]{rawData/enrichment_GC_Sb_proto_boxplot.tsv}{rawData/enrichment_GC_Sb_proto_violin.tsv}
		
		% add sample size
		\samplesize{rawData/enrichment_GC_Sb_proto_boxplot.tsv}{id}{n}
		
		% add significance level
		\signif*{rawData/enrichment_GC_Sb_proto_pvalues.tsv}{1}{2}
		\signif*{rawData/enrichment_GC_Sb_proto_pvalues.tsv}{2}{3}
		\signif*{rawData/enrichment_GC_Sb_proto_pvalues.tsv}{3}{4}
		\signif*{rawData/enrichment_GC_Sb_proto_pvalues.tsv}{4}{5}
			
	\end{hgroupplot}
	
	
	%%% subfigure labels
	\subfiglabel[yshift = .5\columnsep]{density}
	\subfiglabel[yshift = .5\columnsep]{enrGCleaf}
	\subfiglabel[yshift = .5\columnsep]{GCpos}
	\subfiglabel[yshift = .5\columnsep]{enrGCproto}

\end{tikzpicture}%
			\caption{%
				\textbf{GC content affects promoter strength in tobacco leaves.}\titleend
				\subfigref{A} Distribution of GC content for all promoters of the indicated species. Lines denote the mean GC content of promoters (solid line) and the whole genome (dashed line).\nextentry
				\subfigref{B} Violin plots (as defined in \autoref{fig:overview}) of promoter strength for libraries without enhancer in tobacco leaves. Promoters are grouped by GC content to yield groups of approximately similar size.\nextentry
				\subfigref{C} Correlation (Pearson's $r$) between promoter strength and the GC content of a 10 base window around the indicated position in the plant promoters.\nextentry
				\subfigref{D} As \parensubfig{B} but for promoter strength in maize protoplasts.
			}%
			\label{fig:GC}%
		\end{fig}
		
		\begin{fig}
			%\tikzset{external/export next = false}

\begin{tikzpicture}

	%%% TATA position
	\coordinate (TATApos) at (0, 0);

	% will be drawn after the next two plots to get correct height
	
	
	%%% enrichment by TATA pos (tobacco)
	\coordinate (enrTATAposLeaf) at (TATApos -| \textwidth - \twocolumnwidth, 0);
	
	\leafsymbol{enrTATAposLeaf}; 
	
	\begin{hgroupplot}[%
		ylabel = promoter strength,
		ymin = -9,
		ymax = 8,
		ytick = {-10, -8, ..., 10},
		xticklabel style = {name = xticklabel, inner sep = .1em},
		group position = {anchor = above north west, at = {(enrTATAposLeaf)}, xshift = \plotylabelwidth},
		group/every plot/.append style = {x grids = false, typeset ticklabels with strut}
	]{\twocolumnwidth}{3}{}

		\nextgroupplot[
			title = Arabidopsis,
			x tick table = {rawData/enrichment_TATA-pos_At_leaf_boxplot.tsv}{LaTeX.label}
		]
			
		% violin and box plot
		\violinbox[violin color = arabidopsis]{rawData/enrichment_TATA-pos_At_leaf_boxplot.tsv}{rawData/enrichment_TATA-pos_At_leaf_violin.tsv}
			
		% add sample size
		\samplesize{rawData/enrichment_TATA-pos_At_leaf_boxplot.tsv}{id}{n}
		
		% add significance level
		\signif*{rawData/enrichment_TATA-pos_At_leaf_pvalues.tsv}{1}{2}
		\signif*{rawData/enrichment_TATA-pos_At_leaf_pvalues.tsv}{2}{3}
		
		\pgfplotsinvokeforeach{1, ..., 6}{
			\coordinate (A#1) at (#1, 0);
		}
		
		
		\nextgroupplot[
			title = Maize,
			x tick table = {rawData/enrichment_TATA-pos_Zm_leaf_boxplot.tsv}{LaTeX.label}
		]
		
		% violin and box plot
		\violinbox[violin color = maize]{rawData/enrichment_TATA-pos_Zm_leaf_boxplot.tsv}{rawData/enrichment_TATA-pos_Zm_leaf_violin.tsv}
			
		% add sample size
		\samplesize{rawData/enrichment_TATA-pos_Zm_leaf_boxplot.tsv}{id}{n}
		
		% add significance level
		\signif*{rawData/enrichment_TATA-pos_Zm_leaf_pvalues.tsv}{1}{2}
		\signif*{rawData/enrichment_TATA-pos_Zm_leaf_pvalues.tsv}{2}{3}
		
		\pgfplotsinvokeforeach{1, ..., 6}{
			\coordinate (Z#1) at (#1, 0);
		}
		
		
		\nextgroupplot[
			title = Sorghum,
			x tick table = {rawData/enrichment_TATA-pos_Sb_leaf_boxplot.tsv}{LaTeX.label}
		]
			
		% violin and box plot
		\violinbox[violin color = sorghum]{rawData/enrichment_TATA-pos_Sb_leaf_boxplot.tsv}{rawData/enrichment_TATA-pos_Sb_leaf_violin.tsv}
		
		% add sample size
		\samplesize{rawData/enrichment_TATA-pos_Sb_leaf_boxplot.tsv}{id}{n}
		
		% add significance level
		\signif*{rawData/enrichment_TATA-pos_Sb_leaf_pvalues.tsv}{1}{2}
		\signif*{rawData/enrichment_TATA-pos_Sb_leaf_pvalues.tsv}{2}{3}
		
		\pgfplotsinvokeforeach{1, ..., 6}{
			\coordinate (S#1) at (#1, 0);
		}
			
	\end{hgroupplot}
	
	\node[anchor = base east, node font = \figsmall] (lTATA) at (group c1r1.west |- xticklabel.base) {TATA-box};
	\node[anchor = north west, node font = \figsmall] (lpos) at (lTATA.base west) {at $-59$ to $-23$};
	\node[anchor = base, node font = \figsmall] at (lpos.base -| A2) {$-$};
	\node[anchor = base, node font = \figsmall] at (lpos.base -| A3) {$+$};
	\node[anchor = base, node font = \figsmall] at (lpos.base -| Z2) {$-$};
	\node[anchor = base, node font = \figsmall] at (lpos.base -| Z3) {$+$};
	\node[anchor = base, node font = \figsmall] at (lpos.base -| S2) {$-$};
	\node[anchor = base, node font = \figsmall] at (lpos.base -| S3) {$+$};
	
	
	%%% enrichment by TATA pos (protoplasts)
	\coordinate[yshift = -\columnsep] (enrTATAposProto) at (lpos.south -| enrTATAposLeaf);
	
	\protosymbol{enrTATAposProto}; 
	
	\begin{hgroupplot}[%
		ylabel = promoter strength,
		ymin = -9,
		ymax = 8,
		ytick = {-10, -8, ..., 10},
		xticklabel style = {name = xticklabel, inner sep = .1em},
		group position = {anchor = above north west, at = {(enrTATAposProto)}, xshift = \plotylabelwidth},
		group/every plot/.append style = {x grids = false, typeset ticklabels with strut}
	]{\twocolumnwidth}{3}{}

		\nextgroupplot[
			title = Arabidopsis,
			x tick table = {rawData/enrichment_TATA-pos_At_proto_boxplot.tsv}{LaTeX.label}
		]
			
		% violin and box plot
		\violinbox[violin color = arabidopsis]{rawData/enrichment_TATA-pos_At_proto_boxplot.tsv}{rawData/enrichment_TATA-pos_At_proto_violin.tsv}
			
		% add sample size
		\samplesize{rawData/enrichment_TATA-pos_At_proto_boxplot.tsv}{id}{n}
		
		% add significance level
		\signif*{rawData/enrichment_TATA-pos_At_proto_pvalues.tsv}{1}{2}
		\signif*{rawData/enrichment_TATA-pos_At_proto_pvalues.tsv}{2}{3}
		
		\pgfplotsinvokeforeach{1, ..., 6}{
			\coordinate (A#1) at (#1, 0);
		}
		
		
		\nextgroupplot[
			title = Maize,
			x tick table = {rawData/enrichment_TATA-pos_Zm_proto_boxplot.tsv}{LaTeX.label}
		]
		
		% violin and box plot
		\violinbox[violin color = maize]{rawData/enrichment_TATA-pos_Zm_proto_boxplot.tsv}{rawData/enrichment_TATA-pos_Zm_proto_violin.tsv}
			
		% add sample size
		\samplesize{rawData/enrichment_TATA-pos_Zm_proto_boxplot.tsv}{id}{n}
		
		% add significance level
		\signif*{rawData/enrichment_TATA-pos_Zm_proto_pvalues.tsv}{1}{2}
		\signif*{rawData/enrichment_TATA-pos_Zm_proto_pvalues.tsv}{2}{3}
		
		\pgfplotsinvokeforeach{1, ..., 6}{
			\coordinate (Z#1) at (#1, 0);
		}
		
		
		\nextgroupplot[
			title = Sorghum,
			x tick table = {rawData/enrichment_TATA-pos_Sb_proto_boxplot.tsv}{LaTeX.label}
		]
			
		% violin and box plot
		\violinbox[violin color = sorghum]{rawData/enrichment_TATA-pos_Sb_proto_boxplot.tsv}{rawData/enrichment_TATA-pos_Sb_proto_violin.tsv}
		
		% add sample size
		\samplesize{rawData/enrichment_TATA-pos_Sb_proto_boxplot.tsv}{id}{n}
		
		% add significance level
		\signif*{rawData/enrichment_TATA-pos_Sb_proto_pvalues.tsv}{1}{2}
		\signif*{rawData/enrichment_TATA-pos_Sb_proto_pvalues.tsv}{2}{3}
		
		\pgfplotsinvokeforeach{1, ..., 6}{
			\coordinate (S#1) at (#1, 0);
		}
			
	\end{hgroupplot}
	
	\node[anchor = base east, node font = \figsmall] (lTATA) at (group c1r1.west |- xticklabel.base) {TATA-box};
	\node[anchor = north west, node font = \figsmall] (lpos) at (lTATA.base west) {at $-59$ to $-23$};
	\node[anchor = base, node font = \figsmall] at (lpos.base -| A2) {$-$};
	\node[anchor = base, node font = \figsmall] at (lpos.base -| A3) {$+$};
	\node[anchor = base, node font = \figsmall] at (lpos.base -| Z2) {$-$};
	\node[anchor = base, node font = \figsmall] at (lpos.base -| Z3) {$+$};
	\node[anchor = base, node font = \figsmall] at (lpos.base -| S2) {$-$};
	\node[anchor = base, node font = \figsmall] at (lpos.base -| S3) {$+$};
	
	
	%%% TATA position plot
	\distance{group c1r1.south}{enrTATAposLeaf}
	
	\def\prefPos{\addplot [ybar interval, draw, fill, gray, opacity = .2, forget plot] coordinates {(107, \ymax) (143, \ymax)};}
	
	\begin{vgroupplot}[%
		width = \twocolumnwidth - \plotylabelwidth,
			xlabel = {TATA-box position (rel. to TSS)},
			xlabel style = {name = plot xlabel},
			xmin = 0,
			xmax = 170,
			ymin = 0,
			ymax = 1.35,
			y decimals = 1,
			xtick = {6, 26, ..., 166},
			xticklabel = {$\pgfmathparse{\tick < 166 ? \tick - 166 : \tick -165}\pgfmathprintnumber[print sign]{\pgfmathresult}$},
			ytick = {0, 0.2, ..., 2}, 
			group position = {at = {(TATApos)}, anchor = north west, xshift = \plotylabelwidth},
			ybar = 0pt,
			legend pos = north west
	]{\ydistance + \plotxlabelheight}{3}{promoters (\%)}
	
			\nextgroupplot[
				bar width = 1
			]

			\prefPos;
			\addplot [arabidopsis, fill, fill opacity = .5] table [x = pos, y = At] {rawData/TATA_position.tsv};
			
			\addlegendimage{maize, fill, fill opacity = .5}
			\addlegendimage{sorghum, fill, fill opacity = .5}
			
			\legend{Arabidopsis, Maize, Sorghum}
			
			
			\nextgroupplot[
				bar width = 1
			]
			
			\prefPos;
			\addplot [maize, fill, fill opacity = .5] table [x = pos, y = Zm] {rawData/TATA_position.tsv};
			
			
			\nextgroupplot[
				bar width = 1
			]
			
			\prefPos;
			\addplot [sorghum, fill, fill opacity = .5] table [x = pos, y = Sb] {rawData/TATA_position.tsv};
	
	\end{vgroupplot}
	
	
	%%% mutate TATA motif (logo plots)
	\coordinate[yshift = -\columnsep] (variantsTATAmut) at (TATApos |- plot xlabel.south);
	
	\coordinate[shift = {(\plotylabelwidth, \columnsep)}] (last plot) at (variantsTATAmut);
	
	\foreach \variant in {WT, mutA, mutB, mutAB} {
		\logoplot[
			at = {(last plot.south west)},
			yshift = -\columnsep,
			anchor = above north west,
			width = .8\fourcolumnwidth - \plotylabelwidth,
			height = 0.5cm,
			logo axis = IC,
			logo y axis/.append style = {ylabel = IC},
			title = {\variant},
			title style = {draw = none, fill = none}
		]{rawData/motif_mutTATA_\variant.tsv}
	}
	
	
	%%% mutate TATA motif (promoter strength)
	\coordinate (enrTATAmut) at (\twocolumnwidth - 1.2\fourcolumnwidth, 0 |- variantsTATAmut);
	
	\begin{hgroupplot}[%
		ylabel = {rel. promoter strength},
		ymin = -4.5,
		ymax = 2.5,
		ytick = {-10, -9, ..., 10},
		zero line,
		xticklabel style = {rotate = 45, align = right, anchor = north east},
		group position = {anchor = above north west, at = {(enrTATAmut)}, xshift = \plotylabelwidth},
		group/every plot/.append style = {
			x grids = false,
			typeset ticklabels with strut 
		}
	]{1.2\fourcolumnwidth}{2}{}

			
		\nextgroupplot[
			title = {\phantom{tobacco}\\\phantom{leaves}},
			x tick table = {rawData/enrichment_mutateTATA_leaf_boxplot.tsv}{sample}
		]
		
			% boxplot
			\boxplots{%
				box color = leafCol,
				box shade,
				fill opacity = 0.5%
			}{rawData/enrichment_mutateTATA_leaf_boxplot.tsv}{rawData/enrichment_mutateTATA_leaf_boxplot_outliers.tsv}
			
			% add sample size
			\samplesize{rawData/enrichment_mutateTATA_leaf_boxplot.tsv}{id}{n}
			
			% add significance
			\signifallsimple*{rawData/enrichment_mutateTATA_leaf_boxplot.tsv}{id}{p.value}
		
		
		\nextgroupplot[
			title = {\phantom{maize}\\\phantom{protoplasts}},
			x tick table = {rawData/enrichment_mutateTATA_proto_boxplot.tsv}{sample}
		]
		
			% boxplot
			\boxplots{%
				box color = protoCol,
				box shade,
				fill opacity = 0.5%
			}{rawData/enrichment_mutateTATA_proto_boxplot.tsv}{rawData/enrichment_mutateTATA_proto_boxplot_outliers.tsv}
			
			% add sample size
			\samplesize{rawData/enrichment_mutateTATA_proto_boxplot.tsv}{id}{n}
			
			% add significance
			\signifallsimple*{rawData/enrichment_mutateTATA_proto_boxplot.tsv}{id}{p.value}
			
	\end{hgroupplot}
	
	\distance{title.south}{title.north}
	
	\leafsymbol[(.5\ydistance, -.5\ydistance)]{group c1r1.west |- title.north}
	\node[anchor = east, node font = \fignormal, text depth = .15\baselineskip, align = right] at (group c1r1.east |- title) {tobacco\\leaves};
	\protosymbol[(-.5\ydistance, -.5\ydistance)]{group c2r1.east |- title.north}
	\node[anchor = west, node font = \fignormal, text depth = .15\baselineskip, align = left] at (group c2r1.west |- title) {maize\\protoplasts};
	
	
	%%% insert TATA motif (logo plots)
	\coordinate (variantsTATAins) at (\textwidth - \twocolumnwidth, 0 |- variantsTATAmut);
	
	\coordinate[shift = {(\plotylabelwidth, \columnsep)}] (last plot) at (variantsTATAins);
	
	\foreach \variant in {WT, +TATA, +mutTATA} {
		\logoplot[
			at = {(last plot.south west)},
			yshift = -\columnsep,
			anchor = above north west,
			width = .8\fourcolumnwidth - \plotylabelwidth,
			height = 0.5cm,
			logo axis = IC,
			logo y axis/.append style = {ylabel = IC},
			title = {\variant},
			title style = {draw = none, fill = none}
		]{rawData/motif_insTATA_\variant.tsv}
	}
	
	%%% insert TATA motif
	\coordinate (enrTATAins) at (\textwidth - 1.2\fourcolumnwidth, 0 |- variantsTATAmut);
	
	\begin{hgroupplot}[%
		ylabel = {rel. promoter strength},
		ymin = -3,
		ymax = 3.75,
		ytick = {-10, -9, ..., 10},
		zero line,
		xticklabel style = {rotate = 45, align = right, anchor = north east},
		group position = {anchor = above north west, at = {(enrTATAins)}, xshift = \plotylabelwidth},
		group/every plot/.append style = {
			x grids = false,
			typeset ticklabels with strut 
		}
	]{1.2\fourcolumnwidth}{2}{}

			
		\nextgroupplot[
			title = {\phantom{tobacco}\\\phantom{leaves}},
			x tick table = {rawData/enrichment_insertTATA_leaf_boxplot.tsv}{sample}
		]
			
			% boxplot
			\boxplots{%
				box color = leafCol,
				box shade,
				fill opacity = 0.5%
			}{rawData/enrichment_insertTATA_leaf_boxplot.tsv}{rawData/enrichment_insertTATA_leaf_boxplot_outliers.tsv}
			
			% add sample size
			\samplesize{rawData/enrichment_insertTATA_leaf_boxplot.tsv}{id}{n}
			
			% add significance
			\signifallsimple*{rawData/enrichment_insertTATA_leaf_boxplot.tsv}{id}{p.value}
		
		
		\nextgroupplot[
			title = {\phantom{maize}\\\phantom{protoplasts}},
			x tick table = {rawData/enrichment_insertTATA_proto_boxplot.tsv}{sample}
		]
		
			% boxplot
			\boxplots{%
				box color = protoCol,
				box shade,
				fill opacity = 0.5%
			}{rawData/enrichment_insertTATA_proto_boxplot.tsv}{rawData/enrichment_insertTATA_proto_boxplot_outliers.tsv}
			
			% add sample size
			\samplesize{rawData/enrichment_insertTATA_proto_boxplot.tsv}{id}{n}
			
			% add significance
			\signifallsimple*{rawData/enrichment_insertTATA_proto_boxplot.tsv}{id}{p.value}
			
	\end{hgroupplot}
	
	\leafsymbol[(.5\ydistance, -.5\ydistance)]{group c1r1.west |- title.north}
	\node[anchor = east, node font = \fignormal, text depth = .15\baselineskip, align = right] at (group c1r1.east |- title) {tobacco\\leaves};
	\protosymbol[(-.5\ydistance, -.5\ydistance)]{group c2r1.east |- title.north}
	\node[anchor = west, node font = \fignormal, text depth = .15\baselineskip, align = left] at (group c2r1.west |- title) {maize\\protoplasts};
	
	
	%%% subfigure labels
	\subfiglabel[yshift = .5\columnsep]{TATApos}
	\subfiglabel[yshift = .5\columnsep]{enrTATAposLeaf}
	\subfiglabel[yshift = .5\columnsep]{enrTATAposProto}
	\subfiglabel[yshift = .5\columnsep]{variantsTATAmut}
	\subfiglabel[yshift = .5\columnsep]{enrTATAmut}
	\subfiglabel[yshift = .5\columnsep]{variantsTATAins}
	\subfiglabel[yshift = .5\columnsep]{enrTATAins}

\end{tikzpicture}%
			\caption{%
				\textbf{The TATA-box is a key determinant of promoter strength.}\titleend
				\subfigref{A} Histograms showing the percentage of promoters with a TATA-box at the indicated position. The region between positions $-59$ and $-23$ in which most TATA-boxes reside is highlighted in gray.\nextentry
				\subfigref{B}\subfigref{C} Violin plots (as defined in \autoref{fig:overview}) of promoter strength for libraries without enhancer in tobacco leaves \parensubfig{B} or maize protoplasts \parensubfig{C}. Promoters without a TATA-box ($-$) were compared to those with a TATA-box outside ($+$/$-$) or within ($+$/$+$) the $-59$ to $-23$ region.\nextentry
				\subfigrange{D}{G} Thirty plant promoters with a strong \parensubfig[\subfigref{D}]{E} or weak \parensubfig[\subfigref{F}]{G} TATA-box (WT) were tested. One (mutA and mutB) or two (mutAB) T>G mutations were inserted into promoters with a strong TATA-box \parensubfig[\subfigref{D}]{E}. A canonical TATA-box (+TATA) or one with a T>G mutation (+mutTATA) was inserted into promoters with a weak TATA-box \parensubfig[\subfigref{F}]{G}. Logoplots \parensubfig[\subfigref{F}]{D} of the TATA-box regions of these promoters  and their strength \parensubfig[\subfigref{G}]{E} relative to the WT promoter (set to 0, horizontal black line) are shown. Boxplots (center line, median; box limits, upper and lower quartiles; whiskers, 1.5 $\times$ interquartile range; points, outliers) denote the strength of the indicated promoter variants. Numbers at the bottom of the plot indicate the number of tested promoter elements. Significant differences from a null distribution were determined using the Wilcoxon rank-sum test and are indicated: \issignif, $p \leq \signiflevelone$; \issignif\issignif, $p \leq \signifleveltwo$; \issignif\issignif\issignif, $p \leq \signiflevelthree$; \notsignif, not significant.
			}%
			\label{fig:TATA}%
		\end{fig}

		\begin{fig}
			%\tikzset{external/export next = false}

\begin{tikzpicture}

	%%% enhancer responsiveness by tissue specificity (leaf)
	\coordinate (enhSpecLeaf) at (0, 0);
	
	\leafsymbol{enhSpecLeaf};
	
	\begin{hgroupplot}[%
		ylabel = enhancer responsiveness,
		ymin = -4,
		ymax = 11,
		ytick = {-10, -8, ..., 10},
		group position = {anchor = above north west, at = {(enhSpecLeaf)}, xshift = \plotylabelwidth},
		group/every plot/.append style = {x grids = false},
		zero line,
	]{\twocolumnwidth}{3}{tissue specificity}

		\nextgroupplot[
			title = Arabidopsis,
			x tick table = {rawData/enh-effect_specificity_At_leaf_boxplot.tsv}{LaTeX.label}
		]
			
		% violin and box plot
		\violinbox[violin color = arabidopsis]{rawData/enh-effect_specificity_At_leaf_boxplot.tsv}{rawData/enh-effect_specificity_At_leaf_violin.tsv}
			
		% add sample size
		\samplesize{rawData/enh-effect_specificity_At_leaf_boxplot.tsv}{id}{n}
		
		% add significance level
		\signif*{rawData/enh-effect_specificity_At_leaf_pvalues.tsv}{1}{2}
		\signif*{rawData/enh-effect_specificity_At_leaf_pvalues.tsv}{2}{3}
		
		
		\nextgroupplot[
			title = Maize,
			x tick table = {rawData/enh-effect_specificity_Zm_leaf_boxplot.tsv}{LaTeX.label}
		]
		
		% violin and box plot
		\violinbox[violin color = maize]{rawData/enh-effect_specificity_Zm_leaf_boxplot.tsv}{rawData/enh-effect_specificity_Zm_leaf_violin.tsv}
			
		% add sample size
		\samplesize{rawData/enh-effect_specificity_Zm_leaf_boxplot.tsv}{id}{n}
		
		% add significance level
		\signif*{rawData/enh-effect_specificity_Zm_leaf_pvalues.tsv}{1}{2}
		\signif*{rawData/enh-effect_specificity_Zm_leaf_pvalues.tsv}{2}{3}
		
		
		\nextgroupplot[
			title = Sorghum,
			x tick table = {rawData/enh-effect_specificity_Sb_leaf_boxplot.tsv}{LaTeX.label}
		]
			
		% violin and box plot
		\violinbox[violin color = sorghum]{rawData/enh-effect_specificity_Sb_leaf_boxplot.tsv}{rawData/enh-effect_specificity_Sb_leaf_violin.tsv}
		
		% add sample size
		\samplesize{rawData/enh-effect_specificity_Sb_leaf_boxplot.tsv}{id}{n}
		
		% add significance level
		\signif*{rawData/enh-effect_specificity_Sb_leaf_pvalues.tsv}{1}{2}
		\signif*{rawData/enh-effect_specificity_Sb_leaf_pvalues.tsv}{2}{3}
			
	\end{hgroupplot}
	
	
	%%% enhancer responsiveness by type (proto)
	\coordinate (enhSpecProto) at (enhSpecLeaf -| \textwidth - \twocolumnwidth, 0);
	
	\protosymbol{enhSpecProto};
	
	\begin{hgroupplot}[%
		ylabel = enhancer responsiveness,
		ymin = -4,
		ymax = 9.25,
		ytick = {-10, -8, ..., 10},
		group position = {anchor = above north west, at = {(enhSpecProto)}, xshift = \plotylabelwidth},
		group/every plot/.append style = {x grids = false},
		zero line,
	]{\twocolumnwidth}{3}{tissue specificity}

		\nextgroupplot[
			title = Arabidopsis,
			x tick table = {rawData/enh-effect_specificity_At_proto_boxplot.tsv}{LaTeX.label}
		]
			
		% violin and box plot
		\violinbox[violin color = arabidopsis]{rawData/enh-effect_specificity_At_proto_boxplot.tsv}{rawData/enh-effect_specificity_At_proto_violin.tsv}
			
		% add sample size
		\samplesize{rawData/enh-effect_specificity_At_proto_boxplot.tsv}{id}{n}
		
		% add significance level
		\signif*{rawData/enh-effect_specificity_At_proto_pvalues.tsv}{1}{2}
		\signif*{rawData/enh-effect_specificity_At_proto_pvalues.tsv}{2}{3}
		
		
		\nextgroupplot[
			title = Maize,
			x tick table = {rawData/enh-effect_specificity_Zm_proto_boxplot.tsv}{LaTeX.label}
		]
		
		% violin and box plot
		\violinbox[violin color = maize]{rawData/enh-effect_specificity_Zm_proto_boxplot.tsv}{rawData/enh-effect_specificity_Zm_proto_violin.tsv}
			
		% add sample size
		\samplesize{rawData/enh-effect_specificity_Zm_proto_boxplot.tsv}{id}{n}
		
		% add significance level
		\signif*{rawData/enh-effect_specificity_Zm_proto_pvalues.tsv}{1}{2}
		\signif*{rawData/enh-effect_specificity_Zm_proto_pvalues.tsv}{2}{3}
		
		
		\nextgroupplot[
			title = Sorghum,
			x tick table = {rawData/enh-effect_specificity_Sb_proto_boxplot.tsv}{LaTeX.label}
		]
			
		% violin and box plot
		\violinbox[violin color = sorghum]{rawData/enh-effect_specificity_Sb_proto_boxplot.tsv}{rawData/enh-effect_specificity_Sb_proto_violin.tsv}
		
		% add sample size
		\samplesize{rawData/enh-effect_specificity_Sb_proto_boxplot.tsv}{id}{n}
		
		% add significance level
		\signif*{rawData/enh-effect_specificity_Sb_proto_pvalues.tsv}{1}{2}
		\signif*{rawData/enh-effect_specificity_Sb_proto_pvalues.tsv}{2}{3}
			
	\end{hgroupplot}


	%%% enhancer responsiveness by TATA position (tobacco)
	\coordinate[yshift = -\columnsep] (enhTATAposLeaf) at (enhSpecLeaf |- plot xlabel.south);

	\leafsymbol{enhTATAposLeaf};
	
	\begin{hgroupplot}[%
		ylabel = enhancer responsiveness,
		ymin = -4,
		ymax = 11,
		ytick = {-10, -8, ..., 10},
		xticklabel style = {name = xticklabel, inner sep = .1em},
		group position = {anchor = above north west, at = {(enhTATAposLeaf)}, xshift = \plotylabelwidth},
		group/every plot/.append style = {x grids = false, typeset ticklabels with strut},
		zero line,
	]{\twocolumnwidth}{3}{}

		\nextgroupplot[
			title = Arabidopsis,
			x tick table = {rawData/enh-effect_TATA-pos_At_leaf_boxplot.tsv}{LaTeX.label}
		]
			
		% violin and box plot
		\violinbox[violin color = arabidopsis]{rawData/enh-effect_TATA-pos_At_leaf_boxplot.tsv}{rawData/enh-effect_TATA-pos_At_leaf_violin.tsv}
			
		% add sample size
		\samplesize{rawData/enh-effect_TATA-pos_At_leaf_boxplot.tsv}{id}{n}
		
		% add significance level
		\signif*{rawData/enh-effect_TATA-pos_At_leaf_pvalues.tsv}{1}{2}
		\signif*{rawData/enh-effect_TATA-pos_At_leaf_pvalues.tsv}{2}{3}
		
		\pgfplotsinvokeforeach{1, ..., 6}{
			\coordinate (A#1) at (#1, 0);
		}
		
		
		\nextgroupplot[
			title = Maize,
			x tick table = {rawData/enh-effect_TATA-pos_Zm_leaf_boxplot.tsv}{LaTeX.label}
		]
		
		% violin and box plot
		\violinbox[violin color = maize]{rawData/enh-effect_TATA-pos_Zm_leaf_boxplot.tsv}{rawData/enh-effect_TATA-pos_Zm_leaf_violin.tsv}
			
		% add sample size
		\samplesize{rawData/enh-effect_TATA-pos_Zm_leaf_boxplot.tsv}{id}{n}
		
		% add significance level
		\signif*{rawData/enh-effect_TATA-pos_Zm_leaf_pvalues.tsv}{1}{2}
		\signif*{rawData/enh-effect_TATA-pos_Zm_leaf_pvalues.tsv}{2}{3}
		
		\pgfplotsinvokeforeach{1, ..., 6}{
			\coordinate (Z#1) at (#1, 0);
		}
		
		
		\nextgroupplot[
			title = Sorghum,
			x tick table = {rawData/enh-effect_TATA-pos_Sb_leaf_boxplot.tsv}{LaTeX.label}
		]
			
		% violin and box plot
		\violinbox[violin color = sorghum]{rawData/enh-effect_TATA-pos_Sb_leaf_boxplot.tsv}{rawData/enh-effect_TATA-pos_Sb_leaf_violin.tsv}
		
		% add sample size
		\samplesize{rawData/enh-effect_TATA-pos_Sb_leaf_boxplot.tsv}{id}{n}
		
		% add significance level
		\signif*{rawData/enh-effect_TATA-pos_Sb_leaf_pvalues.tsv}{1}{2}
		\signif*{rawData/enh-effect_TATA-pos_Sb_leaf_pvalues.tsv}{2}{3}
		
		\pgfplotsinvokeforeach{1, ..., 6}{
			\coordinate (S#1) at (#1, 0);
		}
			
	\end{hgroupplot}
	
	\node[anchor = base east, inner xsep = 0pt, node font = \figsmall] (lTATA) at (group c1r1.west |- xticklabel.base) {TATA-box};
	\node[anchor = north west, inner xsep = 0pt, node font = \figsmall] (lpos) at (lTATA.base west) {at $-59$ to $-23$};
	\node[anchor = base, node font = \figsmall] at (lpos.base -| A2) {$-$};
	\node[anchor = base, node font = \figsmall] at (lpos.base -| A3) {$+$};
	\node[anchor = base, node font = \figsmall] at (lpos.base -| Z2) {$-$};
	\node[anchor = base, node font = \figsmall] at (lpos.base -| Z3) {$+$};
	\node[anchor = base, node font = \figsmall] at (lpos.base -| S2) {$-$};
	\node[anchor = base, node font = \figsmall] at (lpos.base -| S3) {$+$};
	
	
	%%% enh-effect by TATA pos (protoplasts)
	\coordinate (enhTATAposProto) at (enhTATAposLeaf -| enhSpecProto);
	
	\protosymbol{enhTATAposProto};
	
	\begin{hgroupplot}[%
		ylabel = enhancer responsiveness,
		ymin = -4,
		ymax = 9.25,
		ytick = {-10, -8, ..., 10},
		xticklabel style = {name = xticklabel, inner sep = .1em},
		group position = {anchor = above north west, at = {(enhTATAposProto)}, xshift = \plotylabelwidth},
		group/every plot/.append style = {x grids = false, typeset ticklabels with strut},
		zero line,
	]{\twocolumnwidth}{3}{}

		\nextgroupplot[
			title = Arabidopsis,
			x tick table = {rawData/enh-effect_TATA-pos_At_proto_boxplot.tsv}{LaTeX.label}
		]
			
		% violin and box plot
		\violinbox[violin color = arabidopsis]{rawData/enh-effect_TATA-pos_At_proto_boxplot.tsv}{rawData/enh-effect_TATA-pos_At_proto_violin.tsv}
			
		% add sample size
		\samplesize{rawData/enh-effect_TATA-pos_At_proto_boxplot.tsv}{id}{n}
		
		% add significance level
		\signif*{rawData/enh-effect_TATA-pos_At_proto_pvalues.tsv}{1}{2}
		\signif*{rawData/enh-effect_TATA-pos_At_proto_pvalues.tsv}{2}{3}
		
		\pgfplotsinvokeforeach{1, ..., 6}{
			\coordinate (A#1) at (#1, 0);
		}
		
		
		\nextgroupplot[
			title = Maize,
			x tick table = {rawData/enh-effect_TATA-pos_Zm_proto_boxplot.tsv}{LaTeX.label}
		]
		
		% violin and box plot
		\violinbox[violin color = maize]{rawData/enh-effect_TATA-pos_Zm_proto_boxplot.tsv}{rawData/enh-effect_TATA-pos_Zm_proto_violin.tsv}
			
		% add sample size
		\samplesize{rawData/enh-effect_TATA-pos_Zm_proto_boxplot.tsv}{id}{n}
		
		% add significance level
		\signif*{rawData/enh-effect_TATA-pos_Zm_proto_pvalues.tsv}{1}{2}
		\signif*{rawData/enh-effect_TATA-pos_Zm_proto_pvalues.tsv}{2}{3}
		
		\pgfplotsinvokeforeach{1, ..., 6}{
			\coordinate (Z#1) at (#1, 0);
		}
		
		
		\nextgroupplot[
			title = Sorghum,
			x tick table = {rawData/enh-effect_TATA-pos_Sb_proto_boxplot.tsv}{LaTeX.label}
		]
			
		% violin and box plot
		\violinbox[violin color = sorghum]{rawData/enh-effect_TATA-pos_Sb_proto_boxplot.tsv}{rawData/enh-effect_TATA-pos_Sb_proto_violin.tsv}
		
		% add sample size
		\samplesize{rawData/enh-effect_TATA-pos_Sb_proto_boxplot.tsv}{id}{n}
		
		% add significance level
		\signif*{rawData/enh-effect_TATA-pos_Sb_proto_pvalues.tsv}{1}{2}
		\signif*{rawData/enh-effect_TATA-pos_Sb_proto_pvalues.tsv}{2}{3}
		
		\pgfplotsinvokeforeach{1, ..., 6}{
			\coordinate (S#1) at (#1, 0);
		}
			
	\end{hgroupplot}
	
	\node[anchor = base east, inner xsep = 0pt, node font = \figsmall] (lTATA) at (group c1r1.west |- xticklabel.base) {TATA-box};
	\node[anchor = north west, inner xsep = 0pt, node font = \figsmall] (lpos) at (lTATA.base west) {at $-59$ to $-23$};
	\node[anchor = base, node font = \figsmall] at (lpos.base -| A2) {$-$};
	\node[anchor = base, node font = \figsmall] at (lpos.base -| A3) {$+$};
	\node[anchor = base, node font = \figsmall] at (lpos.base -| Z2) {$-$};
	\node[anchor = base, node font = \figsmall] at (lpos.base -| Z3) {$+$};
	\node[anchor = base, node font = \figsmall] at (lpos.base -| S2) {$-$};
	\node[anchor = base, node font = \figsmall] at (lpos.base -| S3) {$+$};
	
	
	%%% enhancer responsiveness by GC (tobacco)
	\coordinate[yshift = -\columnsep] (enhGCleaf) at (enhSpecLeaf |- lpos.south);
	
	\leafsymbol{enhGCleaf};
	
	\begin{hgroupplot}[%
		ylabel = enhancer responsiveness,
		ymin = -4,
		ymax = 11,
		ytick = {-10, -8, ..., 10},
		group position = {anchor = above north west, at = {(enhGCleaf)}, xshift = \plotylabelwidth},
		group/every plot/.append style = {x grids = false},
		zero line,
	]{\twocolumnwidth}{3}{GC content}

			
		\nextgroupplot[
			title = Arabidopsis,
			x tick table = {rawData/enh-effect_GC_At_leaf_boxplot.tsv}{LaTeX.label}
		]
			
		% violin and box plot
		\violinbox[violin color = arabidopsis]{rawData/enh-effect_GC_At_leaf_boxplot.tsv}{rawData/enh-effect_GC_At_leaf_violin.tsv}
			
		% add sample size
		\samplesize{rawData/enh-effect_GC_At_leaf_boxplot.tsv}{id}{n}
		
		% add significance level
		\signif*{rawData/enh-effect_GC_At_leaf_pvalues.tsv}{1}{2}
		\signif*{rawData/enh-effect_GC_At_leaf_pvalues.tsv}{2}{3}
		\signif*{rawData/enh-effect_GC_At_leaf_pvalues.tsv}{3}{4}
		\signif*{rawData/enh-effect_GC_At_leaf_pvalues.tsv}{4}{5}
		
		
		\nextgroupplot[
			title = Maize,
			x tick table = {rawData/enh-effect_GC_Zm_leaf_boxplot.tsv}{LaTeX.label}
		]
		
		% violin and box plot
		\violinbox[violin color = maize]{rawData/enh-effect_GC_Zm_leaf_boxplot.tsv}{rawData/enh-effect_GC_Zm_leaf_violin.tsv}
			
		% add sample size
		\samplesize{rawData/enh-effect_GC_Zm_leaf_boxplot.tsv}{id}{n}
		
		% add significance level
		\signif*{rawData/enh-effect_GC_Zm_leaf_pvalues.tsv}{1}{2}
		\signif*{rawData/enh-effect_GC_Zm_leaf_pvalues.tsv}{2}{3}
		\signif*{rawData/enh-effect_GC_Zm_leaf_pvalues.tsv}{3}{4}
		\signif*{rawData/enh-effect_GC_Zm_leaf_pvalues.tsv}{4}{5}
		
		
		\nextgroupplot[
			title = Sorghum,
			x tick table = {rawData/enh-effect_GC_Sb_leaf_boxplot.tsv}{LaTeX.label}
		]
			
		% violin and box plot
		\violinbox[violin color = sorghum]{rawData/enh-effect_GC_Sb_leaf_boxplot.tsv}{rawData/enh-effect_GC_Sb_leaf_violin.tsv}
		
		% add sample size
		\samplesize{rawData/enh-effect_GC_Sb_leaf_boxplot.tsv}{id}{n}
		
		% add significance level
		\signif*{rawData/enh-effect_GC_Sb_leaf_pvalues.tsv}{1}{2}
		\signif*{rawData/enh-effect_GC_Sb_leaf_pvalues.tsv}{2}{3}
		\signif*{rawData/enh-effect_GC_Sb_leaf_pvalues.tsv}{3}{4}
		\signif*{rawData/enh-effect_GC_Sb_leaf_pvalues.tsv}{4}{5}
			
	\end{hgroupplot}


	%%% enhancer responsiveness by GC (protoplasts)
	\coordinate (enhGCproto) at (enhSpecProto |- enhGCleaf);
	
	\protosymbol{enhGCproto};
	
	\begin{hgroupplot}[%
		ylabel = enhancer responsiveness,
		ymin = -4,
		ymax = 9.25,
		ytick = {-10, -8, ..., 10},
		group position = {anchor = above north west, at = {(enhGCproto)}, xshift = \plotylabelwidth},
		group/every plot/.append style = {x grids = false},
		zero line,
	]{\twocolumnwidth}{3}{GC content}

			
		\nextgroupplot[
			title = Arabidopsis,
			x tick table = {rawData/enh-effect_GC_At_proto_boxplot.tsv}{LaTeX.label}
		]
			
		% violin and box plot
		\violinbox[violin color = arabidopsis]{rawData/enh-effect_GC_At_proto_boxplot.tsv}{rawData/enh-effect_GC_At_proto_violin.tsv}
			
		% add sample size
		\samplesize{rawData/enh-effect_GC_At_proto_boxplot.tsv}{id}{n}
		
		% add significance level
		\signif*{rawData/enh-effect_GC_At_proto_pvalues.tsv}{1}{2}
		\signif*{rawData/enh-effect_GC_At_proto_pvalues.tsv}{2}{3}
		\signif*{rawData/enh-effect_GC_At_proto_pvalues.tsv}{3}{4}
		\signif*{rawData/enh-effect_GC_At_proto_pvalues.tsv}{4}{5}
		
		
		\nextgroupplot[
			title = Maize,
			x tick table = {rawData/enh-effect_GC_Zm_proto_boxplot.tsv}{LaTeX.label}
		]
		
		% violin and box plot
		\violinbox[violin color = maize]{rawData/enh-effect_GC_Zm_proto_boxplot.tsv}{rawData/enh-effect_GC_Zm_proto_violin.tsv}
			
		% add sample size
		\samplesize{rawData/enh-effect_GC_Zm_proto_boxplot.tsv}{id}{n}
		
		% add significance level
		\signif*{rawData/enh-effect_GC_Zm_proto_pvalues.tsv}{1}{2}
		\signif*{rawData/enh-effect_GC_Zm_proto_pvalues.tsv}{2}{3}
		\signif*{rawData/enh-effect_GC_Zm_proto_pvalues.tsv}{3}{4}
		\signif*{rawData/enh-effect_GC_Zm_proto_pvalues.tsv}{4}{5}
		
		
		\nextgroupplot[
			title = Sorghum,
			x tick table = {rawData/enh-effect_GC_Sb_proto_boxplot.tsv}{LaTeX.label}
		]
			
		% violin and box plot
		\violinbox[violin color = sorghum]{rawData/enh-effect_GC_Sb_proto_boxplot.tsv}{rawData/enh-effect_GC_Sb_proto_violin.tsv}
		
		% add sample size
		\samplesize{rawData/enh-effect_GC_Sb_proto_boxplot.tsv}{id}{n}
		
		% add significance level
		\signif*{rawData/enh-effect_GC_Sb_proto_pvalues.tsv}{1}{2}
		\signif*{rawData/enh-effect_GC_Sb_proto_pvalues.tsv}{2}{3}
		\signif*{rawData/enh-effect_GC_Sb_proto_pvalues.tsv}{3}{4}
		\signif*{rawData/enh-effect_GC_Sb_proto_pvalues.tsv}{4}{5}
			
	\end{hgroupplot}
	
	
	%%% subfigure labels
	\subfiglabel[yshift = .5\columnsep]{enhSpecLeaf}
	\subfiglabel[yshift = .5\columnsep]{enhSpecProto}
	\subfiglabel[yshift = .5\columnsep]{enhTATAposLeaf}
	\subfiglabel[yshift = .5\columnsep]{enhTATAposProto}
	\subfiglabel[yshift = .5\columnsep]{enhGCleaf}
	\subfiglabel[yshift = .5\columnsep]{enhGCproto}

\end{tikzpicture}%
			\caption{%
				\textbf{Enhancer responsiveness of promoters depends on the TATA-box and GC content.}\titleend
				\subfigref{A}\subfigref{B} Violin plots (as defined in \autoref{fig:overview}) of enhancer responsiveness (promoter strength\textsuperscript{with enhancer} $-$ promoter strength\textsuperscript{without enhancer}) in tobacco leaves \parensubfig{A} or maize protoplasts \parensubfig{B}. Promoters were grouped into three bins of approximately similar size according to the tissue-specificity $\tau$ (Yanai et al., 2005) of the expression of the associated gene.\nextentry
				\subfigref{C}\subfigref{D} Violin plots of enhancer responsiveness in tobacco leaves \parensubfig{C} or maize protoplasts \parensubfig{D}. Promoters without a TATA-box ($-$) were compared to those with a TATA-box outside ($+$/$-$) or within ($+$/$+$) the $-59$ to $-23$ region.\nextentry
				\subfigref{E}\subfigref{F} Violin plots of enhancer responsiveness in tobacco leaves \parensubfig{E} or maize protoplasts \parensubfig{F} for promoters grouped by GC content.
			}%
			\label{fig:enhEffect}
		\end{fig}
		
		\begin{fig}
			%\tikzset{external/export next = false}

\begin{tikzpicture}
	
	%%% scheme
	\coordinate (scheme) at (0, 0);
	
	\coordinate (construct1) at ($(scheme) + (.25, -.25)$);
	\coordinate (construct2) at ($(scheme) + (.1, -.75)$);
	
	\STARRconstructShort[barcode = Orchid1, promoter = Aquamarine2!80!black, promoter strength = weak, enhancer = {draw = none}]{construct1};
	\STARRconstructShort[barcode = MediumPurple2, promoter = RoyalBlue2!75!black, promoter strength = strong, enhancer = {draw = none}]{construct2};
	
	\coordinate (leaves) at ($(scheme |- construct2) + (.5\fourcolumnwidth, -1.5)$);
	
	\begin{pgfinterruptboundingbox}
		\leaf[0.35]{$(leaves) + (-.25\fourcolumnwidth + .125cm, 0)$};
		\leaf[0.35]{$(leaves) + (.25\fourcolumnwidth + .275cm, 0)$};
	\end{pgfinterruptboundingbox}
	
	\coordinate (lightBar) at ($(leaves) + (-.5\fourcolumnwidth + .1cm, .75)$);
	\fill[Gold1] (lightBar) rectangle ++(.25, -1.5);
	\fill[Gold1, fill opacity = .2] (lightBar) ++(.25, .5\pgflinewidth) rectangle ++(.5\fourcolumnwidth - .5cm + .5\pgflinewidth, -.375cm - .5\pgflinewidth);
	\fill[Gold1, fill opacity = .2] (lightBar) ++(.25, -.625) rectangle ++(.5\fourcolumnwidth - .5cm + .5\pgflinewidth, -.5);
	\fill[Gold1, fill opacity = .2] (lightBar) ++(.25, -1.375) rectangle ++(.5\fourcolumnwidth - .5cm + .5\pgflinewidth, -.125cm - .5\pgflinewidth);
	\fill[black] (lightBar) ++(0, -.375) rectangle ++(.25, -.25);
	\fill[black, fill opacity = .2] (lightBar) ++(.25, -.375) rectangle ++(.5\fourcolumnwidth - .5cm + .5\pgflinewidth, -.25);
	\fill[black] (lightBar) ++(0, -1.125) rectangle ++(.25, -.25);
	\fill[black, fill opacity = .2] (lightBar) ++(.25, -1.125) rectangle ++(.5\fourcolumnwidth - .5cm + .5\pgflinewidth, -.25);
	\draw[black, thin] (lightBar) rectangle ++(.25, -1.5);
	
	\coordinate (darkBar) at ($(lightBar) + (.5\fourcolumnwidth + .15cm, 0)$);
	\fill[black, opacity = .2, thin] (darkBar) ++(.25, .5\pgflinewidth) rectangle ++(.5\fourcolumnwidth - .5cm + .5\pgflinewidth, -1.5cm - \pgflinewidth);
	\draw[black, fill, thin] (darkBar) rectangle ++(.25, -1.5);
	
	\node[anchor = south west, text depth = 0pt] at (lightBar) {light};
	\node[anchor = south east, text depth = 0pt, xshift = .5\fourcolumnwidth - .25cm] at (darkBar) {dark};
	
	\draw[arrow] (ORF.south -| leaves) ++(-.1\fourcolumnwidth, -.1) -- ++(-.4, -.5) coordinate (a1);
	\draw[arrow] (ORF.south -| leaves) ++(.1\fourcolumnwidth + .15cm, -.1) -- ++(.4, -.5) coordinate (a2);
	
%	\draw[arrow] (a1) ++(0, -1.45) -- ++(.4, -.5);
%	\draw[arrow] (a2) ++(0, -1.45) -- ++(-.4, -.5);
%	
%	\coordinate (transcripts) at ($(leaves) + (-.25\fourcolumnwidth, -1.5)$);
%	
%	\transcript[barcode = MediumPurple2]{transcripts};
%	\transcript[barcode = Orchid1]{$(transcripts) + (-.5, -.33)$};
%	\transcript[barcode = MediumPurple2]{$(transcripts) + (-.17, -.67)$};
	
	
	%%% GO-terms
	\coordinate[yshift = -\columnsep - 1.5cm] (GOterms) at (lightBar -| scheme);
	
	\lightsymbol{GOterms};

	\begin{axis}[%
		width = .5\fourcolumnwidth - 16pt,
		height = 32pt,
		at = {(GOterms)},
		xshift = \fourcolumnwidth,
		anchor = right of north east,
		xlabel = {$-\log_{10}(p\ \textsf{value})$},
		xmin = 0,
		xbar = 0pt,
		bar width = 0.25,
		y dir = reverse,
		legend style = {name = legend, draw = none, anchor = west, at = {(1, .5)}},
		y grids = false,
		y tick table = {rawData/GO_light-dep.tsv}{term_name},
		xticklabel style = {name = xticklabel}
	]
	
		\addlegendimage{xbar legend 1, fill = arabidopsis, fill opacity = .5}
		\addlegendimage{xbar legend 2, fill = maize, fill opacity = .5}
		\addlegendimage{xbar legend 3, fill = sorghum, fill opacity = .5}
		
		\legend{At, Zm, Sb}
		
		\addplot [draw = black, fill = gray, fill opacity = .5, forget plot] table [x = ns_At, y expr = \lineno] {rawData/GO_light-dep.tsv};
		\addplot [draw = black, fill = arabidopsis, fill opacity = .5] table [x = p_At, y expr = \lineno] {rawData/GO_light-dep.tsv};
		
		\addplot [draw = black, fill = gray, fill opacity = .5, forget plot] table [x = ns_Zm, y expr = \lineno] {rawData/GO_light-dep.tsv};
		\addplot [draw = black, fill = maize, fill opacity = .5] table [x = p_Zm, y expr = \lineno] {rawData/GO_light-dep.tsv};
		
		\addplot [draw = black, fill = gray, fill opacity = .5, forget plot] table [x = ns_Sb, y expr = \lineno] {rawData/GO_light-dep.tsv};
		\addplot [draw = black, fill = sorghum, fill opacity = .5] table [x = p_Sb, y expr = \lineno] {rawData/GO_light-dep.tsv};
		
		\addplot [sharp plot, red, update limits = false] coordinates {(1.3, 0) (1.3, 6)};
	
	\end{axis}
	
	\draw[thin] (last plot.north -| legend.west) rectangle (last plot.south -| legend.east) coordinate (c1);
	
%	\distance{legend.north}{legend.south}
%	\node[anchor = north east] at (last plot.outer south) {\convertto{cm}{\ydistance}; \the\ydistance};
	
	
	%%% light-dependency by enhancer
	\coordinate (lightDepEnh) at (scheme -| \twocolumnwidth - \fourcolumnwidth, 0);
	
	\lightsymbol{lightDepEnh};
	
	\begin{hgroupplot}[%
		ylabel = light-dependency,
		height = 3.5cm + 1.25\baselineskip + 8.08pt,
		ymin = -6.75,
		ymax = 7.5,
		ytick = {-10, -8, ..., 10},
		group position = {anchor = above north west, at = {(lightDepEnh)}, xshift = \plotylabelwidth},
		group/every plot/.append style = {x grids = false, typeset ticklabels with strut},
		xticklabel style = {inner sep = .1em},
		zero line,
	]{\fourcolumnwidth}{3}{35S enhancer}

			
		\nextgroupplot[
			title = At,
			x tick table = {rawData/light-dep_enh_At_boxplot.tsv}{LaTeX.label}
		]
			
		% violin and box plot
		\violinbox[violin color = arabidopsis]{rawData/light-dep_enh_At_boxplot.tsv}{rawData/light-dep_enh_At_violin.tsv}
			
		% add sample size
		\samplesize[scatter, no marks, visualization depends on = {mod(x, 2) * .5\baselineskip \as \shift}, scatter/@pre marker code/.append style = {/tikz/yshift = \shift}]{rawData/light-dep_enh_At_boxplot.tsv}{id}{n}
		
		
		\nextgroupplot[
			title = Zm,
			x tick table = {rawData/light-dep_enh_Zm_boxplot.tsv}{LaTeX.label}
		]
		
		% violin and box plot
		\violinbox[violin color = maize]{rawData/light-dep_enh_Zm_boxplot.tsv}{rawData/light-dep_enh_Zm_violin.tsv}
			
		% add sample size
		\samplesize[scatter, no marks, visualization depends on = {mod(x, 2) * .5\baselineskip \as \shift}, scatter/@pre marker code/.append style = {/tikz/yshift = \shift}]{rawData/light-dep_enh_Zm_boxplot.tsv}{id}{n}
		
		
		\nextgroupplot[
			title = Sb,
			x tick table = {rawData/light-dep_enh_Sb_boxplot.tsv}{LaTeX.label}
		]
			
		% violin and box plot
		\violinbox[violin color = sorghum]{rawData/light-dep_enh_Sb_boxplot.tsv}{rawData/light-dep_enh_Sb_violin.tsv}
		
		% add sample size
		\samplesize[scatter, no marks, visualization depends on = {mod(x, 2) * .5\baselineskip \as \shift}, scatter/@pre marker code/.append style = {/tikz/yshift = \shift}]{rawData/light-dep_enh_Sb_boxplot.tsv}{id}{n}
			
	\end{hgroupplot}
	
%	\distance{group c1r1.south}{c1}
%	\node[anchor = north east] at (last plot.outer south) {\convertto{cm}{\ydistance}; \the\ydistance};
	
		
	%%% light-dependency by TATA and GC
	\coordinate (lightDepTATAGC) at (\textwidth - \twocolumnwidth, 0 |- lightDepEnh);
	
	\lightsymbol{lightDepTATAGC}; 
	
	\begin{hgroupplot}[%
		ylabel = light-dependency,
		ymin = -6.75,
		ymax = 9,
		ytick = {-10, -8, ..., 10},
		group position = {anchor = above north west, at = {(lightDepTATAGC)}, xshift = \plotylabelwidth, yshift = -1.25\baselineskip},
		group/every plot/.append style = {x grids = false},
		zero line,
	]{\twocolumnwidth}{3}{GC content}
	
			
		\nextgroupplot[
			title = Arabidopsis,
			x tick table half = {rawData/light-dep_TATA+GC_At_boxplot.tsv}{LaTeX.label}
		]
			
		% half violin and box plot
		\halfviolinbox[violin color half = arabidopsis]{rawData/light-dep_TATA+GC_At_boxplot.tsv}{rawData/light-dep_TATA+GC_At_violin.tsv}
			
		% add sample size
		\samplesizehalf[violin color half = arabidopsis]{rawData/light-dep_TATA+GC_At_boxplot.tsv}{id}{n}
		
		% add pvalues
		\signifall*{rawData/light-dep_TATA+GC_At_pvalues.tsv}	
		
		
		\nextgroupplot[
			title = Maize,
			x tick table half = {rawData/light-dep_TATA+GC_Zm_boxplot.tsv}{LaTeX.label}
		]
			
		% half violin and box plot
		\halfviolinbox[violin color half = maize]{rawData/light-dep_TATA+GC_Zm_boxplot.tsv}{rawData/light-dep_TATA+GC_Zm_violin.tsv}
			
		% add sample size
		\samplesizehalf[violin color half = maize]{rawData/light-dep_TATA+GC_Zm_boxplot.tsv}{id}{n}
		
		% add pvalues
		\signifall*{rawData/light-dep_TATA+GC_Zm_pvalues.tsv}
		
		
		\nextgroupplot[
			title = Sorghum,
			x tick table half = {rawData/light-dep_TATA+GC_Sb_boxplot.tsv}{LaTeX.label}
		]
			
		% half violin and box plot
		\halfviolinbox[violin color half = sorghum]{rawData/light-dep_TATA+GC_Sb_boxplot.tsv}{rawData/light-dep_TATA+GC_Sb_violin.tsv}
			
		% add sample size
		\samplesizehalf[violin color half = sorghum]{rawData/light-dep_TATA+GC_Sb_boxplot.tsv}{id}{n}
		
		% add pvalues
		\signifall*{rawData/light-dep_TATA+GC_Sb_pvalues.tsv}	

	\end{hgroupplot}
	
	\node[anchor = south west] (legend) at (group c1r1.above north west) {\textbf{TATA-box} motif:\vphantom{g} $-$};
	\coordinate (legend symbol) at ($(legend.south east) + (.25em, .2em)$);
	\node[anchor = base west] at ($(legend.base east) + (.6em, 0)$) {$+$};
	
	\fill[thin, draw = black, /pgfplots/violin color half = arabidopsis, fill = viocolleft, fill opacity = .5, x = .25em, y = .1em, rotate = 90, shift = {(legend symbol)}] plot[domain = 0:6] (\x,{4*1/exp(((\x-3)^2)/2)}) -- cycle;
	\fill[thin, draw = black, /pgfplots/violin color half = arabidopsis, fill = viocolright, fill opacity = .5, x = .25em, y = -.1em, rotate = 90, shift = {(legend symbol)}, yshift = -.1em] plot[domain = 0:6] (\x,{4*1/exp(((\x-3)^2)/2)}) -- cycle;
	
	
	%%% light-dependency by TCP binding site
	\coordinate[yshift = -\columnsep] (lightDepTCP) at (scheme |- plot xlabel.south);
	
	\lightsymbol{lightDepTCP}; 
	
	\begin{hgroupplot}[%
		ylabel = light-dependency,
		ymin = -6.75,
		ymax = 9,
		ytick = {-10, -8, ..., 10},
		group position = {anchor = above north west, at = {(lightDepTCP)}, xshift = \plotylabelwidth, yshift = -1.25\baselineskip},
		group/every plot/.append style = {x grids = false},
		zero line,
	]{\twocolumnwidth}{3}{GC content}
	
			
		\nextgroupplot[
			title = Arabidopsis,
			x tick table half = {rawData/light-dep_TCP_At_boxplot.tsv}{LaTeX.label}
		]
			
		% half violin and box plot
		\halfviolinbox[violin color half = arabidopsis]{rawData/light-dep_TCP_At_boxplot.tsv}{rawData/light-dep_TCP_At_violin.tsv}
			
		% add sample size
		\samplesizehalf[violin color half = arabidopsis]{rawData/light-dep_TCP_At_boxplot.tsv}{id}{n}
		
		% add pvalues
		\signifall*{rawData/light-dep_TCP_At_pvalues.tsv}	
		
		
		\nextgroupplot[
			title = Maize,
			x tick table half = {rawData/light-dep_TCP_Zm_boxplot.tsv}{LaTeX.label}
		]
			
		% half violin and box plot
		\halfviolinbox[violin color half = maize]{rawData/light-dep_TCP_Zm_boxplot.tsv}{rawData/light-dep_TCP_Zm_violin.tsv}
			
		% add sample size
		\samplesizehalf[violin color half = maize]{rawData/light-dep_TCP_Zm_boxplot.tsv}{id}{n}
		
		% add pvalues
		\signifall*{rawData/light-dep_TCP_Zm_pvalues.tsv}
		
		
		\nextgroupplot[
			title = Sorghum,
			x tick table half = {rawData/light-dep_TCP_Sb_boxplot.tsv}{LaTeX.label}
		]
			
		% half violin and box plot
		\halfviolinbox[violin color half = sorghum]{rawData/light-dep_TCP_Sb_boxplot.tsv}{rawData/light-dep_TCP_Sb_violin.tsv}
			
		% add sample size
		\samplesizehalf[violin color half = sorghum]{rawData/light-dep_TCP_Sb_boxplot.tsv}{id}{n}
		
		% add pvalues
		\signifall*{rawData/light-dep_TCP_Sb_pvalues.tsv}	

	\end{hgroupplot}

	\node[anchor = south west] (legend) at (group c1r1.above north west) {\textbf{TCP} transcripition factor binding site: $-$};
	\coordinate (legend symbol) at ($(legend.south east) + (.25em, .2em)$);
	\node[anchor = base west] at ($(legend.base east) + (.6em, 0)$) {$+$};
	
	\fill[thin, draw = black, /pgfplots/violin color half = arabidopsis, fill = viocolleft, fill opacity = .5, x = .25em, y = .1em, rotate = 90, shift = {(legend symbol)}] plot[domain = 0:6] (\x,{4*1/exp(((\x-3)^2)/2)}) -- cycle;
	\fill[thin, draw = black, /pgfplots/violin color half = arabidopsis, fill = viocolright, fill opacity = .5, x = .25em, y = -.1em, rotate = 90, shift = {(legend symbol)}, yshift = -.1em] plot[domain = 0:6] (\x,{4*1/exp(((\x-3)^2)/2)}) -- cycle;
	
	
	%%% light-dependency by WRKY binding site
	\coordinate (lightDepWRKY) at (lightDepTATAGC |- lightDepTCP);
	
	\lightsymbol{lightDepWRKY}; 
	
	\begin{hgroupplot}[%
		ylabel = light-dependency,
		ymin = -6.75,
		ymax = 9,
		ytick = {-10, -8, ..., 10},
		group position = {anchor = above north west, at = {(lightDepWRKY)}, xshift = \plotylabelwidth, yshift = -1.25\baselineskip},
		group/every plot/.append style = {x grids = false},
		zero line,
	]{\twocolumnwidth}{3}{GC content}
	
			
		\nextgroupplot[
			title = Arabidopsis,
			x tick table half = {rawData/light-dep_WRKY_At_boxplot.tsv}{LaTeX.label}
		]
			
		% half violin and box plot
		\halfviolinbox[violin color half = arabidopsis]{rawData/light-dep_WRKY_At_boxplot.tsv}{rawData/light-dep_WRKY_At_violin.tsv}
			
		% add sample size
		\samplesizehalf[violin color half = arabidopsis]{rawData/light-dep_WRKY_At_boxplot.tsv}{id}{n}
		
		% add pvalues
		\signifall*{rawData/light-dep_WRKY_At_pvalues.tsv}	
		
		
		\nextgroupplot[
			title = Maize,
			x tick table half = {rawData/light-dep_WRKY_Zm_boxplot.tsv}{LaTeX.label}
		]
			
		% half violin and box plot
		\halfviolinbox[violin color half = maize]{rawData/light-dep_WRKY_Zm_boxplot.tsv}{rawData/light-dep_WRKY_Zm_violin.tsv}
			
		% add sample size
		\samplesizehalf[violin color half = maize]{rawData/light-dep_WRKY_Zm_boxplot.tsv}{id}{n}
		
		% add pvalues
		\signifall*{rawData/light-dep_WRKY_Zm_pvalues.tsv}
		
		
		\nextgroupplot[
			title = Sorghum,
			x tick table half = {rawData/light-dep_WRKY_Sb_boxplot.tsv}{LaTeX.label}
		]
			
		% half violin and box plot
		\halfviolinbox[violin color half = sorghum]{rawData/light-dep_WRKY_Sb_boxplot.tsv}{rawData/light-dep_WRKY_Sb_violin.tsv}
			
		% add sample size
		\samplesizehalf[violin color half = sorghum]{rawData/light-dep_WRKY_Sb_boxplot.tsv}{id}{n}
		
		% add pvalues
		\signifall*{rawData/light-dep_WRKY_Sb_pvalues.tsv}	

	\end{hgroupplot}
	
	\node[anchor = south west] (legend) at (group c1r1.above north west) {\textbf{WRKY/C3H} transcripition factor binding site: $-$};
	\coordinate (legend symbol) at ($(legend.south east) + (.25em, .2em)$);
	\node[anchor = base west] at ($(legend.base east) + (.6em, 0)$) {$+$};
	
	\fill[thin, draw = black, /pgfplots/violin color half = arabidopsis, fill = viocolleft, fill opacity = .5, x = .25em, y = .1em, rotate = 90, shift = {(legend symbol)}] plot[domain = 0:6] (\x,{4*1/exp(((\x-3)^2)/2)}) -- cycle;
	\fill[thin, draw = black, /pgfplots/violin color half = arabidopsis, fill = viocolright, fill opacity = .5, x = .25em, y = -.1em, rotate = 90, shift = {(legend symbol)}, yshift = -.1em] plot[domain = 0:6] (\x,{4*1/exp(((\x-3)^2)/2)}) -- cycle;
	
	
	%%% subfigure labels
	\subfiglabel[yshift = .5\columnsep]{scheme}
	\subfiglabel[yshift = .5\columnsep]{lightDepEnh}
	\subfiglabel[yshift = .5\columnsep]{GOterms}
	\subfiglabel[yshift = .5\columnsep]{lightDepTATAGC}
	\subfiglabel[yshift = .5\columnsep]{lightDepTCP}
	\subfiglabel[yshift = .5\columnsep]{lightDepWRKY}


\end{tikzpicture}%
			\caption{%
				\textbf{Promoter strength can be modulated by light.}\titleend
				\subfigref{A} Tobacco leaves were transiently transformed with STARR-seq promoter libraries and the plants were kept for two days in 16h light/8h dark cycles (light) or completely in the dark (dark) prior to mRNA extraction.\nextentry
				\subfigref{B} Violin plots (as defined in \autoref{fig:overview}) of light-dependency (promoter strength\textsuperscript{light} $-$ promoter strength\textsuperscript{dark}) for promoters in the libraries with ($+$) or without ($-$) the 35S enhancer.\nextentry
				\subfigref{C} Enrichment of selected GO terms for genes associated with the 1000 most light-dependent promoters. The red line marks the significance threshold (adjusted $p\ \textsf{value} \leq 0.05$). Non-significant bars are gray.\nextentry
				\subfigrange{D}{F} Violin plots of light-dependency. Promoters are grouped by GC content and split into promoters without (left half, darker color) or with (right half, lighter color) a TATA-box \parensubfig{D}, or a binding site for TCP \parensubfig{E} or WRKY \parensubfig{F} transcription factors.
			}%
			\label{fig:lightDep}
		\end{fig}
		
		\begin{fig}
			%\tikzset{external/export next = false}

\begin{tikzpicture}

	\begin{pgfinterruptboundingbox} % somehow these plots add about 1.2pt of whitespace to their right ... no idea why

		\setlength{\templength}{\threecolumnwidth}
		\addtolength{\templength}{-.9cm}
	
		%%% synthetic promoter design
		\coordinate (SynProDesign) at (0, 0);
		
		\node[anchor = north, shift = {(.5\threecolumnwidth, .25\columnsep)}, text width = .9\threecolumnwidth, align = center] (SynProText) at (SynProDesign) {synthetic promoters with\\nucleotide frequencies derived from:};
		
		\node[anchor = west, shift = {(0, -.5cm - .1\columnsep)}, align = center] (lAt) at (SynProText.south -| SynProDesign) {Arabidopsis\\promoters:};
		
		\coordinate[xshift = {.5cm + .25\columnsep}] (base freq At) at (lAt.east);
		
		\fill[baseAcol!66] (base freq At) -- ++(0, .5) arc (90:90-127:.5) coordinate (A) -- cycle;
		\fill[baseCcol!66] (base freq At) -- (A) arc (90-127:90-127-60:.5) coordinate (C) -- cycle;
		\fill[baseGcol!66] (base freq At) -- (C) arc (90-127-60:90-127-60-55:.5) coordinate (G) -- cycle;
		\fill[baseTcol!66] (base freq At) -- (G) arc (90-127-60-55:90-360:.5) coordinate (T) -- cycle;
		
		\node at ($(base freq At) + (90-127/2:.33)$) {A};
		\node at ($(base freq At) + (90-127-60/2:.33)$) {C};
		\node at ($(base freq At) + (90-127-60-55/2:.33)$) {G};
		\node at ($(base freq At) + (90+118/2:.33)$) {T};
		
		\node[anchor = base east, shift = {(-1cm -.25\columnsep - 0.3em, 0)}, align = center] (lZm) at (lAt.base -| \threecolumnwidth, 0) {\vphantom{A}maize\\promoters:};
		
		\coordinate[xshift = {.5cm + .25\columnsep}] (base freq Zm) at (lZm.east);
		
		\fill[baseAcol!66] (base freq Zm) -- ++(0, .5) arc (90:90-88:.5) coordinate (A) -- cycle;
		\fill[baseCcol!66] (base freq Zm) -- (A) arc (90-88:90-88-104:.5) coordinate (C) -- cycle;
		\fill[baseGcol!66] (base freq Zm) -- (C) arc (90-88-104:90-88-104-81:.5) coordinate (G) -- cycle;
		\fill[baseTcol!66] (base freq Zm) -- (G) arc (90-88-104-81:90-360:.5) coordinate (T) -- cycle;
		
		\node at ($(base freq Zm) + (90-88/2:.33)$) {A};
		\node at ($(base freq Zm) + (90-88-104/2:.33)$) {C};
		\node at ($(base freq Zm) + (90-88-104-81/2:.33)$) {G};
		\node at ($(base freq Zm) + (90+86/2:.33)$) {T};
		
		\node[xshift = {.5cm + .25\columnsep}] at ($(lAt.east)!.5!(lZm.west)$) {or};
		
		\coordinate[shift = {(.1cm, -.5cm - \columnsep)}] (combis) at (lAt.west);
		
		\draw[very thick, gray] (combis) ++(.7, 0) coordinate (combis1) -- ++(\templength, 0);
		\draw[very thick, gray] (combis) ++(.6, -.2) coordinate (combis2) -- ++(\templength, 0);
		\draw[very thick, gray] (combis) ++(.5, -.4) coordinate (combis3) -- ++(\templength, 0);
		\draw[very thick, gray] (combis) ++(.4, -.6) coordinate (combis4) -- ++(\templength, 0);
		\draw[very thick, gray] (combis) ++(.3, -.8) coordinate (combis5) -- ++(\templength, 0);
		\draw[very thick, gray] (combis) ++(.2, -1) coordinate (combis6) -- ++(\templength, 0);
		\draw[very thick, gray] (combis) ++(.1, -1.2) coordinate (combis7) -- ++(\templength, 0);
		\draw[very thick, gray] (combis) ++(0, -1.4) coordinate (combis8) -- ++(\templength, 0);
		
	%	\node[anchor = east , align = right, xshift = .5cm] at ($(combis1 -| base freq Zm)!.5!(combis8 -| base freq Zm)$) {core\\promoter\\element\\combinations};
		
		\draw[line width = .1cm, MediumPurple2] (combis2) ++(\templength, 0) -- ++(-.06\templength, 0);
		\draw[line width = .1cm, MediumPurple2] (combis5) ++(\templength, 0) -- ++(-.06\templength, 0);
		\draw[line width = .1cm, MediumPurple2] (combis6) ++(\templength, 0) -- ++(-.06\templength, 0);
		\draw[line width = .1cm, MediumPurple2] (combis8) ++(\templength, 0) -- ++(-.06\templength, 0) coordinate[pos = .5] (Inr);
		
		\draw[line width = .1cm, Goldenrod2] (combis3) ++(.885\templength, 0) -- ++(.05\templength, 0);
		\draw[line width = .1cm, Goldenrod2] (combis5) ++(.885\templength, 0) -- ++(.05\templength, 0);
		\draw[line width = .1cm, Goldenrod2] (combis7) ++(.885\templength, 0) -- ++(.05\templength, 0);
		\draw[line width = .1cm, Goldenrod2] (combis8) ++(.885\templength, 0) -- ++(.05\templength, 0) coordinate[pos = .5] (Ypatch);
		
		\draw[line width = .1cm, 35S enhancer] (combis4) ++(.78\templength, 0) -- ++(.05\templength, 0);
		\draw[line width = .1cm, 35S enhancer] (combis6) ++(.78\templength, 0) -- ++(.05\templength, 0);
		\draw[line width = .1cm, 35S enhancer] (combis7) ++(.78\templength, 0) -- ++(.05\templength, 0);
		\draw[line width = .1cm, 35S enhancer] (combis8) ++(.78\templength, 0) -- ++(.05\templength, 0) coordinate[pos = .5] (TATA);
		
		\setlength{\basewidth}{.25\threecolumnwidth/8}
		
		\coordinate[yshift = -.25cm] (motifTATA) at (combis8 -| .167\threecolumnwidth, 0);
		\coordinate[yshift = -.25cm, xshift = -\basewidth] (motifYpatch) at (combis8 -| .5\threecolumnwidth, 0);
		\coordinate[yshift = -.25cm, xshift = -\basewidth] (motifInr) at (combis8 -| .833\threecolumnwidth, 0);
		
		\foreach \motif/\col/\bases/\name in {Inr/MediumPurple2/10/Inr, Ypatch/Goldenrod2/8/Y patch, TATA/35S enhancer/8/TATA-box} {
			\logoplot[
				at = {(motif\motif)},
				yshift = -\columnsep,
				anchor = above north,
				width = \bases*\basewidth,
				height = 0.5cm,
				logo axis = none,
				title = \name,
				title style = {draw = none, fill = \col!25, minimum width = \bases*\basewidth},
				axis background/.style = {fill = \col!25},
				name = logo\motif
			]{rawData/motif_synPRO_\motif.tsv}
					
			\begin{pgfonlayer}{background}
				\draw[thin, gray] (\motif) ++(0, -.1cm) -- (logo\motif.above north);
			
				\draw[very thick, \col] (logo\motif.above north west) rectangle (logo\motif.south east);
			\end{pgfonlayer}
		}
	
		
		%%% promoter strength synthetic promoters (tobacco)
		\coordinate (enrSynProLeaf) at (SynProDesign -| \twothirdcolumnwidth - \threecolumnwidth, 0);
		
		\leafsymbol{enrSynProLeaf}
		
		\begin{hgroupplot}[%
			ylabel = {promoter strength},
			ymin = -5,
			ymax = 1,
			ytick = {-10, -9, ..., 10},
			xticklabel style = {name = xticklabel},
			zero line = 35S enhancer,
			group position = {anchor = above north west, at = {(enrSynProLeaf)}, xshift = \plotylabelwidth},
			group/every plot/.append style = {
				x grids = false,
				typeset ticklabels with strut 
			}
		]{\threecolumnwidth}{2}{}
	
				
			\nextgroupplot[
				title = {Arabidosis\\promoters},
				x tick table = {rawData/enrichment_syntheticPRO_At_leaf_boxplot.tsv}{Inr}
			]
			
				% boxplot
				\boxplots{%
					box color = arabidopsis,
					box shade,
					fill opacity = 0.5%
				}{rawData/enrichment_syntheticPRO_At_leaf_boxplot.tsv}{rawData/enrichment_syntheticPRO_At_leaf_boxplot_outliers.tsv}
				
				% save coordinates
				\pgfplotstablegetrowsof{rawData/enrichment_syntheticPRO_At_leaf_boxplot.tsv}
				\pgfplotsinvokeforeach{1, ..., \pgfplotsretval}{
					\coordinate (A#1) at (#1, 0);
				}
			
			
			\nextgroupplot[
				title = {\vphantom{A}maize\\promoters},
				x tick table = {rawData/enrichment_syntheticPRO_Zm_leaf_boxplot.tsv}{Inr}
			]
			
				% boxplot
				\boxplots{%
					box color = maize,
					box shade,
					fill opacity = 0.5%
				}{rawData/enrichment_syntheticPRO_Zm_leaf_boxplot.tsv}{rawData/enrichment_syntheticPRO_Zm_leaf_boxplot_outliers.tsv}
				
				% save coordinates
				\pgfplotstablegetrowsof{rawData/enrichment_syntheticPRO_Zm_leaf_boxplot.tsv}
				\pgfplotsinvokeforeach{1, ..., \pgfplotsretval}{
					\coordinate (Z#1) at (#1, 0);
				}
				
		\end{hgroupplot}
		
		\node[anchor = base east, node font = \figsmall] (lInr) at (group c1r1.west |- xticklabel.base) {\strut Inr};
		\node[anchor = north east, node font = \figsmall] (lYpatch) at (lInr.base east) {\strut Y patch};
		\node[anchor = north east, node font = \figsmall] (lTATA) at (lYpatch.base east) {\strut TATA-box};
		
		\pgfplotstableforeachcolumnelement{Ypatch}\of{rawData/enrichment_syntheticPRO_At_leaf_boxplot.tsv}\as\xticklabel{
			\pgfmathsetmacro{\i}{\pgfplotstablerow + 1}
			\node[anchor = base, node font = \figsmall] at (lYpatch.base -| A\i) {\strut\xticklabel};
		}
		
		\pgfplotstableforeachcolumnelement{Ypatch}\of{rawData/enrichment_syntheticPRO_Zm_leaf_boxplot.tsv}\as\xticklabel{
			\pgfmathsetmacro{\i}{\pgfplotstablerow + 1}
			\node[anchor = base, node font = \figsmall] at (lYpatch.base -| Z\i) {\strut\xticklabel};
		}
		
		\pgfplotstableforeachcolumnelement{TATA}\of{rawData/enrichment_syntheticPRO_At_leaf_boxplot.tsv}\as\xticklabel{
			\pgfmathsetmacro{\i}{\pgfplotstablerow + 1}
			\node[anchor = base, node font = \figsmall] at (lTATA.base -| A\i) {\strut\xticklabel};
		}
		
		\pgfplotstableforeachcolumnelement{TATA}\of{rawData/enrichment_syntheticPRO_Zm_leaf_boxplot.tsv}\as\xticklabel{
			\pgfmathsetmacro{\i}{\pgfplotstablerow + 1}
			\node[anchor = base, node font = \figsmall] at (lTATA.base -| Z\i) {\strut\xticklabel};
		}
		
	
		%%% promoter strength synthetic promoters (maize)
		\coordinate (enrSynProProto) at (SynProDesign -| \textwidth - \threecolumnwidth, 0);
		
		\protosymbol{enrSynProProto}
		
		\begin{hgroupplot}[%
			ylabel = {promoter strength},
			ymin = -5,
			ymax = 1,
			ytick = {-10, -9, ..., 10},
			xticklabel style = {name = xticklabel},
			zero line = 35S enhancer,
			group position = {anchor = above north west, at = {(enrSynProProto)}, xshift = \plotylabelwidth},
			group/every plot/.append style = {
				x grids = false,
				typeset ticklabels with strut 
			}
		]{\threecolumnwidth}{2}{}
	
				
			\nextgroupplot[
				title = {Arabidosis\\promoters},
				x tick table = {rawData/enrichment_syntheticPRO_At_proto_boxplot.tsv}{Inr}
			]
			
				% boxplot
				\boxplots{%
					box color = arabidopsis,
					box shade,
					fill opacity = 0.5%
				}{rawData/enrichment_syntheticPRO_At_proto_boxplot.tsv}{rawData/enrichment_syntheticPRO_At_proto_boxplot_outliers.tsv}
				
				% save coordinates
				\pgfplotstablegetrowsof{rawData/enrichment_syntheticPRO_At_proto_boxplot.tsv}
				\pgfplotsinvokeforeach{1, ..., \pgfplotsretval}{
					\coordinate (A#1) at (#1, 0);
				}
			
			
			\nextgroupplot[
				title = {\vphantom{A}maize\\promoters},
				x tick table = {rawData/enrichment_syntheticPRO_Zm_proto_boxplot.tsv}{Inr}
			]
			
				% boxplot
				\boxplots{%
					box color = maize,
					box shade,
					fill opacity = 0.5%
				}{rawData/enrichment_syntheticPRO_Zm_proto_boxplot.tsv}{rawData/enrichment_syntheticPRO_Zm_proto_boxplot_outliers.tsv}
				
				% save coordinates
				\pgfplotstablegetrowsof{rawData/enrichment_syntheticPRO_Zm_proto_boxplot.tsv}
				\pgfplotsinvokeforeach{1, ..., \pgfplotsretval}{
					\coordinate (Z#1) at (#1, 0);
				}
				
		\end{hgroupplot}
		
		\node[anchor = base east, node font = \figsmall] (lInr) at (group c1r1.west |- xticklabel.base) {\strut Inr};
		\node[anchor = north east, node font = \figsmall] (lYpatch) at (lInr.base east) {\strut Y patch};
		\node[anchor = north east, node font = \figsmall] (lTATA) at (lYpatch.base east) {\strut TATA-box};
		
		\pgfplotstableforeachcolumnelement{Ypatch}\of{rawData/enrichment_syntheticPRO_At_proto_boxplot.tsv}\as\xticklabel{
			\pgfmathsetmacro{\i}{\pgfplotstablerow + 1}
			\node[anchor = base, node font = \figsmall] at (lYpatch.base -| A\i) {\strut\xticklabel};
		}
		
		\pgfplotstableforeachcolumnelement{Ypatch}\of{rawData/enrichment_syntheticPRO_Zm_proto_boxplot.tsv}\as\xticklabel{
			\pgfmathsetmacro{\i}{\pgfplotstablerow + 1}
			\node[anchor = base, node font = \figsmall] at (lYpatch.base -| Z\i) {\strut\xticklabel};
		}
		
		\pgfplotstableforeachcolumnelement{TATA}\of{rawData/enrichment_syntheticPRO_At_proto_boxplot.tsv}\as\xticklabel{
			\pgfmathsetmacro{\i}{\pgfplotstablerow + 1}
			\node[anchor = base, node font = \figsmall] at (lTATA.base -| A\i) {\strut\xticklabel};
		}
		
		\pgfplotstableforeachcolumnelement{TATA}\of{rawData/enrichment_syntheticPRO_Zm_proto_boxplot.tsv}\as\xticklabel{
			\pgfmathsetmacro{\i}{\pgfplotstablerow + 1}
			\node[anchor = base, node font = \figsmall] at (lTATA.base -| Z\i) {\strut\xticklabel};
		}
		
		
		%%% TF combination test (design)
		\coordinate[yshift = -\columnsep] (TFcomboDesign) at (SynProDesign |- lTATA.south);
		
		\draw[very thick, gray] (TFcomboDesign) ++(.1, -\columnsep) coordinate (TFcomboDesign1) -- ++(\templength, 0);
		
		\draw[line width = .1cm, 35S enhancer] ($(TFcomboDesign1) + (.78\templength, 0)$) -- ++(.05\templength, 0) node[pos = .5, anchor = south, node font = \figsmall] {TATA};
		
		\foreach \x in {35, 65, 95} {
			\fill[SlateBlue4] ($(TFcomboDesign1) + (\x/170*\templength, -.025)$) -- ++(-.075, -.15) -- ++(.15, 0) node[pos = .5, anchor = north, node font = \figsmall, alias = last pos] (pos\x) {\x} -- cycle;
		}
		
		\node[anchor = north, shift = {(.5\threecolumnwidth, -.75cm)}, align = center, text width = .67\threecolumnwidth] (text) at (TFcomboDesign |- last pos.south) {for each postion, no transcription factor binding site or one of:};
		
		\coordinate[yshift = -.5\columnsep] (TFlogos) at (TFcomboDesign |- text.south);
		
		\setlength{\basewidth}{\threecolumnwidth/27}
		
		\logoplot[
			at = {(TFlogos)},
			anchor = above north west,
			width = 10\basewidth,
			xshift = \basewidth,
			height = 0.5cm,
			logo axis = none,
			title = TCP (cluster 15),
			title style = {draw = none, fill = none},
			name = logoTCP15
		]{rawData/motif_TCP15.tsv}
		
		\logoplot[
			at = {(logoTCP15.south west)},
			yshift = -\columnsep,
			anchor = above north west,
			width = 8\basewidth,
			height = 0.5cm,
			logo axis = none,
			title = TCP (cluster 22),
			title style = {draw = none, fill = none, align = left},
			name = logoTCP22
		]{rawData/motif_TCP22.tsv}
		
		\logoplot[
			at = {(logoTCP15.above north -| \threecolumnwidth, 0)},
			anchor = above north east,
			width = 13\basewidth,
			height = 0.5cm,
			logo axis = none,
			title = NAC (cluster 1),
			title style = {draw = none, fill = none},
			name = logoNAC
		]{rawData/motif_NAC.tsv}
		
		\logoplot[
			at = {(logoTCP22.above north -| logoNAC.east)},
			anchor = above north east,
			width = 16\basewidth,
			height = 0.5cm,
			logo axis = none,
			title = HSF/S1Fa-like (cluster 16),
			title style = {draw = none, fill = none},
			name = logoHSF
		]{rawData/motif_HSF.tsv}
		
		\draw[thin, SlateBlue4!50] (pos35.base west) -- (text.north -| logoTCP15.west) (pos95.base east) -- (text.north -| logoNAC.east);
		
		
		%%% single TF
		\coordinate (singleTF) at (TFcomboDesign -| \twothirdcolumnwidth - \threecolumnwidth, 0);
		
		\begin{hgroupplot}[%
			ylabel = {rel. promoter strength},
			ymin = -2.75,
			ymax = 4,
			ytick = {-10, -9, ..., 10},
			zero line,
			group position = {anchor = above north west, at = {(singleTF)}, xshift = \plotylabelwidth},
			group/every plot/.append style = {
				x grids = false,
				typeset ticklabels with strut 
			}
		]{\threecolumnwidth}{2}{TF binding site}
	
				
			\nextgroupplot[
				title = {\phantom{tobacco}\\\phantom{leaves}},
				x tick table = {rawData/enrichment_TFcombo_single_leaf_boxplot.tsv}{sample}
			]
			
				% boxplot
				\boxplots{%
					box color = leafCol,
					box shade,
					fill opacity = 0.5%
				}{rawData/enrichment_TFcombo_single_leaf_boxplot.tsv}{rawData/enrichment_TFcombo_single_leaf_boxplot_outliers.tsv}
				
				% add sample size
				\samplesize{rawData/enrichment_TFcombo_single_leaf_boxplot.tsv}{id}{n}
				
				% add significance
				\signifallsimple*{rawData/enrichment_TFcombo_single_leaf_boxplot.tsv}{id}{p.value}
			
			
			\nextgroupplot[
				title = {\phantom{maize}\\\phantom{protoplasts}},
				x tick table = {rawData/enrichment_TFcombo_single_proto_boxplot.tsv}{sample}
			]
			
				% boxplot
				\boxplots{%
					box color = protoCol,
					box shade,
					fill opacity = 0.5%
				}{rawData/enrichment_TFcombo_single_proto_boxplot.tsv}{rawData/enrichment_TFcombo_single_proto_boxplot_outliers.tsv}
				
				% add sample size
				\samplesize{rawData/enrichment_TFcombo_single_proto_boxplot.tsv}{id}{n}
				
				% add significance
				\signifallsimple*{rawData/enrichment_TFcombo_single_proto_boxplot.tsv}{id}{p.value}
				
		\end{hgroupplot}
		
		\distance{title.south}{title.north}
		
		\leafsymbol[(.5\ydistance, -.5\ydistance)]{group c1r1.west |- title.north}
		\node[anchor = east, node font = \fignormal, text depth = .15\baselineskip, align = right] at (group c1r1.east |- title) {tobacco\\leaves};
		\protosymbol[(-.5\ydistance, -.5\ydistance)]{group c2r1.east |- title.north}
		\node[anchor = west, node font = \fignormal, text depth = .15\baselineskip, align = left] at (group c2r1.west |- title) {maize\\protoplasts};
		
		
		%%% TF number
		\coordinate (TFnumber) at (TFcomboDesign -| \textwidth - \threecolumnwidth, 0);
		
		\begin{hgroupplot}[%
			ylabel = {rel. promoter strength},
			ymin = -3,
			ymax = 6.75,
			ytick = {-10, -9, ..., 10},
			zero line,
			group position = {anchor = above north west, at = {(TFnumber)}, xshift = \plotylabelwidth},
			group/every plot/.append style = {
				x grids = false,
				typeset ticklabels with strut 
			}
		]{\threecolumnwidth}{2}{number of TF binding sites}
	
				
			\nextgroupplot[
				title = {\phantom{tobacco}\\\phantom{leaves}},
				x tick table = {rawData/enrichment_TFcombo_number_leaf_boxplot.tsv}{sample}
			]
			
				% boxplot
				\boxplots{%
					box color = leafCol,
					box shade,
					fill opacity = 0.5%
				}{rawData/enrichment_TFcombo_number_leaf_boxplot.tsv}{rawData/enrichment_TFcombo_number_leaf_boxplot_outliers.tsv}
				
				% add sample size
				\samplesize{rawData/enrichment_TFcombo_number_leaf_boxplot.tsv}{id}{n}
				
				% add significance
				\signifallsimple*{rawData/enrichment_TFcombo_number_leaf_boxplot.tsv}{id}{p.value}
			
			
			\nextgroupplot[
				title = {\phantom{maize}\\\phantom{protoplasts}},
				x tick table = {rawData/enrichment_TFcombo_number_proto_boxplot.tsv}{sample}
			]
			
				% boxplot
				\boxplots{%
					box color = protoCol,
					box shade,
					fill opacity = 0.5%
				}{rawData/enrichment_TFcombo_number_proto_boxplot.tsv}{rawData/enrichment_TFcombo_number_proto_boxplot_outliers.tsv}
				
				% add sample size
				\samplesize{rawData/enrichment_TFcombo_number_proto_boxplot.tsv}{id}{n}
				
				% add significance
				\signifallsimple*{rawData/enrichment_TFcombo_number_proto_boxplot.tsv}{id}{p.value}
				
		\end{hgroupplot}
		
		\distance{title.south}{title.north}
		
		\leafsymbol[(.5\ydistance, -.5\ydistance)]{group c1r1.west |- title.north}
		\node[anchor = east, node font = \fignormal, text depth = .15\baselineskip, align = right] at (group c1r1.east |- title) {tobacco\\leaves};
		\protosymbol[(-.5\ydistance, -.5\ydistance)]{group c2r1.east |- title.north}
		\node[anchor = west, node font = \fignormal, text depth = .15\baselineskip, align = left] at (group c2r1.west |- title) {maize\\protoplasts};
		
	
		%%% TCP positions
		\coordinate[yshift = -\columnsep] (TCPscanPos) at (SynProDesign |- plot xlabel.south);
		
		\setlength{\templength}{\twocolumnwidth - \plotylabelwidth}
		
		\draw[very thick, gray] (TCPscanPos) ++(\plotylabelwidth, -.5\columnsep) coordinate (TF pos1) -- ++(\templength, 0) coordinate (TF pos4end);
		
		\draw[line width = .1cm, 35S enhancer] ($(TF pos1) + (.78\templength, 0)$) -- ++(.05\templength, 0) node[anchor = south, pos = .5, node font = \figsmall] {TATA};
		
		\pgfplotstableforeachcolumnelement{sample}\of{rawData/enrichment_TFscan_TCP_leaf_boxplot.tsv}\as\position{
			\fill[SlateBlue4] ($(TF pos1) + (\position/170*\templength, -.025)$) -- ++(-.075, -.15) -- ++(.15, 0) node[pos = .5, anchor = east, rotate = 90, node font = \figsmall, alias = last pos] (pos\position) {\position} -- cycle;
		}
		
		\node[anchor = east, SlateBlue4, align = right, node font = \figsmall, xshift = \plotylabelwidth] at (TCPscanPos |- last pos) {TCP\\[-.25\baselineskip]position};
	
	
		%%% TCP position scan (tobacco)
		\coordinate[yshift = -.5\columnsep] (TCPscan) at (TCPscanPos |- last pos.west);
		
		\leafsymbol{TCPscan};
		
		\begin{axis}[%
			anchor = north west,
			at = {(TCPscan)},
			xshift = \plotylabelwidth,
			width = \twocolumnwidth - \plotylabelwidth,
			xlabel = TCP binding site position,
			ylabel = rel. promoter strength,
			ylabel style = {xshift = -\baselineskip},
			xmin = 0,
			xmax = 170,
			xtick = {10, 30, ..., 160},
			ymin = -3,
			ymax = 4,
			ytick = {-10, -9, ..., 10},
			zero line,
			x grids = false
		]
		
			% TATA-box
			\fill[35S enhancer, fill opacity = .1] (132.5, \ymin) rectangle (140.5, \ymax) node[pos = .5, opacity = 1, node font = \figsmall] {TATA};
		
			% boxplots
			\pgfplotstableforeachcolumnelement{sample}\of{rawData/enrichment_TFscan_TCP_leaf_boxplot.tsv}\as\sample{
				\ifthenelse{\pgfplotstablerow<0}{}{
					\pgfmathparse{50 + (50 * \sample / 170)}
					\edef\colpct{\pgfmathresult}
					\pgfmathparse{int(\pgfplotstablerow + 1)}
					\edef\thisboxplot{
						\noexpand\boxplot[%
							boxplot/draw position = \sample,
							box color = leafCol,
							fill opacity = .5,
							fill = boxcol!\colpct!boxshade,
							boxplot/box extend = 4%
						]{rawData/enrichment_TFscan_TCP_leaf_boxplot.tsv}{\pgfmathresult};
						\noexpand\addplot [black, only marks, mark = solido] table [x expr = \sample, y = outlier.\pgfmathresult] {rawData/enrichment_TFscan_TCP_leaf_boxplot_outliers.tsv};
					}
					\thisboxplot
				}
			}
			
	%		% add sample size
	%		\samplesize{rawData/enrichment_TFscan_TCP_leaf_boxplot.tsv}{sample}{n}
	%
	%		% add significance
	%		\signifallsimple*{rawData/enrichment_TFscan_TCP_leaf_boxplot.tsv}{sample}{p.value}
				
		\end{axis}
		
		
		%%% HSF positions
		\coordinate (HSFscanPos) at (TCPscanPos -| \textwidth - \twocolumnwidth, 0);
		
		\setlength{\templength}{\twocolumnwidth - \plotylabelwidth}
		
		\draw[very thick, gray] (HSFscanPos) ++(\plotylabelwidth, -.5\columnsep) coordinate (TF pos1) -- ++(\templength, 0) coordinate (TF pos4end);
		
		\draw[line width = .1cm, 35S enhancer] ($(TF pos1) + (.78\templength, 0)$) -- ++(.05\templength, 0) node[anchor = south, pos = .5, node font = \figsmall] {TATA};
		
		\pgfplotstableforeachcolumnelement{sample}\of{rawData/enrichment_TFscan_HSF_proto_boxplot.tsv}\as\position{
			\fill[SlateBlue4] ($(TF pos1) + (\position/170*\templength, -.025)$) -- ++(-.075, -.15) -- ++(.15, 0) node[pos = .5, anchor = east, rotate = 90, node font = \figsmall, alias = last pos] (pos\position) {\position} -- cycle;
		}
		
		\node[anchor = east, SlateBlue4, align = right, node font = \figsmall, xshift = \plotylabelwidth] at (HSFscanPos |- last pos) {HSF\\[-.25\baselineskip]position};
		
		
		%%% HSF position scan (protoplasts)
		\coordinate (HSFscan) at (HSFscanPos |- TCPscan);
		
		\protosymbol{HSFscan};
		
		\begin{axis}[%
			anchor = north west,
			at = {(HSFscan)},
			xshift = \plotylabelwidth,
			width = \twocolumnwidth - \plotylabelwidth,
			xlabel = HSF binding site position (rel. to TSS),
			xlabel style = {name = plot xlabel},
			ylabel = rel. promoter strength,
			ylabel style = {xshift = -\baselineskip},
			xmin = 0,
			xmax = 170,
			xtick = {10, 30, ..., 160},
			ymin = -3,
			ymax = 4,
			ytick = {-10, -9, ..., 10},
			zero line,
			x grids = false
		]
		
			% TATA-box
			\fill[35S enhancer, fill opacity = .1] (132.5, \ymin) rectangle (140.5, \ymax) node[pos = .5, opacity = 1, node font = \figsmall] {TATA};
		
			% boxplots
			\pgfplotstableforeachcolumnelement{sample}\of{rawData/enrichment_TFscan_HSF_proto_boxplot.tsv}\as\sample{
				\ifthenelse{\pgfplotstablerow<0}{}{
					\pgfmathparse{50 + (50 * \sample / 170)}
					\edef\colpct{\pgfmathresult}
					\pgfmathparse{int(\pgfplotstablerow + 1)}
					\edef\thisboxplot{
						\noexpand\boxplot[boxplot/draw position = \sample, box color = protoCol, fill opacity = .5, fill = boxcol!\colpct!boxshade, boxplot/box extend = 4]{rawData/enrichment_TFscan_HSF_proto_boxplot.tsv}{\pgfmathresult};
						\noexpand\addplot [black, only marks, mark = solido] table [x expr = \sample, y = outlier.\pgfmathresult] {rawData/enrichment_TFscan_HSF_proto_boxplot_outliers.tsv};
					}
					\thisboxplot
				}
			}
			
	%		% add sample size
	%		\samplesize{rawData/enrichment_TFscan_HSF_proto_boxplot.tsv}{sample}{n}
	%
	%		% add significance
	%		\signifallsimple*{rawData/enrichment_TFscan_HSF_proto_boxplot.tsv}{sample}{p.value}
				
		\end{axis}
		
		
		%%% subfigure labels
		\subfiglabel[yshift = .5\columnsep]{SynProDesign}
		\subfiglabel[yshift = .5\columnsep]{enrSynProLeaf}
		\subfiglabel[yshift = .5\columnsep]{enrSynProProto}
		\subfiglabel[yshift = .5\columnsep]{TFcomboDesign}
		\subfiglabel[yshift = .5\columnsep]{singleTF}
		\subfiglabel[yshift = .5\columnsep]{TFnumber}
		\subfiglabel[yshift = .5\columnsep]{TCPscanPos}
		\subfiglabel[yshift = .5\columnsep]{HSFscanPos}
	
	\end{pgfinterruptboundingbox}
	
	% manually add the correct bounding box
	\path (SynProDesign) ++(.2pt, .5\columnsep) rectangle (\textwidth + .2pt, 0 |- plot xlabel.south);
	
\end{tikzpicture}%
			\caption{
				\textbf{Design and validation of synthetic promoters.}\titleend
				\subfigrange{A}{C} Synthetic promoters with nucleotide frequencies similar to an average Arabidopsis (35.2\% A, 16.6\% C, 15.3\% G, 32.8\% T) or maize (24.5\% A, 29.0\% C, 22.5\% G, 23.9\% T) promoter were created and modified by adding a TATA-box, Y patch, and/or Inr element \parensubfig{A}. Promoter strength was determined by STARR-seq in tobacco leaves \parensubfig{B} and maize protoplasts \parensubfig{C}. Promoters with an Arabidopsis-like nucleotide composition are shown on the left, those with maize-like base frequencies on the right. The strength of the 35S minimal promoter is indicated by a horizontal blue line.\nextentry
				\subfigrange{D}{F} Transcription factor binding sites for TCP, NAC, and HSF transcription factors were inserted at positions 35, 65, and/or 95 of the synthetic promoters with a TATA-box \parensubfig{D} and the activity of promoters with a single binding site for the indicated transcription factor \parensubfig{E} or multiple binding sites \parensubfig{F} was determined in tobacco leaves (left panel) or maize protoplasts (right panel).
				\subfigref{G}\subfigref{H} A single TCP \parensubfig{G} or HSF \parensubfig{H} transcription factor binding site was inserted at the indicated position in the synthetic promoters containing a TATA-box. The strength of these promoters was measured in tobacco leaves \parensubfig{G} or maize protoplasts \parensubfig{H}. Boxplots are as defined in \autoref{fig:TATA}. In (\textbf{\plainsubfigref{E}}-\textbf{\plainsubfigref{H}}), the corresponding promoter without any transcription factor binding was set to 0 (horizontal black line).
			}%
			\label{fig:synthetic}
		\end{fig}
		
		\begin{fig}
			\tikzset{png export}%
			%\tikzset{external/export next = false}

\begin{tikzpicture}
		
	%%% linear model correlation plot
	\coordinate (LMcor) at (0, 0);
	
	\begin{hgroupplot}[%
		ylabel = predicted promoter strength,
		xymin = -5.5,
		xymax = 6.5,
		xytick = {-10, -8, ..., 10},
		xlabel style = {align = center, name = xlabel},
		group position = {anchor = above north west, at = {(LMcor)}, xshift = \plotylabelwidth},
		scatter/classes = {
			At={arabidopsis},
			Zm={maize},
			Sb={sorghum},
			all={black}
		}	,
		legend pos = south east,
		legend columns = 3,
		legend image post style = {mark size = 1, fill opacity = 1}
	]{\twocolumnwidth}{2}{}

		\nextgroupplot[
			title = tobacco model,
			xlabel = measured promoter strength\\in tobacco leaves
		]
		
			\addplot [
				scatter,
				scatter src = explicit symbolic,
				only marks,
				mark = solido,
				fill opacity = 0.2
			] table[x = enrichment, y = prediction, meta = sample.name] {rawData/linear-model_leaf_pred.tsv};
			
			\addplot [
				scatter,
				scatter src = explicit symbolic,
				only marks,
				mark = text,
				text mark as node = true,
				text mark style = {align = left, node font = \figsmall, anchor = north west, yshift = -\baselineskip * \coordindex},
				text mark = {\rsquare},
				visualization depends on = value \thisrow{rsquare}\as\rsquare
			] table [x expr = \xmin, y expr = \ymax, meta = sample.name] {rawData/linear-model_leaf_stats.tsv};
			
			
		\nextgroupplot[
			title = maize model,
			xlabel = measured promoter strength\\in maize protoplasts
		]
		
			\addplot [
				scatter,
				scatter src = explicit symbolic,
				only marks,
				mark = solido,
				fill opacity = 0.2
			] table[x = enrichment, y = prediction, meta = sample.name] {rawData/linear-model_proto_pred.tsv};
			
			\addplot [
				scatter,
				scatter src = explicit symbolic,
				only marks,
				mark = text,
				text mark as node = true,
				text mark style = {align = left, node font = \figsmall, anchor = north west, yshift = -\baselineskip * \coordindex},
				text mark = {\rsquare},
				visualization depends on = value \thisrow{rsquare}\as\rsquare
			] table [x expr = \xmin, y expr = \ymax, meta = sample.name] {rawData/linear-model_proto_stats.tsv};
			
			\legend{At\phantom{A}, Zm\phantom{A}, Sb}

	\end{hgroupplot}

	\leafsymbol[(-.3, .3)]{group c1r1.south east};
	\protosymbol[(.3, .3)]{group c2r1.south west};
	

	%%% CNN correlation plot
	\coordinate (CNNcor) at (LMcor -| \textwidth - \twocolumnwidth, 0);
	
	\begin{hgroupplot}[%
		ylabel = predicted promoter strength,
		xymin = -5.5,
		xymax = 6.5,
		xytick = {-10, -8, ..., 10},
		xlabel style = {align = center, name = xlabel},
		group position = {anchor = above north west, at = {(CNNcor)}, xshift = \plotylabelwidth},
		scatter/classes = {
			At={arabidopsis},
			Zm={maize},
			Sb={sorghum},
			all={black}
		}	,
		legend pos = south east,
		legend columns = 3,
		legend image post style = {mark size = 1, fill opacity = 1}
	]{\twocolumnwidth}{2}{}

		\nextgroupplot[
			title = tobacco model,
			xlabel = measured promoter strength\\in tobacco leaves
		]
		
			\addplot [
				scatter,
				scatter src = explicit symbolic,
				only marks,
				mark = solido,
				fill opacity = 0.2
			] table[x = enrichment, y = prediction, meta = sample.name] {rawData/CNN_test_leaf_pred.tsv};
			
			\addplot [
				scatter,
				scatter src = explicit symbolic,
				only marks,
				mark = text,
				text mark as node = true,
				text mark style = {align = left, node font = \figsmall, anchor = north west, yshift = -\baselineskip * \coordindex},
				text mark = {$R^2=\rsquare$},
				visualization depends on = value \thisrow{rsquare}\as\rsquare
			] table [x expr = \xmin, y expr = \ymax, meta = sample.name] {rawData/CNN_test_leaf_stats.tsv};
			
			
		\nextgroupplot[
			title = maize model,
			xlabel = measured promoter strength\\in maize protoplasts
		]
		
			\addplot [
				scatter,
				scatter src = explicit symbolic,
				only marks,
				mark = solido,
				fill opacity = 0.2
			] table[x = enrichment, y = prediction, meta = sample.name] {rawData/CNN_test_proto_pred.tsv};
			
			\addplot [
				scatter,
				scatter src = explicit symbolic,
				only marks,
				mark = text,
				text mark as node = true,
				text mark style = {align = left, node font = \figsmall, anchor = north west, yshift = -\baselineskip * \coordindex},
				text mark = {$R^2=\rsquare$},
				visualization depends on = value \thisrow{rsquare}\as\rsquare
			] table [x expr = \xmin, y expr = \ymax, meta = sample.name] {rawData/CNN_test_proto_stats.tsv};
			
			\legend{At\phantom{A}, Zm\phantom{A}, Sb}

	\end{hgroupplot}

	\leafsymbol[(-.3, .3)]{group c1r1.south east};
	\protosymbol[(.3, .3)]{group c2r1.south west};
	
	
	%%% promoter strength of evolved promoters with enhancer (leaf)
	\coordinate[yshift = -\columnsep] (enrEvoLeaf) at (LMcor |- xlabel.south);
	
	\leafsymbol{enrEvoLeaf};
	
	\begin{hgroupplot}[%
		ylabel = promoter strength,
		ymin = -3,
		ymax = 7.75,
		ytick = {-10, -8, ..., 10},
		execute at begin axis = {\draw[very thin, 35S enhancer] (\xmin, 4.55) -- (\xmax, 4.55);},
		group position = {anchor = above north west, at = {(enrEvoLeaf)}, xshift = \plotylabelwidth, yshift = -1.25\baselineskip},
		group/every plot/.append style = {x grids = false},
	]{\twocolumnwidth}{3}{rounds of \textit{in silico} evolution}

		\nextgroupplot[
			title = tobacco model,
			x tick table = {rawData/enrichment_evolution_leaf_leaf_boxplot.tsv}{sample}
		]
			
		% violin and box plot
		\violinbox[violin color = leafCol]{rawData/enrichment_evolution_leaf_leaf_boxplot.tsv}{rawData/enrichment_evolution_leaf_leaf_violin.tsv}
			
		% add sample size
		\samplesize{rawData/enrichment_evolution_leaf_leaf_boxplot.tsv}{id}{n}
		
		% add significance level
		\signif*{rawData/enrichment_evolution_leaf_leaf_pvalues.tsv}{1}{2}
		\signif*{rawData/enrichment_evolution_leaf_leaf_pvalues.tsv}{2}{3}
		
		
		\nextgroupplot[
			title = maize model,
			x tick table = {rawData/enrichment_evolution_proto_leaf_boxplot.tsv}{sample}
		]
			
		% violin and box plot
		\violinbox[violin color = protoCol]{rawData/enrichment_evolution_proto_leaf_boxplot.tsv}{rawData/enrichment_evolution_proto_leaf_violin.tsv}
			
		% add sample size
		\samplesize{rawData/enrichment_evolution_proto_leaf_boxplot.tsv}{id}{n}
		
		% add significance level
		\signif*{rawData/enrichment_evolution_proto_leaf_pvalues.tsv}{1}{2}
		\signif*{rawData/enrichment_evolution_proto_leaf_pvalues.tsv}{2}{3}
		
		
		\nextgroupplot[
			title = both models,
			x tick table = {rawData/enrichment_evolution_both_leaf_boxplot.tsv}{sample}
		]
			
		% violin and box plot
		\violinbox[violin color = leafCol!50!protoCol]{rawData/enrichment_evolution_both_leaf_boxplot.tsv}{rawData/enrichment_evolution_both_leaf_violin.tsv}
			
		% add sample size
		\samplesize{rawData/enrichment_evolution_both_leaf_boxplot.tsv}{id}{n}
		
		% add significance level
		\signif*{rawData/enrichment_evolution_both_leaf_pvalues.tsv}{1}{2}
		\signif*{rawData/enrichment_evolution_both_leaf_pvalues.tsv}{2}{3}
			
	\end{hgroupplot}
	
	\node[anchor = south west] (legend) at (group c1r1.above north west) {\textbf{with} 35S enhancer};
	
	
	%%% promoter strength of evolved promoters with enhancer (proto)
	\begin{pgfinterruptboundingbox} % somehow this plot adds about 8pt of whitespace to its right ... no idea why
	
		\coordinate (enrEvoProto) at (enrEvoLeaf -| CNNcor);
		
		\protosymbol{enrEvoProto};
		
		\begin{hgroupplot}[%
			ylabel = promoter strength,
			ymin = -3.5,
			ymax = 5.25,
			ytick = {-10, -8, ..., 10},
			execute at begin axis = {\draw[very thin, 35S enhancer] (\xmin, 2.37) -- (\xmax, 2.37);},
			group position = {anchor = above north west, at = {(enrEvoProto)}, xshift = \plotylabelwidth, yshift = -1.25\baselineskip},
			group/every plot/.append style = {x grids = false},
		]{\twocolumnwidth}{3}{rounds of \textit{in silico} evolution}
	
			\nextgroupplot[
				title = tobacco model,
				x tick table = {rawData/enrichment_evolution_leaf_proto_boxplot.tsv}{sample}
			]
				
			% violin and box plot
			\violinbox[violin color = leafCol]{rawData/enrichment_evolution_leaf_proto_boxplot.tsv}{rawData/enrichment_evolution_leaf_proto_violin.tsv}
				
			% add sample size
			\samplesize{rawData/enrichment_evolution_leaf_proto_boxplot.tsv}{id}{n}
			
			% add significance level
			\signif*{rawData/enrichment_evolution_leaf_proto_pvalues.tsv}{1}{2}
			\signif*{rawData/enrichment_evolution_leaf_proto_pvalues.tsv}{2}{3}
			
			
			\nextgroupplot[
				title = maize model,
				x tick table = {rawData/enrichment_evolution_proto_proto_boxplot.tsv}{sample}
			]
				
			% violin and box plot
			\violinbox[violin color = protoCol]{rawData/enrichment_evolution_proto_proto_boxplot.tsv}{rawData/enrichment_evolution_proto_proto_violin.tsv}
				
			% add sample size
			\samplesize{rawData/enrichment_evolution_proto_proto_boxplot.tsv}{id}{n}
			
			% add significance level
			\signif*{rawData/enrichment_evolution_proto_proto_pvalues.tsv}{1}{2}
			\signif*{rawData/enrichment_evolution_proto_proto_pvalues.tsv}{2}{3}
			
			
			\nextgroupplot[
				title = both models,
				x tick table = {rawData/enrichment_evolution_both_proto_boxplot.tsv}{sample}
			]
				
			% violin and box plot
			\violinbox[violin color = leafCol!50!protoCol]{rawData/enrichment_evolution_both_proto_boxplot.tsv}{rawData/enrichment_evolution_both_proto_violin.tsv}
				
			% add sample size
			\samplesize{rawData/enrichment_evolution_both_proto_boxplot.tsv}{id}{n}
			
			% add significance level
			\signif*{rawData/enrichment_evolution_both_proto_pvalues.tsv}{1}{2}
			\signif*{rawData/enrichment_evolution_both_proto_pvalues.tsv}{2}{3}
		
		\end{hgroupplot}
		
		\node[anchor = south west] (legend) at (group c1r1.above north west) {\textbf{with} 35S enhancer};
		
	\end{pgfinterruptboundingbox}
	
	
	%%% promoter strength of evolved promoters without enhancer (leaf)
	\coordinate[yshift = -\columnsep] (enrEvoLeafnoE) at (LMcor |- plot xlabel.south);
	
	\leafsymbol{enrEvoLeafnoE};
	
	\begin{hgroupplot}[%
		ylabel = promoter strength,
		ymin = -6,
		ymax = 4.5,
		ytick = {-10, -8, ..., 10},
		zero line = 35S enhancer,
		group position = {anchor = above north west, at = {(enrEvoLeafnoE)}, xshift = \plotylabelwidth, yshift = -1.25\baselineskip},
		group/every plot/.append style = {x grids = false},
	]{\twocolumnwidth}{3}{rounds of \textit{in silico} evolution}

		\nextgroupplot[
			title = tobacco model,
			x tick table = {rawData/enrichment_evolution_noEnh_leaf_leaf_boxplot.tsv}{sample}
		]
			
		% violin and box plot
		\violinbox[violin color = leafCol]{rawData/enrichment_evolution_noEnh_leaf_leaf_boxplot.tsv}{rawData/enrichment_evolution_noEnh_leaf_leaf_violin.tsv}
			
		% add sample size
		\samplesize{rawData/enrichment_evolution_noEnh_leaf_leaf_boxplot.tsv}{id}{n}
		
		% add significance level
		\signif*{rawData/enrichment_evolution_noEnh_leaf_leaf_pvalues.tsv}{1}{2}
		\signif*{rawData/enrichment_evolution_noEnh_leaf_leaf_pvalues.tsv}{2}{3}
		
		
		\nextgroupplot[
			title = maize model,
			x tick table = {rawData/enrichment_evolution_noEnh_proto_leaf_boxplot.tsv}{sample}
		]
			
		% violin and box plot
		\violinbox[violin color = protoCol]{rawData/enrichment_evolution_noEnh_proto_leaf_boxplot.tsv}{rawData/enrichment_evolution_noEnh_proto_leaf_violin.tsv}
			
		% add sample size
		\samplesize{rawData/enrichment_evolution_noEnh_proto_leaf_boxplot.tsv}{id}{n}
		
		% add significance level
		\signif*{rawData/enrichment_evolution_noEnh_proto_leaf_pvalues.tsv}{1}{2}
		\signif*{rawData/enrichment_evolution_noEnh_proto_leaf_pvalues.tsv}{2}{3}
		
		
		\nextgroupplot[
			title = both models,
			x tick table = {rawData/enrichment_evolution_noEnh_both_leaf_boxplot.tsv}{sample}
		]
			
		% violin and box plot
		\violinbox[violin color = leafCol!50!protoCol]{rawData/enrichment_evolution_noEnh_both_leaf_boxplot.tsv}{rawData/enrichment_evolution_noEnh_both_leaf_violin.tsv}
			
		% add sample size
		\samplesize{rawData/enrichment_evolution_noEnh_both_leaf_boxplot.tsv}{id}{n}
		
		% add significance level
		\signif*{rawData/enrichment_evolution_noEnh_both_leaf_pvalues.tsv}{1}{2}
		\signif*{rawData/enrichment_evolution_noEnh_both_leaf_pvalues.tsv}{2}{3}
			
	\end{hgroupplot}
		
	\node[anchor = south west] (legend) at (group c1r1.above north west) {\textbf{without} 35S enhancer};
	
	
	%%% promoter strength of evolved promoters without enhancer (proto)
	\begin{pgfinterruptboundingbox} % somehow this plot adds about 8pt of whitespace to its right ... no idea why
	
		\coordinate (enrEvoProtonoE) at (enrEvoLeafnoE -| CNNcor);
		
		\protosymbol{enrEvoProtonoE};
		
		\begin{hgroupplot}[%
			ylabel = promoter strength,
			ymin = -6.5,
			ymax = 3,
			ytick = {-10, -8, ..., 10},
			zero line = 35S enhancer,
			group position = {anchor = above north west, at = {(enrEvoProtonoE)}, xshift = \plotylabelwidth, yshift = -1.25\baselineskip},
			group/every plot/.append style = {x grids = false},
		]{\twocolumnwidth}{3}{rounds of \textit{in silico} evolution}
	
			\nextgroupplot[
				title = tobacco model,
				x tick table = {rawData/enrichment_evolution_noEnh_leaf_proto_boxplot.tsv}{sample}
			]
				
			% violin and box plot
			\violinbox[violin color = leafCol]{rawData/enrichment_evolution_noEnh_leaf_proto_boxplot.tsv}{rawData/enrichment_evolution_noEnh_leaf_proto_violin.tsv}
				
			% add sample size
			\samplesize{rawData/enrichment_evolution_noEnh_leaf_proto_boxplot.tsv}{id}{n}
			
			% add significance level
			\signif*{rawData/enrichment_evolution_noEnh_leaf_proto_pvalues.tsv}{1}{2}
			\signif*{rawData/enrichment_evolution_noEnh_leaf_proto_pvalues.tsv}{2}{3}
			
			
			\nextgroupplot[
				title = maize model,
				x tick table = {rawData/enrichment_evolution_noEnh_proto_proto_boxplot.tsv}{sample}
			]
				
			% violin and box plot
			\violinbox[violin color = protoCol]{rawData/enrichment_evolution_noEnh_proto_proto_boxplot.tsv}{rawData/enrichment_evolution_noEnh_proto_proto_violin.tsv}
				
			% add sample size
			\samplesize{rawData/enrichment_evolution_noEnh_proto_proto_boxplot.tsv}{id}{n}
			
			% add significance level
			\signif*{rawData/enrichment_evolution_noEnh_proto_proto_pvalues.tsv}{1}{2}
			\signif*{rawData/enrichment_evolution_noEnh_proto_proto_pvalues.tsv}{2}{3}
			
			
			\nextgroupplot[
				title = both models,
				x tick table = {rawData/enrichment_evolution_noEnh_both_proto_boxplot.tsv}{sample}
			]
				
			% violin and box plot
			\violinbox[violin color = leafCol!50!protoCol]{rawData/enrichment_evolution_noEnh_both_proto_boxplot.tsv}{rawData/enrichment_evolution_noEnh_both_proto_violin.tsv}
				
			% add sample size
			\samplesize{rawData/enrichment_evolution_noEnh_both_proto_boxplot.tsv}{id}{n}
			
			% add significance level
			\signif*{rawData/enrichment_evolution_noEnh_both_proto_pvalues.tsv}{1}{2}
			\signif*{rawData/enrichment_evolution_noEnh_both_proto_pvalues.tsv}{2}{3}
		
		\end{hgroupplot}
		
		\node[anchor = south west] (legend) at (group c1r1.above north west) {\textbf{without} 35S enhancer};
		
	\end{pgfinterruptboundingbox}
	
	
	%%% subfigure labels
	\subfiglabel[yshift = .5\columnsep]{LMcor}
	\subfiglabel[yshift = .5\columnsep]{CNNcor}
	\subfiglabel[yshift = .5\columnsep]{enrEvoLeaf}
	\subfiglabel[yshift = .5\columnsep]{enrEvoProto}
	\subfiglabel[yshift = .5\columnsep]{enrEvoLeafnoE}
	\subfiglabel[yshift = .5\columnsep]{enrEvoProtonoE}

\end{tikzpicture}%
			\caption{%
				\textbf{Computational models can predict promoter strength and enable \textit{in silico} evolution of plant promoters.}\titleend
				\subfigref{A} Correlation between the promoter strength as determined by STARR-seq using promoter libraries with the 35S enhancer and predictions from a linear model based on the GC content and motif scores for core promoter elements and transcription factors. The models were trained on data from the tobacco leaf system (tobacco model) or the maize protoplasts (maize model). The overall correlation is indicated in black and correlations for each species are colored as indicated (inset). Correlations are shown for a test set of 10\% of all promoters.\nextentry
				\subfigref{B} Similar to \parensubfig{A} but the prediction is based on a convolutional neural network trained on promoter sequences.\nextentry
				\subfigrange{C}{F} Violin plots (as defined in \autoref{fig:overview}) of promoter strength of the unmodified promoters (0 rounds of evolution) or after they were subjected to three or ten rounds of  \textit{in silico} evolution as determined in tobacco leaves \parensubfig[\subfigref{C}]{E} or maize protoplasts \parensubfig[\subfigref{D}]{F}. The promoters were tested in a library with \parensubfig[\subfigref{C}]{D} or without \parensubfig[\subfigref{E}]{F} an upstream 35S enhancer. The model(s) used for the \textit{in silico} evolution is indicated on each plot. The promoter strength of the 35S promoter is indicated by a horizontal blue line.
			}%
			\label{fig:models}
		\end{fig}
	
	\fi
	
	%%% Main end
	
	
	%%% Supp start
	
	\ifsupp
	
		\renewcommand{\figurename}{Supplementary Fig.}
		\setcounter{figure}{0}
		
		\renewcommand{\tablename}{Supplementary Table}
		\setcounter{table}{0}
	
		\begin{sfig}
			\tikzset{png export}%
			%\tikzset{external/export next = false}

\begin{tikzpicture}
		
	%%% correlation plot (arabidopsis library in tobacco)
	\coordinate (corLeafAt) at (0, 0);
	
	\begin{axis}[
		width = \threecolumnwidth - \plotylabelwidth,
		height = \threecolumnwidth - \plotylabelwidth,
		at = {(corLeafAt)},
		anchor = north west,
		xshift = \plotylabelwidth,
		xymin = -7.95,
		xymax = 6.75,
		xytick = {-10, -8, ..., 10},
		xlabel = {promoter strength (rep 1)},
		ylabel = {promoter strength (rep 2)},
		xlabel style = {name = plot xlabel},
		scatter/classes = {
			noENH={black},
			withENH={35S enhancer}
		}
	]
	
		\addplot [
			scatter,
			scatter src = explicit symbolic,
			only marks,
			mark = solido,
			mark size = 0.25,
			fill opacity = 0.1
		] table[x = rep1,y = rep2, meta = sample.name] {rawData/enrichment_correlation_At_dark.tsv};
		
		\addplot [
			scatter,
			scatter src = explicit symbolic,
			only marks,
			mark = text,
			text mark as node = true,
			text mark style = {align = left, node font = \figsmall, anchor = north west, yshift = -2\baselineskip * \coordindex},
			text mark = {\spearman\\[-.25\baselineskip]\rsquare},
			visualization depends on = value \thisrow{spearman}\as\spearman,
			visualization depends on = value \thisrow{rsquare}\as\rsquare
		] table [x expr = \xmin, y expr = \ymax, meta = sample.name] {rawData/enrichment_correlation_At_dark_stats.tsv};
		
		\node[anchor = south east, thin, draw, fill = white, align = center, node font = \figsmall] at (rel axis cs: 0.97, 0.03) {enhancer\\\tikz\path[fill = black] (-.3em -1pt, 0) (0, 0) circle (1pt) node[anchor = west] {$-$\vphantom{A}}; ~ \tikz\path[fill = DodgerBlue1] (0, 0) circle (1pt) node[anchor = west] {$+$\vphantom{A}};};
		
	\end{axis}
	
	\leafsymbol[(-.3, -.3)]{last plot.south west};
	

	%%% correlation plot (arabidopsis library in protoplasts)
	\coordinate (corProtoAt) at (corLeafAt -| \twothirdcolumnwidth - \threecolumnwidth, 0);
	
	\begin{axis}[
		width = \threecolumnwidth - \plotylabelwidth,
		height = \threecolumnwidth - \plotylabelwidth,
		at = {(corProtoAt)},
		anchor = north west,
		xshift = \plotylabelwidth,
		xymin = -7.95,
		xymax = 6.75,
		xytick = {-10, -8, ..., 10},
		xlabel = {promoter strength (rep 1)},
		ylabel = {promoter strength (rep 2)},
		scatter/classes = {
			noENH={black},
			withENH={35S enhancer}
		}
	]
	
		\addplot [
			scatter,
			scatter src = explicit symbolic,
			only marks,
			mark = solido,
			mark size = 0.25,
			fill opacity = 0.1
		] table[x = rep1,y = rep2, meta = sample.name] {rawData/enrichment_correlation_At_proto.tsv};
		
		\addplot [
			scatter,
			scatter src = explicit symbolic,
			only marks,
			mark = text,
			text mark as node = true,
			text mark style = {align = left, node font = \figsmall, anchor = north west, yshift = -2\baselineskip * \coordindex},
			text mark = {\spearman\\[-.25\baselineskip]\rsquare},
			visualization depends on = value \thisrow{spearman}\as\spearman,
			visualization depends on = value \thisrow{rsquare}\as\rsquare
		] table [x expr = \xmin, y expr = \ymax, meta = sample.name] {rawData/enrichment_correlation_At_proto_stats.tsv};
		
		\node[anchor = south east, thin, draw, fill = white, align = center, node font = \figsmall] at (rel axis cs: 0.97, 0.03) {enhancer\\\tikz\path[fill = black] (-.3em -1pt, 0) (0, 0) circle (1pt) node[anchor = west] {$-$\vphantom{A}}; ~ \tikz\path[fill = DodgerBlue1] (0, 0) circle (1pt) node[anchor = west] {$+$\vphantom{A}};};
		
	\end{axis}
	
	\protosymbol[(-.3, -.3)]{last plot.south west};
	
	
	%%% comparison tobacco leaves and maize protoplasts (arabidopsis library)
	\coordinate (leafVprotoAt) at (corLeafAt -| \textwidth - \threecolumnwidth, 0);
	
	\begin{axis}[
		width = \threecolumnwidth - \plotylabelwidth,
		height = \threecolumnwidth - \plotylabelwidth,
		at = {(leafVprotoAt)},
		anchor = north west,
		xshift = \plotylabelwidth,
		xymin = -7.95,
		xymax = 6.75,
		xytick = {-10, -8, ..., 10},
		ylabel = {tobacco leaves},
		xlabel = {maize protoplasts},
		xlabel style = {align = center},
		ylabel style = {align = center},
		scatter/classes = {
			noENH={black},
			withENH={35S enhancer}
		}
	]
	
		\addplot [
			scatter,
			scatter src = explicit symbolic,
			only marks,
			mark = solido,
			mark size = 0.25,
			fill opacity = 0.1
		] table[x = proto,y = leaf, meta = sample.name] {rawData/enrichment_leaf-vs-proto_At.tsv};
		
		\addplot [
			scatter,
			scatter src = explicit symbolic,
			only marks,
			mark = text,
			text mark as node = true,
			text mark style = {align = left, node font = \figsmall, anchor = north west, yshift = -2\baselineskip * \coordindex},
			text mark = {\spearman\\[-.25\baselineskip]\rsquare},
			visualization depends on = value \thisrow{spearman}\as\spearman,
			visualization depends on = value \thisrow{rsquare}\as\rsquare
		] table [x expr = \xmin, y expr = \ymax, meta = sample.name] {rawData/enrichment_leaf-vs-proto_At_stats.tsv};
		
		\node[anchor = south east, thin, draw, fill = white, align = center, node font = \figsmall] at (rel axis cs: 0.97, 0.03) {enhancer\\\tikz\path[fill = black] (-.3em -1pt, 0) (0, 0) circle (1pt) node[anchor = west] {$-$\vphantom{A}}; ~ \tikz\path[fill = DodgerBlue1] (0, 0) circle (1pt) node[anchor = west] {$+$\vphantom{A}};};
		
	\end{axis}
	
	\leafsymbol[(-.4, 0)]{last plot.south west};
	\protosymbol[(0, -.4)]{last plot.south west};
	\node[node font = \figsmall, shift = {(-.4, -.4)}] at (last plot.south west) {vs.};
	
	
	%%% correlation plot (sorghum library in tobacco)
	\coordinate[yshift = -\columnsep] (corLeafSb) at (plot xlabel.south -| corLeafAt);
	
	\begin{axis}[
		width = \threecolumnwidth - \plotylabelwidth,
		height = \threecolumnwidth - \plotylabelwidth,
		at = {(corLeafSb)},
		anchor = north west,
		xshift = \plotylabelwidth,
		xymin = -7.95,
		xymax = 6.75,
		xytick = {-10, -8, ..., 10},
		xlabel = {promoter strength (rep 1)},
		ylabel = {promoter strength (rep 2)},
		xlabel style = {name = plot xlabel},
		scatter/classes = {
			noENH={black},
			withENH={35S enhancer}
		}
	]
	
		\addplot [
			scatter,
			scatter src = explicit symbolic,
			only marks,
			mark = solido,
			mark size = 0.25,
			fill opacity = 0.1
		] table[x = rep1,y = rep2, meta = sample.name] {rawData/enrichment_correlation_Sb_dark.tsv};
		
		\addplot [
			scatter,
			scatter src = explicit symbolic,
			only marks,
			mark = text,
			text mark as node = true,
			text mark style = {align = left, node font = \figsmall, anchor = north west, yshift = -2\baselineskip * \coordindex},
			text mark = {\spearman\\[-.25\baselineskip]\rsquare},
			visualization depends on = value \thisrow{spearman}\as\spearman,
			visualization depends on = value \thisrow{rsquare}\as\rsquare
		] table [x expr = \xmin, y expr = \ymax, meta = sample.name] {rawData/enrichment_correlation_Sb_dark_stats.tsv};
		
		\node[anchor = south east, thin, draw, fill = white, align = center, node font = \figsmall] at (rel axis cs: 0.97, 0.03) {enhancer\\\tikz\path[fill = black] (-.3em -1pt, 0) (0, 0) circle (1pt) node[anchor = west] {$-$\vphantom{A}}; ~ \tikz\path[fill = DodgerBlue1] (0, 0) circle (1pt) node[anchor = west] {$+$\vphantom{A}};};
		
	\end{axis}
	
	\leafsymbol[(-.3, -.3)]{last plot.south west};
	

	%%% correlation plot (sorghum library in protoplasts)
	\coordinate (corProtoSb) at (corLeafSb -| corProtoAt);
	
	\begin{axis}[
		width = \threecolumnwidth - \plotylabelwidth,
		height = \threecolumnwidth - \plotylabelwidth,
		at = {(corProtoSb)},
		anchor = north west,
		xshift = \plotylabelwidth,
		xymin = -7.95,
		xymax = 6.75,
		xytick = {-10, -8, ..., 10},
		xlabel = {promoter strength (rep 1)},
		ylabel = {promoter strength (rep 2)},
		scatter/classes = {
			noENH={black},
			withENH={35S enhancer}
		}
	]
	
		\addplot [
			scatter,
			scatter src = explicit symbolic,
			only marks,
			mark = solido,
			mark size = 0.25,
			fill opacity = 0.1
		] table[x = rep1,y = rep2, meta = sample.name] {rawData/enrichment_correlation_Sb_proto.tsv};
		
		\addplot [
			scatter,
			scatter src = explicit symbolic,
			only marks,
			mark = text,
			text mark as node = true,
			text mark style = {align = left, node font = \figsmall, anchor = north west, yshift = -2\baselineskip * \coordindex},
			text mark = {\spearman\\[-.25\baselineskip]\rsquare},
			visualization depends on = value \thisrow{spearman}\as\spearman,
			visualization depends on = value \thisrow{rsquare}\as\rsquare
		] table [x expr = \xmin, y expr = \ymax, meta = sample.name] {rawData/enrichment_correlation_Sb_proto_stats.tsv};
		
		\node[anchor = south east, thin, draw, fill = white, align = center, node font = \figsmall] at (rel axis cs: 0.97, 0.03) {enhancer\\\tikz\path[fill = black] (-.3em -1pt, 0) (0, 0) circle (1pt) node[anchor = west] {$-$\vphantom{A}}; ~ \tikz\path[fill = DodgerBlue1] (0, 0) circle (1pt) node[anchor = west] {$+$\vphantom{A}};};
		
	\end{axis}
	
	\protosymbol[(-.3, -.3)]{last plot.south west};
	
	
	%%% comparison tobacco leaves and maize protoplasts (sorghum library)
	\coordinate (leafVprotoSb) at (corLeafSb -| leafVprotoAt);
	
	\begin{axis}[
		width = \threecolumnwidth - \plotylabelwidth,
		height = \threecolumnwidth - \plotylabelwidth,
		at = {(leafVprotoSb)},
		anchor = north west,
		xshift = \plotylabelwidth,
		xymin = -7.95,
		xymax = 6.75,
		xytick = {-10, -8, ..., 10},
		ylabel = {tobacco leaves},
		xlabel = {maize protoplasts},
		xlabel style = {align = center},
		ylabel style = {align = center},
		scatter/classes = {
			noENH={black},
			withENH={35S enhancer}
		}
	]
	
		\addplot [
			scatter,
			scatter src = explicit symbolic,
			only marks,
			mark = solido,
			mark size = 0.25,
			fill opacity = 0.1
		] table[x = proto,y = leaf, meta = sample.name] {rawData/enrichment_leaf-vs-proto_Sb.tsv};
		
		\addplot [
			scatter,
			scatter src = explicit symbolic,
			only marks,
			mark = text,
			text mark as node = true,
			text mark style = {align = left, node font = \figsmall, anchor = north west, yshift = -2\baselineskip * \coordindex},
			text mark = {\spearman\\[-.25\baselineskip]\rsquare},
			visualization depends on = value \thisrow{spearman}\as\spearman,
			visualization depends on = value \thisrow{rsquare}\as\rsquare
		] table [x expr = \xmin, y expr = \ymax, meta = sample.name] {rawData/enrichment_leaf-vs-proto_Sb_stats.tsv};
		
		\node[anchor = south east, thin, draw, fill = white, align = center, node font = \figsmall] at (rel axis cs: 0.97, 0.03) {enhancer\\\tikz\path[fill = black] (-.3em -1pt, 0) (0, 0) circle (1pt) node[anchor = west] {$-$\vphantom{A}}; ~ \tikz\path[fill = DodgerBlue1] (0, 0) circle (1pt) node[anchor = west] {$+$\vphantom{A}};};
		
	\end{axis}
	
	\leafsymbol[(-.4, 0)]{last plot.south west};
	\protosymbol[(0, -.4)]{last plot.south west};
	\node[node font = \figsmall, shift = {(-.4, -.4)}] at (last plot.south west) {vs.};
	
	
	%%% correlation plot (arabidopsis library in tobacco, light condition)
	\coordinate[yshift = -\columnsep] (corLightAt) at (corLeafAt |- plot xlabel.south);
	
	\begin{axis}[
		width = \threecolumnwidth - \plotylabelwidth,
		height = \threecolumnwidth - \plotylabelwidth,
		at = {(corLightAt)},
		anchor = north west,
		xshift = \plotylabelwidth,
		xymin = -7.95,
		xymax = 6.75,
		xytick = {-10, -8, ..., 10},
		xlabel = {promoter strength (rep 1)},
		ylabel = {promoter strength (rep 2)},
		xlabel style = {name = plot xlabel},
		scatter/classes = {
			noENH={black},
			withENH={35S enhancer}
		}
	]
	
		\addplot [
			scatter,
			scatter src = explicit symbolic,
			only marks,
			mark = solido,
			mark size = 0.25,
			fill opacity = 0.1
		] table[x = rep1,y = rep2, meta = sample.name] {rawData/enrichment_correlation_At_light.tsv};
		
		\addplot [
			scatter,
			scatter src = explicit symbolic,
			only marks,
			mark = text,
			text mark as node = true,
			text mark style = {align = left, node font = \figsmall, anchor = north west, yshift = -2\baselineskip * \coordindex},
			text mark = {\spearman\\[-.25\baselineskip]\rsquare},
			visualization depends on = value \thisrow{spearman}\as\spearman,
			visualization depends on = value \thisrow{rsquare}\as\rsquare
		] table [x expr = \xmin, y expr = \ymax, meta = sample.name] {rawData/enrichment_correlation_At_light_stats.tsv};
		
		\node[anchor = south east, thin, draw, fill = white, align = center, node font = \figsmall] at (rel axis cs: 0.97, 0.03) {enhancer\\\tikz\path[fill = black] (-.3em -1pt, 0) (0, 0) circle (1pt) node[anchor = west] {$-$\vphantom{A}}; ~ \tikz\path[fill = DodgerBlue1] (0, 0) circle (1pt) node[anchor = west] {$+$\vphantom{A}};};
		
	\end{axis}
	
	\lightsymbol[(-.3, -.3)]{last plot.south west};
	

	%%% correlation plot (maize library in tobacco, light condition)
	\coordinate (corLightZm) at (corLightAt -| corProtoAt);
	
	\begin{axis}[
		width = \threecolumnwidth - \plotylabelwidth,
		height = \threecolumnwidth - \plotylabelwidth,
		at = {(corLightZm)},
		anchor = north west,
		xshift = \plotylabelwidth,
		xymin = -7.95,
		xymax = 6.75,
		xytick = {-10, -8, ..., 10},
		xlabel = {promoter strength (rep 1)},
		ylabel = {promoter strength (rep 2)},
		scatter/classes = {
			noENH={black},
			withENH={35S enhancer}
		}
	]
	
		\addplot [
			scatter,
			scatter src = explicit symbolic,
			only marks,
			mark = solido,
			mark size = 0.25,
			fill opacity = 0.1
		] table[x = rep1,y = rep2, meta = sample.name] {rawData/enrichment_correlation_Zm_light.tsv};
		
		\addplot [
			scatter,
			scatter src = explicit symbolic,
			only marks,
			mark = text,
			text mark as node = true,
			text mark style = {align = left, node font = \figsmall, anchor = north west, yshift = -2\baselineskip * \coordindex},
			text mark = {\spearman\\[-.25\baselineskip]\rsquare},
			visualization depends on = value \thisrow{spearman}\as\spearman,
			visualization depends on = value \thisrow{rsquare}\as\rsquare
		] table [x expr = \xmin, y expr = \ymax, meta = sample.name] {rawData/enrichment_correlation_Zm_light_stats.tsv};
		
		\node[anchor = south east, thin, draw, fill = white, align = center, node font = \figsmall] at (rel axis cs: 0.97, 0.03) {enhancer\\\tikz\path[fill = black] (-.3em -1pt, 0) (0, 0) circle (1pt) node[anchor = west] {$-$\vphantom{A}}; ~ \tikz\path[fill = DodgerBlue1] (0, 0) circle (1pt) node[anchor = west] {$+$\vphantom{A}};};
		
	\end{axis}
	
	\lightsymbol[(-.3, -.3)]{last plot.south west};
	
	
	%%% correlation plot (sorghum library in tobacco, light condition)
	\coordinate (corLightSb) at (corLightAt -| leafVprotoAt);
	
	\begin{axis}[
		width = \threecolumnwidth - \plotylabelwidth,
		height = \threecolumnwidth - \plotylabelwidth,
		at = {(corLightSb)},
		anchor = north west,
		xshift = \plotylabelwidth,
		xymin = -7.95,
		xymax = 6.75,
		xytick = {-10, -8, ..., 10},
		xlabel = {promoter strength (rep 1)},
		ylabel = {promoter strength (rep 2)},
		scatter/classes = {
			noENH={black},
			withENH={35S enhancer}
		}
	]
	
		\addplot [
			scatter,
			scatter src = explicit symbolic,
			only marks,
			mark = solido,
			mark size = 0.25,
			fill opacity = 0.1
		] table[x = rep1,y = rep2, meta = sample.name] {rawData/enrichment_correlation_Sb_light.tsv};
		
		\addplot [
			scatter,
			scatter src = explicit symbolic,
			only marks,
			mark = text,
			text mark as node = true,
			text mark style = {align = left, node font = \figsmall, anchor = north west, yshift = -2\baselineskip * \coordindex},
			text mark = {\spearman\\[-.25\baselineskip]\rsquare},
			visualization depends on = value \thisrow{spearman}\as\spearman,
			visualization depends on = value \thisrow{rsquare}\as\rsquare
		] table [x expr = \xmin, y expr = \ymax, meta = sample.name] {rawData/enrichment_correlation_Sb_light_stats.tsv};
		
		\node[anchor = south east, thin, draw, fill = white, align = center, node font = \figsmall] at (rel axis cs: 0.97, 0.03) {enhancer\\\tikz\path[fill = black] (-.3em -1pt, 0) (0, 0) circle (1pt) node[anchor = west] {$-$\vphantom{A}}; ~ \tikz\path[fill = DodgerBlue1] (0, 0) circle (1pt) node[anchor = west] {$+$\vphantom{A}};};
		
	\end{axis}
	
	\lightsymbol[(-.3, -.3)]{last plot.south west};
	
	
	%%% subfigure labels
	\subfiglabel[yshift = .5\columnsep]{corLeafAt}
	\subfiglabel[yshift = .5\columnsep]{corProtoAt}
	\subfiglabel[yshift = .5\columnsep]{leafVprotoAt}
	\subfiglabel[yshift = .5\columnsep]{corLeafSb}
	\subfiglabel[yshift = .5\columnsep]{corProtoSb}
	\subfiglabel[yshift = .5\columnsep]{leafVprotoSb}
	\subfiglabel[yshift = .5\columnsep]{corLightAt}
	\subfiglabel[yshift = .5\columnsep]{corLightZm}
	\subfiglabel[yshift = .5\columnsep]{corLightSb}

\end{tikzpicture}%
			\caption{%
				\textbf{The promoter STARR-seq assay is highly reproducible but promoter strength depends on the assay system.}\titleend
				\subfigref{A}\subfigref{B} Correlation of two biological replicates of STARR-seq using the Arabidopsis promoter libraries in tobacco leaves \parensubfig{A} or in maize protoplasts \parensubfig{B}.\nextentry
				\subfigref{C} Comparison of the strength of Arabidopsis promoters in tobacco leaves and maize protoplasts.\nextentry
				\subfigref{D}\subfigref{E} Correlation of two biological replicates of STARR-seq using the sorghum promoter libraries in tobacco leaves \parensubfig{D} or in maize protoplasts \parensubfig{E}.\nextentry
				\subfigref{F} Comparison of the strength of sorghum promoters in tobacco leaves and maize protoplasts.\nextentry
				\subfigrange{G}{I}  Correlation of two biological replicates of STARR-seq using the Arabidopsis \parensubfig{G}, maize \parensubfig{H}, or sorghum \parensubfig{I} promoter libraries in tobacco leaves that were kept for two days in 16h light/8h dark cycles prior to mRNA extraction.
			}%
			\label{sfig:correlation}%
		\end{sfig}
		
		\begin{sfig}
			%\tikzset{external/export next = false}

\begin{tikzpicture}
		
	%%% correlation plot (first validation set in tobacco, dark)
	\coordinate (PROvalLeaf) at (0, 0);
	
	\begin{axis}[
		width = \threecolumnwidth - \plotylabelwidth,
		height = \threecolumnwidth - \plotylabelwidth,
		at = {(PROvalLeaf)},
		anchor = north west,
		xshift = \plotylabelwidth,
		xymin = -6.25,
		xymax = 6.25,
		xytick = {-10, -8, ..., 10},
		xlabel = {promoter strength (validation set 1)},
		ylabel = {promoter strength (main data)},
		xlabel style = {name = plot xlabel},
		scatter/classes = {
			noENH={black},
			withENH={35S enhancer}
		}
	]
	
		\addplot [
			scatter,
			scatter src = explicit symbolic,
			only marks,
			mark = solido,
			mark size = 0.5
		] table[x = enrichment.val, y = enrichment.pro, meta = sample.name] {rawData/enrichment_correlation_PROval_dark.tsv};
		
		\addplot [
			scatter,
			scatter src = explicit symbolic,
			only marks,
			mark = text,
			text mark as node = true,
			text mark style = {align = left, node font = \figsmall, anchor = north west, yshift = -2\baselineskip * \coordindex},
			text mark = {\spearman\\[-.25\baselineskip]\rsquare},
			visualization depends on = value \thisrow{spearman}\as\spearman,
			visualization depends on = value \thisrow{rsquare}\as\rsquare
		] table [x expr = \xmin, y expr = \ymax, meta = sample.name] {rawData/enrichment_correlation_PROval_dark_stats.tsv};
		
		\node[anchor = south east, thin, draw, fill = white, align = center, node font = \figsmall] at (rel axis cs: 0.97, 0.03) {enhancer\\\tikz\path[fill = black] (-.3em -1pt, 0) (0, 0) circle (1pt) node[anchor = west] {$-$\vphantom{A}}; ~ \tikz\path[fill = DodgerBlue1] (0, 0) circle (1pt) node[anchor = west] {$+$\vphantom{A}};};
		
	\end{axis}
	
	\leafsymbol[(-.4, -.4)]{last plot.south west};
	

	%%% correlation plot (first validation set in protoplasts, dark)
	\coordinate (PROvalProto) at (PROvalLeaf -| \twothirdcolumnwidth - \threecolumnwidth, 0);
	
	\begin{axis}[
		width = \threecolumnwidth - \plotylabelwidth,
		height = \threecolumnwidth - \plotylabelwidth,
		at = {(PROvalProto)},
		anchor = north west,
		xshift = \plotylabelwidth,
		xymin = -6.25,
		xymax = 6.25,
		xytick = {-10, -8, ..., 10},
		xlabel = {promoter strength (validation set 1)},
		ylabel = {promoter strength (main data)},
		xlabel style = {name = plot xlabel},
		scatter/classes = {
			noENH={black},
			withENH={35S enhancer}
		}
	]
	
		\addplot [
			scatter,
			scatter src = explicit symbolic,
			only marks,
			mark = solido,
			mark size = 0.5
		] table[x = enrichment.val, y = enrichment.pro, meta = sample.name] {rawData/enrichment_correlation_PROval_proto.tsv};
		
		\addplot [
			scatter,
			scatter src = explicit symbolic,
			only marks,
			mark = text,
			text mark as node = true,
			text mark style = {align = left, node font = \figsmall, anchor = north west, yshift = -2\baselineskip * \coordindex},
			text mark = {\spearman\\[-.25\baselineskip]\rsquare},
			visualization depends on = value \thisrow{spearman}\as\spearman,
			visualization depends on = value \thisrow{rsquare}\as\rsquare
		] table [x expr = \xmin, y expr = \ymax, meta = sample.name] {rawData/enrichment_correlation_PROval_proto_stats.tsv};
		
		\node[anchor = south east, thin, draw, fill = white, align = center, node font = \figsmall] at (rel axis cs: 0.97, 0.03) {enhancer\\\tikz\path[fill = black] (-.3em -1pt, 0) (0, 0) circle (1pt) node[anchor = west] {$-$\vphantom{A}}; ~ \tikz\path[fill = DodgerBlue1] (0, 0) circle (1pt) node[anchor = west] {$+$\vphantom{A}};};
		
	\end{axis}
	
	\protosymbol[(-.4, -.4)]{last plot.south west};
	
	
	%%% correlation plot (first validation set in tobacco, light)
	\coordinate (PROvalLight) at (PROvalLeaf -| \textwidth - \threecolumnwidth, 0);
	
	\begin{axis}[
		width = \threecolumnwidth - \plotylabelwidth,
		height = \threecolumnwidth - \plotylabelwidth,
		at = {(PROvalLight)},
		anchor = north west,
		xshift = \plotylabelwidth,
		xymin = -6.25,
		xymax = 6.25,
		xytick = {-10, -8, ..., 10},
		xlabel = {promoter strength (validation set 1)},
		ylabel = {promoter strength (main data)},
		xlabel style = {name = plot xlabel},
		scatter/classes = {
			noENH={black},
			withENH={35S enhancer}
		}
	]
	
		\addplot [
			scatter,
			scatter src = explicit symbolic,
			only marks,
			mark = solido,
			mark size = 0.5
		] table[x = enrichment.val, y = enrichment.pro, meta = sample.name] {rawData/enrichment_correlation_PROval_light.tsv};
		
		\addplot [
			scatter,
			scatter src = explicit symbolic,
			only marks,
			mark = text,
			text mark as node = true,
			text mark style = {align = left, node font = \figsmall, anchor = north west, yshift = -2\baselineskip * \coordindex},
			text mark = {\spearman\\[-.25\baselineskip]\rsquare},
			visualization depends on = value \thisrow{spearman}\as\spearman,
			visualization depends on = value \thisrow{rsquare}\as\rsquare
		] table [x expr = \xmin, y expr = \ymax, meta = sample.name] {rawData/enrichment_correlation_PROval_light_stats.tsv};
		
		\node[anchor = south east, thin, draw, fill = white, align = center, node font = \figsmall] at (rel axis cs: 0.97, 0.03) {enhancer\\\tikz\path[fill = black] (-.3em -1pt, 0) (0, 0) circle (1pt) node[anchor = west] {$-$\vphantom{A}}; ~ \tikz\path[fill = DodgerBlue1] (0, 0) circle (1pt) node[anchor = west] {$+$\vphantom{A}};};
		
	\end{axis}
	
	\lightsymbol[(-.4, -.4)]{last plot.south west};
	
	
	%%% correlation plot (second validation set in tobacco, dark)
	\coordinate[yshift = -\columnsep] (PROevoLeaf) at (plot xlabel.south -| PROvalLeaf);
	
	\begin{axis}[
		width = \threecolumnwidth - \plotylabelwidth,
		height = \threecolumnwidth - \plotylabelwidth,
		at = {(PROevoLeaf)},
		anchor = north west,
		xshift = \plotylabelwidth,
		xymin = -6.25,
		xymax = 6.25,
		xytick = {-10, -8, ..., 10},
		xlabel = {promoter strength (validation set 2)},
		ylabel = {promoter strength (main data)},
		xlabel style = {name = plot xlabel},
		scatter/classes = {
			noENH={black},
			withENH={35S enhancer}
		}
	]
	
		\addplot [
			scatter,
			scatter src = explicit symbolic,
			only marks,
			mark = solido,
			mark size = 0.5
		] table[x = enrichment.val, y = enrichment.pro, meta = sample.name] {rawData/enrichment_correlation_PROevo_dark.tsv};
		
		\addplot [
			scatter,
			scatter src = explicit symbolic,
			only marks,
			mark = text,
			text mark as node = true,
			text mark style = {align = left, node font = \figsmall, anchor = north west, yshift = -2\baselineskip * \coordindex},
			text mark = {\spearman\\[-.25\baselineskip]\rsquare},
			visualization depends on = value \thisrow{spearman}\as\spearman,
			visualization depends on = value \thisrow{rsquare}\as\rsquare
		] table [x expr = \xmin, y expr = \ymax, meta = sample.name] {rawData/enrichment_correlation_PROevo_dark_stats.tsv};
		
		\node[anchor = south east, thin, draw, fill = white, align = center, node font = \figsmall] at (rel axis cs: 0.97, 0.03) {enhancer\\\tikz\path[fill = black] (-.3em -1pt, 0) (0, 0) circle (1pt) node[anchor = west] {$-$\vphantom{A}}; ~ \tikz\path[fill = DodgerBlue1] (0, 0) circle (1pt) node[anchor = west] {$+$\vphantom{A}};};
		
	\end{axis}
	
	\leafsymbol[(-.4, -.4)]{last plot.south west};
	

	%%% correlation plot (second validation set in protoplasts, dark)
	\coordinate (PROevoProto) at (PROevoLeaf -| PROvalProto);
	
	\begin{axis}[
		width = \threecolumnwidth - \plotylabelwidth,
		height = \threecolumnwidth - \plotylabelwidth,
		at = {(PROevoProto)},
		anchor = north west,
		xshift = \plotylabelwidth,
		xymin = -6.25,
		xymax = 6.25,
		xytick = {-10, -8, ..., 10},
		xlabel = {promoter strength (validation set 2)},
		ylabel = {promoter strength (main data)},
		xlabel style = {name = plot xlabel},
		scatter/classes = {
			noENH={black},
			withENH={35S enhancer}
		}
	]
	
		\addplot [
			scatter,
			scatter src = explicit symbolic,
			only marks,
			mark = solido,
			mark size = 0.5
		] table[x = enrichment.val, y = enrichment.pro, meta = sample.name] {rawData/enrichment_correlation_PROevo_proto.tsv};
		
		\addplot [
			scatter,
			scatter src = explicit symbolic,
			only marks,
			mark = text,
			text mark as node = true,
			text mark style = {align = left, node font = \figsmall, anchor = north west, yshift = -2\baselineskip * \coordindex},
			text mark = {\spearman\\[-.25\baselineskip]\rsquare},
			visualization depends on = value \thisrow{spearman}\as\spearman,
			visualization depends on = value \thisrow{rsquare}\as\rsquare
		] table [x expr = \xmin, y expr = \ymax, meta = sample.name] {rawData/enrichment_correlation_PROevo_proto_stats.tsv};
		
		\node[anchor = south east, thin, draw, fill = white, align = center, node font = \figsmall] at (rel axis cs: 0.97, 0.03) {enhancer\\\tikz\path[fill = black] (-.3em -1pt, 0) (0, 0) circle (1pt) node[anchor = west] {$-$\vphantom{A}}; ~ \tikz\path[fill = DodgerBlue1] (0, 0) circle (1pt) node[anchor = west] {$+$\vphantom{A}};};
		
	\end{axis}
	
	\protosymbol[(-.4, -.4)]{last plot.south west};
	
	
	%%% correlation plot (second validation set in tobacco, light)
	\coordinate (PROevoLight) at (PROevoLeaf -| PROvalLight);
	
	\begin{axis}[
		width = \threecolumnwidth - \plotylabelwidth,
		height = \threecolumnwidth - \plotylabelwidth,
		at = {(PROevoLight)},
		anchor = north west,
		xshift = \plotylabelwidth,
		xymin = -6.25,
		xymax = 6.25,
		xytick = {-10, -8, ..., 10},
		xlabel = {promoter strength (validation set 2)},
		ylabel = {promoter strength (main data)},
		xlabel style = {name = plot xlabel},
		scatter/classes = {
			noENH={black},
			withENH={35S enhancer}
		}
	]
	
		\addplot [
			scatter,
			scatter src = explicit symbolic,
			only marks,
			mark = solido,
			mark size = 0.5
		] table[x = enrichment.val, y = enrichment.pro, meta = sample.name] {rawData/enrichment_correlation_PROevo_light.tsv};
		
		\addplot [
			scatter,
			scatter src = explicit symbolic,
			only marks,
			mark = text,
			text mark as node = true,
			text mark style = {align = left, node font = \figsmall, anchor = north west, yshift = -2\baselineskip * \coordindex},
			text mark = {\spearman\\[-.25\baselineskip]\rsquare},
			visualization depends on = value \thisrow{spearman}\as\spearman,
			visualization depends on = value \thisrow{rsquare}\as\rsquare
		] table [x expr = \xmin, y expr = \ymax, meta = sample.name] {rawData/enrichment_correlation_PROevo_light_stats.tsv};
		
		\node[anchor = south east, thin, draw, fill = white, align = center, node font = \figsmall] at (rel axis cs: 0.97, 0.03) {enhancer\\\tikz\path[fill = black] (-.3em -1pt, 0) (0, 0) circle (1pt) node[anchor = west] {$-$\vphantom{A}}; ~ \tikz\path[fill = DodgerBlue1] (0, 0) circle (1pt) node[anchor = west] {$+$\vphantom{A}};};
		
	\end{axis}
	
	\lightsymbol[(-.4, -.4)]{last plot.south west};
	
	
	
	%%% subfigure labels
	\subfiglabel[yshift = .5\columnsep]{PROvalLeaf}
	\subfiglabel[yshift = .5\columnsep]{PROvalProto}
	\subfiglabel[yshift = .5\columnsep]{PROvalLight}
	\subfiglabel[yshift = .5\columnsep]{PROevoLeaf}
	\subfiglabel[yshift = .5\columnsep]{PROevoProto}
	\subfiglabel[yshift = .5\columnsep]{PROevoLight}

\end{tikzpicture}%
			\caption{%
				\textbf{Promoter strength in small validation libraries correlates highly with comprehensive data.}\titleend
				\subfigrange{A}{C} Correlation between the strength of promoters present in the comprehensive promoter libraries (main data) and in a separate, smaller validation library. The promoter strength was determined in tobacco leaves \parensubfig{A} and maize protoplasts \parensubfig{B} that were kept in the dark prior to mRNA extraction. Additionally, promoter strength was measured in tobacco leaves that were kept for two days in 16h light/8h dark cycles prior to mRNA extraction \parensubfig{C}.\nextentry
				\subfigrange{D}{F} As in (\textbf{\plainsubfigref{A}}-\textbf{\plainsubfigref{C}}) but for a second validation library.
			}%
			\label{sfig:val-cor}%
		\end{sfig}
		
		\begin{sfig}
			%\tikzset{external/export next = false}

\begin{tikzpicture}
	
	%%% enrichment by TATA and GC (tobacco)
	\coordinate[yshift = -\columnsep] (enrTATAGCleaf) at (0, 0);
	
	\leafsymbol{enrTATAGCleaf}; 
	
	\begin{hgroupplot}[%
		ylabel = promoter strength,
		ymin = -9.5,
		ymax = 8,
		ytick = {-10, -8, ..., 10},
		group position = {anchor = above north west, at = {(enrTATAGCleaf)}, xshift = \plotylabelwidth, yshift = -1.25\baselineskip},
		group/every plot/.append style = {x grids = false}
	]{\twocolumnwidth}{3}{GC content}
	
			
		\nextgroupplot[
			title = Arabidopsis,
			x tick table half = {rawData/enrichment_TATA+GC_At_leaf_boxplot.tsv}{LaTeX.label}
		]
			
		% half violin and box plot
		\halfviolinbox[violin color half = arabidopsis]{rawData/enrichment_TATA+GC_At_leaf_boxplot.tsv}{rawData/enrichment_TATA+GC_At_leaf_violin.tsv}
			
		% add sample size
		\samplesizehalf[violin color half = arabidopsis]{rawData/enrichment_TATA+GC_At_leaf_boxplot.tsv}{id}{n}
		
		% add pvalues
		\signifall*{rawData/enrichment_TATA+GC_At_leaf_pvalues.tsv}	
		
		
		\nextgroupplot[
			title = Maize,
			x tick table half = {rawData/enrichment_TATA+GC_Zm_leaf_boxplot.tsv}{LaTeX.label}
		]
			
		% half violin and box plot
		\halfviolinbox[violin color half = maize]{rawData/enrichment_TATA+GC_Zm_leaf_boxplot.tsv}{rawData/enrichment_TATA+GC_Zm_leaf_violin.tsv}
			
		% add sample size
		\samplesizehalf[violin color half = maize]{rawData/enrichment_TATA+GC_Zm_leaf_boxplot.tsv}{id}{n}
		
		% add pvalues
		\signifall*{rawData/enrichment_TATA+GC_Zm_leaf_pvalues.tsv}
		
		
		\nextgroupplot[
			title = Sorghum,
			x tick table half = {rawData/enrichment_TATA+GC_Sb_leaf_boxplot.tsv}{LaTeX.label}
		]
			
		% half violin and box plot
		\halfviolinbox[violin color half = sorghum]{rawData/enrichment_TATA+GC_Sb_leaf_boxplot.tsv}{rawData/enrichment_TATA+GC_Sb_leaf_violin.tsv}
			
		% add sample size
		\samplesizehalf[violin color half = sorghum]{rawData/enrichment_TATA+GC_Sb_leaf_boxplot.tsv}{id}{n}
		
		% add pvalues
		\signifall*{rawData/enrichment_TATA+GC_Sb_leaf_pvalues.tsv}	

	\end{hgroupplot}
	
	\node[anchor = south west] (legend) at (group c1r1.above north west) {\textbf{TATA-box} motif:\vphantom{g} $-$};
	\coordinate (legend symbol) at ($(legend.south east) + (.25em, .2em)$);
	\node[anchor = base west] at ($(legend.base east) + (.6em, 0)$) {$+$};
	
	\fill[thin, draw = black, /pgfplots/violin color half = arabidopsis, fill = viocolleft, fill opacity = .5, x = .25em, y = .1em, rotate = 90, shift = {(legend symbol)}] plot[domain = 0:6] (\x,{4*1/exp(((\x-3)^2)/2)}) -- cycle;
	\fill[thin, draw = black, /pgfplots/violin color half = arabidopsis, fill = viocolright, fill opacity = .5, x = .25em, y = -.1em, rotate = 90, shift = {(legend symbol)}, yshift = -.1em] plot[domain = 0:6] (\x,{4*1/exp(((\x-3)^2)/2)}) -- cycle;
	
	
	%%% enrichment by TATA and GC (protoplasts)
	\coordinate (enrTATAGCproto) at (\textwidth - \twocolumnwidth, 0 |- enrTATAGCleaf);
	
	\protosymbol{enrTATAGCproto}; 
	
	\begin{hgroupplot}[%
		ylabel = promoter strength,
		ymin = -9.5,
		ymax = 8,
		ytick = {-10, -8, ..., 10},
		group position = {anchor = above north west, at = {(enrTATAGCproto)}, xshift = \plotylabelwidth, yshift = -\baselineskip},
		group/every plot/.append style = {x grids = false}
	]{\twocolumnwidth}{3}{GC content}
	
			
		\nextgroupplot[
			title = Arabidopsis,
			x tick table half = {rawData/enrichment_TATA+GC_At_proto_boxplot.tsv}{LaTeX.label}
		]
			
		% half violin and box plot
		\halfviolinbox[violin color half = arabidopsis]{rawData/enrichment_TATA+GC_At_proto_boxplot.tsv}{rawData/enrichment_TATA+GC_At_proto_violin.tsv}
			
		% add sample size
		\samplesizehalf[violin color half = arabidopsis]{rawData/enrichment_TATA+GC_At_proto_boxplot.tsv}{id}{n}
		
		% add pvalues
		\signifall*{rawData/enrichment_TATA+GC_At_proto_pvalues.tsv}	
		
		
		\nextgroupplot[
			title = Maize,
			x tick table half = {rawData/enrichment_TATA+GC_Zm_proto_boxplot.tsv}{LaTeX.label}
		]
			
		% half violin and box plot
		\halfviolinbox[violin color half = maize]{rawData/enrichment_TATA+GC_Zm_proto_boxplot.tsv}{rawData/enrichment_TATA+GC_Zm_proto_violin.tsv}
			
		% add sample size
		\samplesizehalf[violin color half = maize]{rawData/enrichment_TATA+GC_Zm_proto_boxplot.tsv}{id}{n}
		
		% add pvalues
		\signifall*{rawData/enrichment_TATA+GC_Zm_proto_pvalues.tsv}
		
		
		\nextgroupplot[
			title = Sorghum,
			x tick table half = {rawData/enrichment_TATA+GC_Sb_proto_boxplot.tsv}{LaTeX.label}
		]
			
		% half violin and box plot
		\halfviolinbox[violin color half = sorghum]{rawData/enrichment_TATA+GC_Sb_proto_boxplot.tsv}{rawData/enrichment_TATA+GC_Sb_proto_violin.tsv}
			
		% add sample size
		\samplesizehalf[violin color half = sorghum]{rawData/enrichment_TATA+GC_Sb_proto_boxplot.tsv}{id}{n}
		
		% add pvalues
		\signifall*{rawData/enrichment_TATA+GC_Sb_proto_pvalues.tsv}	

	\end{hgroupplot}
	
	\node[anchor = south west] (legend) at (group c1r1.above north west) {\textbf{TATA-box} motif:\vphantom{g} $-$};
	\coordinate (legend symbol) at ($(legend.south east) + (.25em, .2em)$);
	\node[anchor = base west] at ($(legend.base east) + (.6em, 0)$) {$+$};
	
	\fill[thin, draw = black, /pgfplots/violin color half = arabidopsis, fill = viocolleft, fill opacity = .5, x = .25em, y = .1em, rotate = 90, shift = {(legend symbol)}] plot[domain = 0:6] (\x,{4*1/exp(((\x-3)^2)/2)}) -- cycle;
	\fill[thin, draw = black, /pgfplots/violin color half = arabidopsis, fill = viocolright, fill opacity = .5, x = .25em, y = -.1em, rotate = 90, shift = {(legend symbol)}, yshift = -.1em] plot[domain = 0:6] (\x,{4*1/exp(((\x-3)^2)/2)}) -- cycle;
	
	
	%%% subfigure labels
	\subfiglabel[yshift = .5\columnsep]{enrTATAGCleaf}
	\subfiglabel[yshift = .5\columnsep]{enrTATAGCproto}

\end{tikzpicture}%
			\caption{%
				\textbf{The effect of the TATA-box on promoter strength is not a result of decreased GC content.}\titleend
				\subfigref{A}\subfigref{B} Violin plots of promoter strength in tobacco leaves \parensubfig{A} or maize protoplasts \parensubfig{B}. Promoters were grouped by GC content and split into promoters without (left half, darker color) or with (right half, lighter color) a TATA-box. Violin plots are as defined in \autoref{fig:overview}, except only one half is shown.
			}%
			\label{sfig:TATA-GC}%
		\end{sfig}
		
		\begin{sfig}
			%\tikzset{external/export next = false}

\begin{tikzpicture}

	%%% enrichment by BREu (tobacco)
	\coordinate (enrBREuLeaf) at (0, 0);
	
	\leafsymbol{enrBREuLeaf};
	
	\begin{axis}[%
			anchor = above north west,
			at = {(enrBREuLeaf)},
			xshift = \plotylabelwidth,
			yshift = -1.25\baselineskip,
			ylabel = promoter strength,
			ymin = -9.5,
			ymax = 8,
			ytick = {-10, -8, ..., 10},
			x tick table half = {rawData/enrichment_BREu_leaf_boxplot.tsv}{LaTeX.label},
			xticklabel style = {name = xticklabel},
			x grids = false
		]
	
		% half violin and box plot
		\halfviolinbox[violin colors half = {arabidopsis}{maize}{sorghum}]{rawData/enrichment_BREu_leaf_boxplot.tsv}{rawData/enrichment_BREu_leaf_violin.tsv}
			
		% add sample size
		\samplesizehalf[sample color = LaTeX.label]{rawData/enrichment_BREu_leaf_boxplot.tsv}{id}{n}
		
		% add pvalues
		\signif*{rawData/enrichment_BREu_leaf_pvalues.tsv}{1}{2}
		\signif*{rawData/enrichment_BREu_leaf_pvalues.tsv}{3}{4}
		\signif*{rawData/enrichment_BREu_leaf_pvalues.tsv}{5}{6}
	
	\end{axis}
	
	\node[anchor = south west] (legend) at (last plot.north west) {\textbf{BRE\textsuperscript{u}} element:\vphantom{g} $-$};
	\coordinate (legend symbol) at ($(legend.south east) + (.25em, .2em)$);
	\node[anchor = base west] at ($(legend.base east) + (.6em, 0)$) {$+$};
	
	\fill[thin, draw = black, /pgfplots/violin color half = arabidopsis, fill = viocolleft, fill opacity = .5, x = .25em, y = .1em, rotate = 90, shift = {(legend symbol)}] plot[domain = 0:6] (\x,{4*1/exp(((\x-3)^2)/2)}) -- cycle;
	\fill[thin, draw = black, /pgfplots/violin color half = arabidopsis, fill = viocolright, fill opacity = .5, x = .25em, y = -.1em, rotate = 90, shift = {(legend symbol)}, yshift = -.1em] plot[domain = 0:6] (\x,{4*1/exp(((\x-3)^2)/2)}) -- cycle;
	
	
	%%% enrichment by BREu (protoplasts)
	\coordinate (enrBREuProto) at (enrBREuLeaf -| \twocolumnwidth - \fourcolumnwidth, 0);
	
	\protosymbol{enrBREuProto};
	
	\begin{axis}[%
			anchor = above north west,
			at = {(enrBREuProto)},
			xshift = \plotylabelwidth,
			yshift = -1.25\baselineskip,
			ylabel = promoter strength,
			ymin = -9.5,
			ymax = 8,
			ytick = {-10, -8, ..., 10},
			x tick table half = {rawData/enrichment_BREu_Proto_boxplot.tsv}{LaTeX.label},
			x grids = false
		]
	
		% half violin and box plot
		\halfviolinbox[violin colors half = {arabidopsis}{maize}{sorghum}]{rawData/enrichment_BREu_Proto_boxplot.tsv}{rawData/enrichment_BREu_Proto_violin.tsv}
			
		% add sample size
		\samplesizehalf[sample color = LaTeX.label]{rawData/enrichment_BREu_Proto_boxplot.tsv}{id}{n}
		
		% add pvalues
		\signif*{rawData/enrichment_BREu_Proto_pvalues.tsv}{1}{2}
		\signif*{rawData/enrichment_BREu_Proto_pvalues.tsv}{3}{4}
		\signif*{rawData/enrichment_BREu_Proto_pvalues.tsv}{5}{6}
	
	\end{axis}
	
	\node[anchor = south west] (legend) at (last plot.north west) {\textbf{BRE\textsuperscript{u}} element:\vphantom{g} $-$};
	\coordinate (legend symbol) at ($(legend.south east) + (.25em, .2em)$);
	\node[anchor = base west] at ($(legend.base east) + (.6em, 0)$) {$+$};
	
	\fill[thin, draw = black, /pgfplots/violin color half = arabidopsis, fill = viocolleft, fill opacity = .5, x = .25em, y = .1em, rotate = 90, shift = {(legend symbol)}] plot[domain = 0:6] (\x,{4*1/exp(((\x-3)^2)/2)}) -- cycle;
	\fill[thin, draw = black, /pgfplots/violin color half = arabidopsis, fill = viocolright, fill opacity = .5, x = .25em, y = -.1em, rotate = 90, shift = {(legend symbol)}, yshift = -.1em] plot[domain = 0:6] (\x,{4*1/exp(((\x-3)^2)/2)}) -- cycle;
	

	%%% enrichment by BREd (tobacco)
	\coordinate (enrBREdLeaf) at (enrBREuLeaf -| \textwidth - \twocolumnwidth, 0);
	
	\leafsymbol{enrBREdLeaf};
	
	\begin{axis}[%
			anchor = above north west,
			at = {(enrBREdLeaf)},
			xshift = \plotylabelwidth,
			yshift = -1.25\baselineskip,
			ylabel = promoter strength,
			ymin = -9.5,
			ymax = 8,
			ytick = {-10, -8, ..., 10},
			x tick table half = {rawData/enrichment_BREd_leaf_boxplot.tsv}{LaTeX.label},
			x grids = false
		]
	
		% half violin and box plot
		\halfviolinbox[violin colors half = {arabidopsis}{maize}{sorghum}]{rawData/enrichment_BREd_leaf_boxplot.tsv}{rawData/enrichment_BREd_leaf_violin.tsv}
			
		% add sample size
		\samplesizehalf[sample color = LaTeX.label]{rawData/enrichment_BREd_leaf_boxplot.tsv}{id}{n}
		
		% add pvalues
		\signif*{rawData/enrichment_BREd_leaf_pvalues.tsv}{1}{2}
		\signif*{rawData/enrichment_BREd_leaf_pvalues.tsv}{3}{4}
		\signif*{rawData/enrichment_BREd_leaf_pvalues.tsv}{5}{6}
	
	\end{axis}
	
	\node[anchor = south west] (legend) at (last plot.north west) {\textbf{BRE\textsuperscript{d}} element:\vphantom{g} $-$};
	\coordinate (legend symbol) at ($(legend.south east) + (.25em, .2em)$);
	\node[anchor = base west] at ($(legend.base east) + (.6em, 0)$) {$+$};
	
	\fill[thin, draw = black, /pgfplots/violin color half = arabidopsis, fill = viocolleft, fill opacity = .5, x = .25em, y = .1em, rotate = 90, shift = {(legend symbol)}] plot[domain = 0:6] (\x,{4*1/exp(((\x-3)^2)/2)}) -- cycle;
	\fill[thin, draw = black, /pgfplots/violin color half = arabidopsis, fill = viocolright, fill opacity = .5, x = .25em, y = -.1em, rotate = 90, shift = {(legend symbol)}, yshift = -.1em] plot[domain = 0:6] (\x,{4*1/exp(((\x-3)^2)/2)}) -- cycle;
	
	
	%%% enrichment by BREd (protoplasts)
	\coordinate (enrBREdProto) at (enrBREdLeaf -| \textwidth - \fourcolumnwidth, 0);
	
	\protosymbol{enrBREdProto};
	
	\begin{axis}[%
			anchor = above north west,
			at = {(enrBREdProto)},
			xshift = \plotylabelwidth,
			yshift = -1.25\baselineskip,
			ylabel = promoter strength,
			ymin = -9.5,
			ymax = 8,
			ytick = {-10, -8, ..., 10},
			x tick table half = {rawData/enrichment_BREd_Proto_boxplot.tsv}{LaTeX.label},
			x grids = false
		]
	
		% half violin and box plot
		\halfviolinbox[violin colors half = {arabidopsis}{maize}{sorghum}]{rawData/enrichment_BREd_Proto_boxplot.tsv}{rawData/enrichment_BREd_Proto_violin.tsv}
			
		% add sample size
		\samplesizehalf[sample color = LaTeX.label]{rawData/enrichment_BREd_Proto_boxplot.tsv}{id}{n}
		
		% add pvalues
		\signif*{rawData/enrichment_BREd_Proto_pvalues.tsv}{1}{2}
		\signif*{rawData/enrichment_BREd_Proto_pvalues.tsv}{3}{4}
		\signif*{rawData/enrichment_BREd_Proto_pvalues.tsv}{5}{6}
	
	\end{axis}
	
	\node[anchor = south west] (legend) at (last plot.north west) {\textbf{BRE\textsuperscript{d}} element:\vphantom{g} $-$};
	\coordinate (legend symbol) at ($(legend.south east) + (.25em, .2em)$);
	\node[anchor = base west] at ($(legend.base east) + (.6em, 0)$) {$+$};
	
	\fill[thin, draw = black, /pgfplots/violin color half = arabidopsis, fill = viocolleft, fill opacity = .5, x = .25em, y = .1em, rotate = 90, shift = {(legend symbol)}] plot[domain = 0:6] (\x,{4*1/exp(((\x-3)^2)/2)}) -- cycle;
	\fill[thin, draw = black, /pgfplots/violin color half = arabidopsis, fill = viocolright, fill opacity = .5, x = .25em, y = -.1em, rotate = 90, shift = {(legend symbol)}, yshift = -.1em] plot[domain = 0:6] (\x,{4*1/exp(((\x-3)^2)/2)}) -- cycle;
	
	
	%%% mutate BRE motif (logo plots)
	\coordinate[yshift = -\columnsep] (variantsBREumut) at (enrBREuLeaf |- xticklabel.south);
	
	\coordinate[shift = {(\plotylabelwidth, \columnsep)}] (last plot) at (variantsBREumut);
	
	\foreach \variant in {WT, mut} {
			\begin{axis}[
				at = {(last plot.south west)},
				yshift = -.75\columnsep,
				anchor = above north west,
				width = .85\threecolumnwidth - \plotylabelwidth,
				height = 0.5cm,
				logo axis = IC,
				logo y axis/.append style = {ylabel = IC},
				title = {\vphantom{\textsuperscript{d}}BRE\textsuperscript{u} (\variant)},
				title style = {draw = none, fill = none},
				logo plot
			]
			
				\fill[gray, fill opacity = 0.2] (8.5, \ymin) rectangle (16.5, \ymax);
			
				\addlogoplot{rawData/motif_mutBREu_\variant.tsv};
				
			\end{axis}
	}
	
	
	\coordinate[shift = {(-\plotylabelwidth, -\columnsep)}] (variantsBREdmut) at (last plot.south west);
	
	\foreach \variant in {WT, mut} {
			\begin{axis}[
				at = {(last plot.south west)},
				yshift = -.75\columnsep,
				anchor = above north west,
				width = .85\threecolumnwidth - \plotylabelwidth,
				height = 0.5cm,
				logo axis = IC,
				logo y axis/.append style = {ylabel = IC},
				title = {\vphantom{\textsuperscript{d}}BRE\textsuperscript{d} (\variant)},
				title style = {draw = none, fill = none},
				logo plot
			]
			
				\fill[gray, fill opacity = 0.2] (8.5, \ymin) rectangle (16.5, \ymax);
			
				\addlogoplot{rawData/motif_mutBREd_\variant.tsv};
				
			\end{axis}
	}
	
	
	\coordinate(variantsBREins) at (\twothirdcolumnwidth - 1.15\threecolumnwidth, 0 |- variantsBREumut);
	
	\coordinate[shift = {(\plotylabelwidth, \columnsep)}] (last plot) at (variantsBREins);
	
	\foreach \variant/\vartitle in {WT, +BREu/+ BRE\textsuperscript{u}, +BREd/+ BRE\textsuperscript{d}} {
			\begin{axis}[
				at = {(last plot.south west)},
				yshift = -.75\columnsep,
				anchor = above north west,
				width = .85\threecolumnwidth - \plotylabelwidth,
				height = 0.5cm,
				logo axis = IC,
				logo y axis/.append style = {ylabel = IC},
				title = {\vphantom{\textsuperscript{d}}no BRE (\vartitle)},
				title style = {draw = none, fill = none},
				logo plot
			]
			
				\fill[gray, fill opacity = 0.2] (8.5, \ymin) rectangle (16.5, \ymax);
			
				\addlogoplot{rawData/motif_insBRE_\variant.tsv};
				
			\end{axis}
	}
	
	
	%%% mutate BRE motif (promoter strength)
	\coordinate (enrBREmut) at (\textwidth - 1.3\threecolumnwidth, 0 |- variantsBREumut);
	
	\begin{hgroupplot}[%
		ylabel = {rel. promoter strength},
		ymin = -1.75,
		ymax = 2,
		ytick = {-10, -9, ..., 10},
		zero line,
		xticklabel style = {rotate = 45, align = right, anchor = north east, name = xticklabel},
		group position = {anchor = above north west, at = {(enrBREmut)}, xshift = \plotylabelwidth},
		group/every plot/.append style = {
			x grids = false,
			typeset ticklabels with strut 
		}
	]{1.3\threecolumnwidth}{2}{}

			
		\nextgroupplot[
			title = {\phantom{tobacco}\\\phantom{leaves}},
			x tick table = {rawData/enrichment_mutateBRE_leaf_boxplot.tsv}{LaTeX.label}
		]
		
			% boxplot
			\boxplots{%
				box color = leafCol,
				box shade,
				fill opacity = 0.5%
			}{rawData/enrichment_mutateBRE_leaf_boxplot.tsv}{rawData/enrichment_mutateBRE_leaf_boxplot_outliers.tsv}
			
			% add sample size
			\samplesize{rawData/enrichment_mutateBRE_leaf_boxplot.tsv}{id}{n}
			
			% add significance
			\signifallsimple*{rawData/enrichment_mutateBRE_leaf_boxplot.tsv}{id}{p.value}
		
		
		\nextgroupplot[
			title = {\phantom{maize}\\\phantom{protoplasts}},
			x tick table = {rawData/enrichment_mutateBRE_proto_boxplot.tsv}{LaTeX.label}
		]
		
			% boxplot
			\boxplots{%
				box color = protoCol,
				box shade,
				fill opacity = 0.5%
			}{rawData/enrichment_mutateBRE_proto_boxplot.tsv}{rawData/enrichment_mutateBRE_proto_boxplot_outliers.tsv}
			
			% add sample size
			\samplesize{rawData/enrichment_mutateBRE_proto_boxplot.tsv}{id}{n}
			
			% add significance
			\signifallsimple*{rawData/enrichment_mutateBRE_proto_boxplot.tsv}{id}{p.value}
			
	\end{hgroupplot}
	
	\distance{title.south}{title.north}
	
	\leafsymbol[(.5\ydistance, -.5\ydistance)]{group c1r1.west |- title.north}
	\node[anchor = east, node font = \fignormal, text depth = .15\baselineskip, align = right] at (group c1r1.east |- title) {tobacco\\leaves};
	\protosymbol[(-.5\ydistance, -.5\ydistance)]{group c2r1.east |- title.north}
	\node[anchor = west, node font = \fignormal, text depth = .15\baselineskip, align = left] at (group c2r1.west |- title) {maize\\protoplasts};

	
	%%% subfigure labels
	\subfiglabel[yshift = .5\columnsep]{enrBREuLeaf}
	\subfiglabel[yshift = .5\columnsep]{enrBREuProto}
	\subfiglabel[yshift = .5\columnsep]{enrBREdLeaf}
	\subfiglabel[yshift = .5\columnsep]{enrBREdProto}
	\subfiglabel[yshift = .5\columnsep]{variantsBREumut}
	\subfiglabel[yshift = .5\columnsep]{variantsBREdmut}
	\subfiglabel[yshift = .5\columnsep]{variantsBREins}
	\subfiglabel[yshift = .5\columnsep]{enrBREmut}

\end{tikzpicture}%
			\caption{%
				\textbf{The BRE\textsuperscript{u} element is most active in maize protoplasts.}\titleend
				\subfigrange{A}{D} Violin plots of promoter strength in tobacco leaves \parensubfig[\subfigref{A}]{C} or maize protoplasts \parensubfig[\subfigref{B}]{D}. Promoters were grouped by GC content and split into promoters without (left half, darker color) or with (right half, lighter color) a BRE\textsuperscript{u} \parensubfig[\subfigref{A}]{B}, or BRE\textsuperscript{d} \parensubfig[\subfigref{C}]{D} element. Violin plots are as defined in \autoref{fig:overview}, except only one half is shown.\nextentry
				\subfigref{E}\subfigref{G} Logoplots for promoters with a BRE\textsuperscript{u} \parensubfig{E} or BRE\textsuperscript{d} \parensubfig{F} before (WT) and after (mut) introducing mutations that disrupt the elements.\nextentry
				\subfigref{G} Logoplots for promoters without a BRE (WT) and with an inserted BRE\textsuperscript{u} (+ BRE\textsuperscript{u}) or BRE\textsuperscript{d} (+ BRE\textsuperscript{d}) element.\nextentry
				\subfigref{H} Boxplots (as defined in \autoref{fig:TATA}) for the relative strength of the promoter variants shown in (\textbf{\plainsubfigref{E}}-\textbf{\plainsubfigref{G}}). The corresponding WT promoter was set to 0 (horizontal black line).
			}%
			\label{sfig:BRE}%
		\end{sfig}
					
		\begin{sfig}
			%\tikzset{external/export next = false}

\begin{tikzpicture}

	%%% alignment of TFIIBs
	\coordinate[yshift = .25\columnsep] (alignment) at (0, 0);
	
	\begin{pgfinterruptboundingbox}
		\node[anchor = north west, inner sep = 0pt, text width = \textwidth, xshift = -\columnsep] (alignment node) at (alignment) {%
			\begin{texshade}[alignments.def]{../data/misc/TFIIB-alignment.fa}%
				\noblockskip%
				\hideconsensus%
				\shownumbering{leftright}%
				\threshold[80]{50}%
				\emphblock{1}{283..283}%
				\shadeblock{1}{283..283}{black}{maize!50}%
				\shaderegion{4}{279..279}{black}{arabidopsis!50}%
				\shaderegion{5}{280..280}{black}{arabidopsis!50}%
				\shaderegion{6}{276..276}{black}{arabidopsis!50}%
				\shaderegion{7}{279..279}{black}{arabidopsis!50}%
				\shaderegion{8}{279..279}{black}{arabidopsis!50}%
				\shaderegion{9}{279..279}{black}{arabidopsis!50}%
				\nameseq{1}{Human TFIIB}%
				\nameseq{2}{Mouse TFIIB}%
				\nameseq{3}{Drosophila TFIIB}%
				\nameseq{4}{Arabidopsis TFIIB}%
				\nameseq{5}{Soybean TFIIB}%
				\nameseq{6}{Tobacco TFIIB}%
				\nameseq{7}{Rice TFIIB}%
				\nameseq{8}{Maize TFIIB}%
				\nameseq{9}{Sorghum TFIIB}%
				\nameseq{10}{Maize TFIIB-related}%
			\end{texshade}%
		};
	\end{pgfinterruptboundingbox}

	\path (0, 0) rectangle (alignment node.south -| \textwidth, 0);
	

%	\similaritytable
%
%	\hspace{1cm}%
%	\rowcolors{2}{white}{gray!20}
%
%	\begin{tabular}{ll}
%		\toprule
%		\textbf{name} & \textbf{Uniprot accession} \\
%		\midrule
%		human & Q00403 \\
%		mouse & P62915 \\
%		Drosophila & P29052 \\
%		Arabidopsis & P48512 \\
%		Soybean & P48513 \\
%		Tobacco & A0A1S3YMX6 \\
%		Rice & Q8W0W3 \\
%		Maize & B4FHS1  \\
%		Sorghum & C5X694  \\
%		Maize-2 & A0A3L6EB89 \\
%		\bottomrule
%	\end{tabular}


%	%%% subfigure labels
%	\subfiglabel[yshift = .25\columnsep]{alignment}

\end{tikzpicture}
			\caption{%
				\textbf{The maize genome encodes a TFIIB-related protein with a conserved valine residue required for BRE\textsuperscript{u} recognition.}\titleend
				Alignment of TFIIB and TFIIB-like protein sequences from indicated species. Residues conserved in 80 or 50\% of the sequences are highlighted in dark or light gray, respectively. The valine residue required for recognition of BRE\textsuperscript{u} is highlighted in green.
			}%
			\label{sfig:TFIIB}%
		\end{sfig}

		\begin{sfig}
			%\tikzset{external/export next = false}

\begin{tikzpicture}

	%%% Ypatch position
	\coordinate (YpatchPos) at (0, 0);

	\begin{axis}[
		at = {(YpatchPos)},
		anchor = north west,
		xshift = \plotylabelwidth,
		width = \twocolumnwidth - \plotylabelwidth,
		xlabel = {Y patch position (rel. to TSS)},
		xlabel style = {name = plot xlabel},
		ylabel = promoters (\%),
		xmin = 0,
		xmax = 170,
		ymin = 0,
		y decimals = 1,
		xtick = {6, 26, ..., 166},
		xticklabel = {$\pgfmathparse{\tick < 166 ? \tick - 166 : \tick -165}\pgfmathprintnumber[print sign]{\pgfmathresult}$},
		ybar stacked,
		bar width = 1,
		ybar legend,
		legend columns = 3,
		legend pos = north west
	]

		\addplot [arabidopsis, fill, fill opacity = .5] table [x = pos, y = At] {rawData/Ypatch_position.tsv};
		\addplot [maize, fill, fill opacity = .5] table [x = pos, y = Zm] {rawData/Ypatch_position.tsv};
		\addplot [sorghum, fill, fill opacity = .5] table [x = pos, y = Sb] {rawData/Ypatch_position.tsv};
		
		\legend{Arabidopsis\phantom{A}, Maize\phantom{A}, Sorghum}
	
	\end{axis}


	%%% enrichment by Y patch (tobacco)
	\coordinate[yshift = -\columnsep] (enrYpatchLeaf) at (YpatchPos |- plot xlabel.south);
	
	\leafsymbol{enrYpatchLeaf}; 
	
	\begin{hgroupplot}[%
		ylabel = promoter strength,
		ymin = -9.5,
		ymax = 8,
		ytick = {-10, -8, ..., 10},
		group position = {anchor = above north west, at = {(enrYpatchLeaf)}, xshift = \plotylabelwidth, yshift = -1.25\baselineskip},
		group/every plot/.append style = {x grids = false}
	]{\twocolumnwidth}{3}{GC content}
	
			
		\nextgroupplot[
			title = Arabidopsis,
			x tick table half = {rawData/enrichment_Ypatch+GC_At_leaf_boxplot.tsv}{LaTeX.label}
		]
			
		% half violin and box plot
		\halfviolinbox[violin color half = arabidopsis]{rawData/enrichment_Ypatch+GC_At_leaf_boxplot.tsv}{rawData/enrichment_Ypatch+GC_At_leaf_violin.tsv}
			
		% add sample size
		\samplesizehalf[violin color half = arabidopsis]{rawData/enrichment_Ypatch+GC_At_leaf_boxplot.tsv}{id}{n}
		
		% add pvalues
		\signifall*{rawData/enrichment_Ypatch+GC_At_leaf_pvalues.tsv}	
		
		
		\nextgroupplot[
			title = Maize,
			x tick table half = {rawData/enrichment_Ypatch+GC_Zm_leaf_boxplot.tsv}{LaTeX.label}
		]
			
		% half violin and box plot
		\halfviolinbox[violin color half = maize]{rawData/enrichment_Ypatch+GC_Zm_leaf_boxplot.tsv}{rawData/enrichment_Ypatch+GC_Zm_leaf_violin.tsv}
			
		% add sample size
		\samplesizehalf[violin color half = maize]{rawData/enrichment_Ypatch+GC_Zm_leaf_boxplot.tsv}{id}{n}
		
		% add pvalues
		\signifall*{rawData/enrichment_Ypatch+GC_Zm_leaf_pvalues.tsv}
		
		
		\nextgroupplot[
			title = Sorghum,
			x tick table half = {rawData/enrichment_Ypatch+GC_Sb_leaf_boxplot.tsv}{LaTeX.label}
		]
			
		% half violin and box plot
		\halfviolinbox[violin color half = sorghum]{rawData/enrichment_Ypatch+GC_Sb_leaf_boxplot.tsv}{rawData/enrichment_Ypatch+GC_Sb_leaf_violin.tsv}
			
		% add sample size
		\samplesizehalf[violin color half = sorghum]{rawData/enrichment_Ypatch+GC_Sb_leaf_boxplot.tsv}{id}{n}
		
		% add pvalues
		\signifall*{rawData/enrichment_Ypatch+GC_Sb_leaf_pvalues.tsv}	

	\end{hgroupplot}
	
	\node[anchor = south west] (legend) at (group c1r1.above north west) {\textbf{Y patch} element:\vphantom{g} $-$};
	\coordinate (legend symbol) at ($(legend.south east) + (.25em, .2em)$);
	\node[anchor = base west] at ($(legend.base east) + (.6em, 0)$) {$+$};
	
	\fill[thin, draw = black, /pgfplots/violin color half = arabidopsis, fill = viocolleft, fill opacity = .5, x = .25em, y = .1em, rotate = 90, shift = {(legend symbol)}] plot[domain = 0:6] (\x,{4*1/exp(((\x-3)^2)/2)}) -- cycle;
	\fill[thin, draw = black, /pgfplots/violin color half = arabidopsis, fill = viocolright, fill opacity = .5, x = .25em, y = -.1em, rotate = 90, shift = {(legend symbol)}, yshift = -.1em] plot[domain = 0:6] (\x,{4*1/exp(((\x-3)^2)/2)}) -- cycle;
	
	
	%%% enrichment by Y patch (protoplasts)
	\coordinate[yshift = -\columnsep] (enrYpatchProto) at (YpatchPos |- plot xlabel.south);
	
	\protosymbol{enrYpatchProto}; 
	
	\begin{hgroupplot}[%
		ylabel = promoter strength,
		ymin = -9.5,
		ymax = 8,
		ytick = {-10, -8, ..., 10},
		group position = {anchor = above north west, at = {(enrYpatchProto)}, xshift = \plotylabelwidth, yshift = -1.25\baselineskip},
		group/every plot/.append style = {x grids = false}
	]{\twocolumnwidth}{3}{GC content}
	
			
		\nextgroupplot[
			title = Arabidopsis,
			x tick table half = {rawData/enrichment_Ypatch+GC_At_proto_boxplot.tsv}{LaTeX.label}
		]
			
		% half violin and box plot
		\halfviolinbox[violin color half = arabidopsis]{rawData/enrichment_Ypatch+GC_At_proto_boxplot.tsv}{rawData/enrichment_Ypatch+GC_At_proto_violin.tsv}
			
		% add sample size
		\samplesizehalf[violin color half = arabidopsis]{rawData/enrichment_Ypatch+GC_At_proto_boxplot.tsv}{id}{n}
		
		% add pvalues
		\signifall*{rawData/enrichment_Ypatch+GC_At_proto_pvalues.tsv}	
		
		
		\nextgroupplot[
			title = Maize,
			x tick table half = {rawData/enrichment_Ypatch+GC_Zm_proto_boxplot.tsv}{LaTeX.label}
		]
			
		% half violin and box plot
		\halfviolinbox[violin color half = maize]{rawData/enrichment_Ypatch+GC_Zm_proto_boxplot.tsv}{rawData/enrichment_Ypatch+GC_Zm_proto_violin.tsv}
			
		% add sample size
		\samplesizehalf[violin color half = maize]{rawData/enrichment_Ypatch+GC_Zm_proto_boxplot.tsv}{id}{n}
		
		% add pvalues
		\signifall*{rawData/enrichment_Ypatch+GC_Zm_proto_pvalues.tsv}
		
		
		\nextgroupplot[
			title = Sorghum,
			x tick table half = {rawData/enrichment_Ypatch+GC_Sb_proto_boxplot.tsv}{LaTeX.label}
		]
			
		% half violin and box plot
		\halfviolinbox[violin color half = sorghum]{rawData/enrichment_Ypatch+GC_Sb_proto_boxplot.tsv}{rawData/enrichment_Ypatch+GC_Sb_proto_violin.tsv}
			
		% add sample size
		\samplesizehalf[violin color half = sorghum]{rawData/enrichment_Ypatch+GC_Sb_proto_boxplot.tsv}{id}{n}
		
		% add pvalues
		\signifall*{rawData/enrichment_Ypatch+GC_Sb_proto_pvalues.tsv}	

	\end{hgroupplot}
	
	\node[anchor = south west] (legend) at (group c1r1.above north west) {\textbf{Y patch} element:\vphantom{g} $-$};
	\coordinate (legend symbol) at ($(legend.south east) + (.25em, .2em)$);
	\node[anchor = base west] at ($(legend.base east) + (.6em, 0)$) {$+$};
	
	\fill[thin, draw = black, /pgfplots/violin color half = arabidopsis, fill = viocolleft, fill opacity = .5, x = .25em, y = .1em, rotate = 90, shift = {(legend symbol)}] plot[domain = 0:6] (\x,{4*1/exp(((\x-3)^2)/2)}) -- cycle;
	\fill[thin, draw = black, /pgfplots/violin color half = arabidopsis, fill = viocolright, fill opacity = .5, x = .25em, y = -.1em, rotate = 90, shift = {(legend symbol)}, yshift = -.1em] plot[domain = 0:6] (\x,{4*1/exp(((\x-3)^2)/2)}) -- cycle;
	

	
	%%% subfigure labels
	\subfiglabel[yshift = .5\columnsep]{YpatchPos}
	\subfiglabel[yshift = .5\columnsep]{enrYpatchLeaf}
	\subfiglabel[yshift = .5\columnsep]{enrYpatchProto}

\end{tikzpicture}%
			\caption{%
				\textbf{The Y patch is a plant-specific core promoter element.}\titleend
				\subfigref{A} Histogram showing the percentage of promoters with a TATA-box at the indicated position.\nextentry
				\subfigref{B}\subfigref{C} Violin plots of promoter strength in tobacco leaves \parensubfig{B} or maize protoplasts \parensubfig{C}. Promoters were grouped by GC content and split into promoters without (left half, darker color) or with (right half, lighter color) a Y patch. Violin plots are as defined in \autoref{fig:overview}, except only one half is shown.
			}%
			\label{sfig:Ypatch}%
		\end{sfig}
		
		\begin{sfig}
			%\tikzset{external/export next = false}

\begin{tikzpicture}

	%%% enrichment by Inr (tobacco)
	\coordinate (enrInrLeaf) at (0, 0);
	
	\leafsymbol{enrInrLeaf}; 
	
	\begin{hgroupplot}[%
		ylabel = promoter strength,
		ymin = -9.5,
		ymax = 8,
		ytick = {-10, -8, ..., 10},
		group position = {anchor = above north west, at = {(enrInrLeaf)}, xshift = \plotylabelwidth, yshift = -1.25\baselineskip},
		group/every plot/.append style = {x grids = false}
	]{\twocolumnwidth}{3}{GC content}
	
			
		\nextgroupplot[
			title = Arabidopsis,
			x tick table half = {rawData/enrichment_Inr+GC_At_leaf_boxplot.tsv}{LaTeX.label}
		]
			
		% half violin and box plot
		\halfviolinbox[violin color half = arabidopsis]{rawData/enrichment_Inr+GC_At_leaf_boxplot.tsv}{rawData/enrichment_Inr+GC_At_leaf_violin.tsv}
			
		% add sample size
		\samplesizehalf[violin color half = arabidopsis]{rawData/enrichment_Inr+GC_At_leaf_boxplot.tsv}{id}{n}
		
		% add pvalues
		\signifall*{rawData/enrichment_Inr+GC_At_leaf_pvalues.tsv}	
		
		
		\nextgroupplot[
			title = Maize,
			x tick table half = {rawData/enrichment_Inr+GC_Zm_leaf_boxplot.tsv}{LaTeX.label}
		]
			
		% half violin and box plot
		\halfviolinbox[violin color half = maize]{rawData/enrichment_Inr+GC_Zm_leaf_boxplot.tsv}{rawData/enrichment_Inr+GC_Zm_leaf_violin.tsv}
			
		% add sample size
		\samplesizehalf[violin color half = maize]{rawData/enrichment_Inr+GC_Zm_leaf_boxplot.tsv}{id}{n}
		
		% add pvalues
		\signifall*{rawData/enrichment_Inr+GC_Zm_leaf_pvalues.tsv}
		
		
		\nextgroupplot[
			title = Sorghum,
			x tick table half = {rawData/enrichment_Inr+GC_Sb_leaf_boxplot.tsv}{LaTeX.label}
		]
			
		% half violin and box plot
		\halfviolinbox[violin color half = sorghum]{rawData/enrichment_Inr+GC_Sb_leaf_boxplot.tsv}{rawData/enrichment_Inr+GC_Sb_leaf_violin.tsv}
			
		% add sample size
		\samplesizehalf[violin color half = sorghum]{rawData/enrichment_Inr+GC_Sb_leaf_boxplot.tsv}{id}{n}
		
		% add pvalues
		\signifall*{rawData/enrichment_Inr+GC_Sb_leaf_pvalues.tsv}	

	\end{hgroupplot}
	
	\node[anchor = south west] (legend) at (group c1r1.above north west) {\textbf{Inr} element:\vphantom{g} $-$};
	\coordinate (legend symbol) at ($(legend.south east) + (.25em, .2em)$);
	\node[anchor = base west] at ($(legend.base east) + (.6em, 0)$) {$+$};
	
	\fill[thin, draw = black, /pgfplots/violin color half = arabidopsis, fill = viocolleft, fill opacity = .5, x = .25em, y = .1em, rotate = 90, shift = {(legend symbol)}] plot[domain = 0:6] (\x,{4*1/exp(((\x-3)^2)/2)}) -- cycle;
	\fill[thin, draw = black, /pgfplots/violin color half = arabidopsis, fill = viocolright, fill opacity = .5, x = .25em, y = -.1em, rotate = 90, shift = {(legend symbol)}, yshift = -.1em] plot[domain = 0:6] (\x,{4*1/exp(((\x-3)^2)/2)}) -- cycle;
	
	
	%%% enrichment by Inr (protoplasts)
	\coordinate (enrInrProto) at (\textwidth - \twocolumnwidth, 0 |- enrInrLeaf);
	
	\protosymbol{enrInrProto}; 
	
	\begin{hgroupplot}[%
		ylabel = promoter strength,
		ymin = -9.5,
		ymax = 8,
		ytick = {-10, -8, ..., 10},
		group position = {anchor = above north west, at = {(enrInrProto)}, xshift = \plotylabelwidth, yshift = -1.25\baselineskip},
		group/every plot/.append style = {x grids = false}
	]{\twocolumnwidth}{3}{GC content}
	
			
		\nextgroupplot[
			title = Arabidopsis,
			x tick table half = {rawData/enrichment_Inr+GC_At_proto_boxplot.tsv}{LaTeX.label}
		]
			
		% half violin and box plot
		\halfviolinbox[violin color half = arabidopsis]{rawData/enrichment_Inr+GC_At_proto_boxplot.tsv}{rawData/enrichment_Inr+GC_At_proto_violin.tsv}
			
		% add sample size
		\samplesizehalf[violin color half = arabidopsis]{rawData/enrichment_Inr+GC_At_proto_boxplot.tsv}{id}{n}
		
		% add pvalues
		\signifall*{rawData/enrichment_Inr+GC_At_proto_pvalues.tsv}	
		
		
		\nextgroupplot[
			title = Maize,
			x tick table half = {rawData/enrichment_Inr+GC_Zm_proto_boxplot.tsv}{LaTeX.label}
		]
			
		% half violin and box plot
		\halfviolinbox[violin color half = maize]{rawData/enrichment_Inr+GC_Zm_proto_boxplot.tsv}{rawData/enrichment_Inr+GC_Zm_proto_violin.tsv}
			
		% add sample size
		\samplesizehalf[violin color half = maize]{rawData/enrichment_Inr+GC_Zm_proto_boxplot.tsv}{id}{n}
		
		% add pvalues
		\signifall*{rawData/enrichment_Inr+GC_Zm_proto_pvalues.tsv}
		
		
		\nextgroupplot[
			title = Sorghum,
			x tick table half = {rawData/enrichment_Inr+GC_Sb_proto_boxplot.tsv}{LaTeX.label}
		]
			
		% half violin and box plot
		\halfviolinbox[violin color half = sorghum]{rawData/enrichment_Inr+GC_Sb_proto_boxplot.tsv}{rawData/enrichment_Inr+GC_Sb_proto_violin.tsv}
			
		% add sample size
		\samplesizehalf[violin color half = sorghum]{rawData/enrichment_Inr+GC_Sb_proto_boxplot.tsv}{id}{n}
		
		% add pvalues
		\signifall*{rawData/enrichment_Inr+GC_Sb_proto_pvalues.tsv}	

	\end{hgroupplot}
	
	\node[anchor = south west] (legend) at (group c1r1.above north west) {\textbf{Inr} element:\vphantom{g} $-$};
	\coordinate (legend symbol) at ($(legend.south east) + (.25em, .2em)$);
	\node[anchor = base west] at ($(legend.base east) + (.6em, 0)$) {$+$};
	
	\fill[thin, draw = black, /pgfplots/violin color half = arabidopsis, fill = viocolleft, fill opacity = .5, x = .25em, y = .1em, rotate = 90, shift = {(legend symbol)}] plot[domain = 0:6] (\x,{4*1/exp(((\x-3)^2)/2)}) -- cycle;
	\fill[thin, draw = black, /pgfplots/violin color half = arabidopsis, fill = viocolright, fill opacity = .5, x = .25em, y = -.1em, rotate = 90, shift = {(legend symbol)}, yshift = -.1em] plot[domain = 0:6] (\x,{4*1/exp(((\x-3)^2)/2)}) -- cycle;
	

	%%% enrichment by TCT (tobacco)
	\coordinate[yshift = -\columnsep] (enrTCTLeaf) at (enrInrLeaf |- plot xlabel.south);
	
	\leafsymbol{enrTCTLeaf}; 
	
	\begin{hgroupplot}[%
		ylabel = promoter strength,
		ymin = -9.5,
		ymax = 8,
		ytick = {-10, -8, ..., 10},
		group position = {anchor = above north west, at = {(enrTCTLeaf)}, xshift = \plotylabelwidth, yshift = -1.25\baselineskip},
		group/every plot/.append style = {x grids = false}
	]{\twocolumnwidth}{3}{GC content}
	
			
		\nextgroupplot[
			title = Arabidopsis,
			x tick table half = {rawData/enrichment_TCT+GC_At_leaf_boxplot.tsv}{LaTeX.label}
		]
			
		% half violin and box plot
		\halfviolinbox[violin color half = arabidopsis]{rawData/enrichment_TCT+GC_At_leaf_boxplot.tsv}{rawData/enrichment_TCT+GC_At_leaf_violin.tsv}
			
		% add sample size
		\samplesizehalf[violin color half = arabidopsis]{rawData/enrichment_TCT+GC_At_leaf_boxplot.tsv}{id}{n}
		
		% add pvalues
		\signifall*{rawData/enrichment_TCT+GC_At_leaf_pvalues.tsv}	
		
		
		\nextgroupplot[
			title = Maize,
			x tick table half = {rawData/enrichment_TCT+GC_Zm_leaf_boxplot.tsv}{LaTeX.label}
		]
			
		% half violin and box plot
		\halfviolinbox[violin color half = maize]{rawData/enrichment_TCT+GC_Zm_leaf_boxplot.tsv}{rawData/enrichment_TCT+GC_Zm_leaf_violin.tsv}
			
		% add sample size
		\samplesizehalf[violin color half = maize]{rawData/enrichment_TCT+GC_Zm_leaf_boxplot.tsv}{id}{n}
		
		% add pvalues
		\signifall*{rawData/enrichment_TCT+GC_Zm_leaf_pvalues.tsv}
		
		
		\nextgroupplot[
			title = Sorghum,
			x tick table half = {rawData/enrichment_TCT+GC_Sb_leaf_boxplot.tsv}{LaTeX.label}
		]
			
		% half violin and box plot
		\halfviolinbox[violin color half = sorghum]{rawData/enrichment_TCT+GC_Sb_leaf_boxplot.tsv}{rawData/enrichment_TCT+GC_Sb_leaf_violin.tsv}
			
		% add sample size
		\samplesizehalf[violin color half = sorghum]{rawData/enrichment_TCT+GC_Sb_leaf_boxplot.tsv}{id}{n}
		
		% add pvalues
		\signifall*{rawData/enrichment_TCT+GC_Sb_leaf_pvalues.tsv}	

	\end{hgroupplot}
	
	\node[anchor = south west] (legend) at (group c1r1.above north west) {\textbf{TCT} element:\vphantom{g} $-$};
	\coordinate (legend symbol) at ($(legend.south east) + (.25em, .2em)$);
	\node[anchor = base west] at ($(legend.base east) + (.6em, 0)$) {$+$};
	
	\fill[thin, draw = black, /pgfplots/violin color half = arabidopsis, fill = viocolleft, fill opacity = .5, x = .25em, y = .1em, rotate = 90, shift = {(legend symbol)}] plot[domain = 0:6] (\x,{4*1/exp(((\x-3)^2)/2)}) -- cycle;
	\fill[thin, draw = black, /pgfplots/violin color half = arabidopsis, fill = viocolright, fill opacity = .5, x = .25em, y = -.1em, rotate = 90, shift = {(legend symbol)}, yshift = -.1em] plot[domain = 0:6] (\x,{4*1/exp(((\x-3)^2)/2)}) -- cycle;
	
	
	%%% enrichment by TCT+GC binding site (protoplasts)
	\coordinate (enrTCTProto) at (\textwidth - \twocolumnwidth, 0 |- enrTCTLeaf);
	
	\protosymbol{enrTCTProto}; 
	
	\begin{hgroupplot}[%
		ylabel = promoter strength,
		ymin = -9.5,
		ymax = 8,
		ytick = {-10, -8, ..., 10},
		group position = {anchor = above north west, at = {(enrTCTProto)}, xshift = \plotylabelwidth, yshift = -1.25\baselineskip},
		group/every plot/.append style = {x grids = false}
	]{\twocolumnwidth}{3}{GC content}
	
			
		\nextgroupplot[
			title = Arabidopsis,
			x tick table half = {rawData/enrichment_TCT+GC_At_proto_boxplot.tsv}{LaTeX.label}
		]
			
		% half violin and box plot
		\halfviolinbox[violin color half = arabidopsis]{rawData/enrichment_TCT+GC_At_proto_boxplot.tsv}{rawData/enrichment_TCT+GC_At_proto_violin.tsv}
			
		% add sample size
		\samplesizehalf[violin color half = arabidopsis]{rawData/enrichment_TCT+GC_At_proto_boxplot.tsv}{id}{n}
		
		% add pvalues
		\signifall*{rawData/enrichment_TCT+GC_At_proto_pvalues.tsv}	
		
		
		\nextgroupplot[
			title = Maize,
			x tick table half = {rawData/enrichment_TCT+GC_Zm_proto_boxplot.tsv}{LaTeX.label}
		]
			
		% half violin and box plot
		\halfviolinbox[violin color half = maize]{rawData/enrichment_TCT+GC_Zm_proto_boxplot.tsv}{rawData/enrichment_TCT+GC_Zm_proto_violin.tsv}
			
		% add sample size
		\samplesizehalf[violin color half = maize]{rawData/enrichment_TCT+GC_Zm_proto_boxplot.tsv}{id}{n}
		
		% add pvalues
		\signifall*{rawData/enrichment_TCT+GC_Zm_proto_pvalues.tsv}
		
		
		\nextgroupplot[
			title = Sorghum,
			x tick table half = {rawData/enrichment_TCT+GC_Sb_proto_boxplot.tsv}{LaTeX.label}
		]
			
		% half violin and box plot
		\halfviolinbox[violin color half = sorghum]{rawData/enrichment_TCT+GC_Sb_proto_boxplot.tsv}{rawData/enrichment_TCT+GC_Sb_proto_violin.tsv}
			
		% add sample size
		\samplesizehalf[violin color half = sorghum]{rawData/enrichment_TCT+GC_Sb_proto_boxplot.tsv}{id}{n}
		
		% add pvalues
		\signifall*{rawData/enrichment_TCT+GC_Sb_proto_pvalues.tsv}	

	\end{hgroupplot}
	
	\node[anchor = south west] (legend) at (group c1r1.above north west) {\textbf{TCT} element:\vphantom{g} $-$};
	\coordinate (legend symbol) at ($(legend.south east) + (.25em, .2em)$);
	\node[anchor = base west] at ($(legend.base east) + (.6em, 0)$) {$+$};
	
	\fill[thin, draw = black, /pgfplots/violin color half = arabidopsis, fill = viocolleft, fill opacity = .5, x = .25em, y = .1em, rotate = 90, shift = {(legend symbol)}] plot[domain = 0:6] (\x,{4*1/exp(((\x-3)^2)/2)}) -- cycle;
	\fill[thin, draw = black, /pgfplots/violin color half = arabidopsis, fill = viocolright, fill opacity = .5, x = .25em, y = -.1em, rotate = 90, shift = {(legend symbol)}, yshift = -.1em] plot[domain = 0:6] (\x,{4*1/exp(((\x-3)^2)/2)}) -- cycle;
	
	
	%%% subfigure labels
	\subfiglabel[yshift = .5\columnsep]{enrInrLeaf}
	\subfiglabel[yshift = .5\columnsep]{enrInrProto}
	\subfiglabel[yshift = .5\columnsep]{enrTCTLeaf}
	\subfiglabel[yshift = .5\columnsep]{enrTCTProto}

\end{tikzpicture}%
			\caption{%
				\textbf{Core promoter elements at the TSS influence promoter strength.}\titleend
				\subfigrange{A}{D} Violin plots of promoter strength in tobacco leaves \parensubfig[\subfigref{A}]{C} or maize protoplasts \parensubfig[\subfigref{B}]{D}. Promoters were grouped by GC content and split into promoters without (left half, darker color) or with (right half, lighter color) an Inr \parensubfig[\subfigref{A}]{B}, or TCT \parensubfig[\subfigref{C}]{D} element at the TSS. Violin plots are as defined in \autoref{fig:overview}, except only one half is shown.
			}%
			\label{sfig:Inr}%
		\end{sfig}
		
		\begin{sfig}
			%\tikzset{external/export next = false}

\begin{tikzpicture}
	
	%%% enrichment by TCP binding site (tobacco)
	\coordinate (enrTCPleaf) at (0, 0);
	
	\leafsymbol{enrTCPleaf}; 
	
	\begin{hgroupplot}[%
		ylabel = promoter strength,
		ymin = -9.5,
		ymax = 8,
		ytick = {-10, -8, ..., 10},
		group position = {anchor = above north west, at = {(enrTCPleaf)}, xshift = \plotylabelwidth, yshift = -1.25\baselineskip},
		group/every plot/.append style = {x grids = false}
	]{\twocolumnwidth}{3}{GC content}
	
			
		\nextgroupplot[
			title = Arabidopsis,
			x tick table half = {rawData/enrichment_TCP_At_leaf_boxplot.tsv}{LaTeX.label}
		]
			
		% half violin and box plot
		\halfviolinbox[violin color half = arabidopsis]{rawData/enrichment_TCP_At_leaf_boxplot.tsv}{rawData/enrichment_TCP_At_leaf_violin.tsv}
			
		% add sample size
		\samplesizehalf[violin color half = arabidopsis]{rawData/enrichment_TCP_At_leaf_boxplot.tsv}{id}{n}
		
		% add pvalues
		\signifall*{rawData/enrichment_TCP_At_leaf_pvalues.tsv}	
		
		
		\nextgroupplot[
			title = Maize,
			x tick table half = {rawData/enrichment_TCP_Zm_leaf_boxplot.tsv}{LaTeX.label}
		]
			
		% half violin and box plot
		\halfviolinbox[violin color half = maize]{rawData/enrichment_TCP_Zm_leaf_boxplot.tsv}{rawData/enrichment_TCP_Zm_leaf_violin.tsv}
			
		% add sample size
		\samplesizehalf[violin color half = maize]{rawData/enrichment_TCP_Zm_leaf_boxplot.tsv}{id}{n}
		
		% add pvalues
		\signifall*{rawData/enrichment_TCP_Zm_leaf_pvalues.tsv}
		
		
		\nextgroupplot[
			title = Sorghum,
			x tick table half = {rawData/enrichment_TCP_Sb_leaf_boxplot.tsv}{LaTeX.label}
		]
			
		% half violin and box plot
		\halfviolinbox[violin color half = sorghum]{rawData/enrichment_TCP_Sb_leaf_boxplot.tsv}{rawData/enrichment_TCP_Sb_leaf_violin.tsv}
			
		% add sample size
		\samplesizehalf[violin color half = sorghum]{rawData/enrichment_TCP_Sb_leaf_boxplot.tsv}{id}{n}
		
		% add pvalues
		\signifall*{rawData/enrichment_TCP_Sb_leaf_pvalues.tsv}	

	\end{hgroupplot}
	
	\node[anchor = south west] (legend) at (group c1r1.above north west) {\textbf{TCP} transcripition factor binding site: $-$};
	\coordinate (legend symbol) at ($(legend.south east) + (.25em, .2em)$);
	\node[anchor = base west] at ($(legend.base east) + (.6em, 0)$) {$+$};
	
	\fill[thin, draw = black, /pgfplots/violin color half = arabidopsis, fill = viocolleft, fill opacity = .5, x = .25em, y = .1em, rotate = 90, shift = {(legend symbol)}] plot[domain = 0:6] (\x,{4*1/exp(((\x-3)^2)/2)}) -- cycle;
	\fill[thin, draw = black, /pgfplots/violin color half = arabidopsis, fill = viocolright, fill opacity = .5, x = .25em, y = -.1em, rotate = 90, shift = {(legend symbol)}, yshift = -.1em] plot[domain = 0:6] (\x,{4*1/exp(((\x-3)^2)/2)}) -- cycle;
	
	
	%%% enrichment by TCP binding site (protoplasts)
	\coordinate (enrTCPproto) at (\textwidth - \twocolumnwidth, 0 |- enrTCPleaf);
	
	\protosymbol{enrTCPproto}; 
	
	\begin{hgroupplot}[%
		ylabel = promoter strength,
		ymin = -9.5,
		ymax = 8,
		ytick = {-10, -8, ..., 10},
		group position = {anchor = above north west, at = {(enrTCPproto)}, xshift = \plotylabelwidth, yshift = -1.25\baselineskip},
		group/every plot/.append style = {x grids = false}
	]{\twocolumnwidth}{3}{GC content}
	
			
		\nextgroupplot[
			title = Arabidopsis,
			x tick table half = {rawData/enrichment_TCP_At_proto_boxplot.tsv}{LaTeX.label}
		]
			
		% half violin and box plot
		\halfviolinbox[violin color half = arabidopsis]{rawData/enrichment_TCP_At_proto_boxplot.tsv}{rawData/enrichment_TCP_At_proto_violin.tsv}
			
		% add sample size
		\samplesizehalf[violin color half = arabidopsis]{rawData/enrichment_TCP_At_proto_boxplot.tsv}{id}{n}
		
		% add pvalues
		\signifall*{rawData/enrichment_TCP_At_proto_pvalues.tsv}	
		
		
		\nextgroupplot[
			title = Maize,
			x tick table half = {rawData/enrichment_TCP_Zm_proto_boxplot.tsv}{LaTeX.label}
		]
			
		% half violin and box plot
		\halfviolinbox[violin color half = maize]{rawData/enrichment_TCP_Zm_proto_boxplot.tsv}{rawData/enrichment_TCP_Zm_proto_violin.tsv}
			
		% add sample size
		\samplesizehalf[violin color half = maize]{rawData/enrichment_TCP_Zm_proto_boxplot.tsv}{id}{n}
		
		% add pvalues
		\signifall*{rawData/enrichment_TCP_Zm_proto_pvalues.tsv}
		
		
		\nextgroupplot[
			title = Sorghum,
			x tick table half = {rawData/enrichment_TCP_Sb_proto_boxplot.tsv}{LaTeX.label}
		]
			
		% half violin and box plot
		\halfviolinbox[violin color half = sorghum]{rawData/enrichment_TCP_Sb_proto_boxplot.tsv}{rawData/enrichment_TCP_Sb_proto_violin.tsv}
			
		% add sample size
		\samplesizehalf[violin color half = sorghum]{rawData/enrichment_TCP_Sb_proto_boxplot.tsv}{id}{n}
		
		% add pvalues
		\signifall*{rawData/enrichment_TCP_Sb_proto_pvalues.tsv}	

	\end{hgroupplot}
	
	\node[anchor = south west] (legend) at (group c1r1.above north west) {\textbf{TCP} transcripition factor binding site: $-$};
	\coordinate (legend symbol) at ($(legend.south east) + (.25em, .2em)$);
	\node[anchor = base west] at ($(legend.base east) + (.6em, 0)$) {$+$};
	
	\fill[thin, draw = black, /pgfplots/violin color half = arabidopsis, fill = viocolleft, fill opacity = .5, x = .25em, y = .1em, rotate = 90, shift = {(legend symbol)}] plot[domain = 0:6] (\x,{4*1/exp(((\x-3)^2)/2)}) -- cycle;
	\fill[thin, draw = black, /pgfplots/violin color half = arabidopsis, fill = viocolright, fill opacity = .5, x = .25em, y = -.1em, rotate = 90, shift = {(legend symbol)}, yshift = -.1em] plot[domain = 0:6] (\x,{4*1/exp(((\x-3)^2)/2)}) -- cycle;
	

	%%% enrichment by HSF binding site (tobacco)
	\coordinate[yshift = -\columnsep] (enrHSFleaf) at (enrTCPleaf |- plot xlabel.south);
	
	\leafsymbol{enrHSFleaf}; 
	
	\begin{hgroupplot}[%
		ylabel = promoter strength,
		ymin = -9.5,
		ymax = 8,
		ytick = {-10, -8, ..., 10},
		group position = {anchor = above north west, at = {(enrHSFleaf)}, xshift = \plotylabelwidth, yshift = -1.25\baselineskip},
		group/every plot/.append style = {x grids = false}
	]{\twocolumnwidth}{3}{GC content}
	
			
		\nextgroupplot[
			title = Arabidopsis,
			x tick table half = {rawData/enrichment_HSF_At_leaf_boxplot.tsv}{LaTeX.label}
		]
			
		% half violin and box plot
		\halfviolinbox[violin color half = arabidopsis]{rawData/enrichment_HSF_At_leaf_boxplot.tsv}{rawData/enrichment_HSF_At_leaf_violin.tsv}
			
		% add sample size
		\samplesizehalf[violin color half = arabidopsis]{rawData/enrichment_HSF_At_leaf_boxplot.tsv}{id}{n}
		
		% add pvalues
		\signifall*{rawData/enrichment_HSF_At_leaf_pvalues.tsv}	
		
		
		\nextgroupplot[
			title = Maize,
			x tick table half = {rawData/enrichment_HSF_Zm_leaf_boxplot.tsv}{LaTeX.label}
		]
			
		% half violin and box plot
		\halfviolinbox[violin color half = maize]{rawData/enrichment_HSF_Zm_leaf_boxplot.tsv}{rawData/enrichment_HSF_Zm_leaf_violin.tsv}
			
		% add sample size
		\samplesizehalf[violin color half = maize]{rawData/enrichment_HSF_Zm_leaf_boxplot.tsv}{id}{n}
		
		% add pvalues
		\signifall*{rawData/enrichment_HSF_Zm_leaf_pvalues.tsv}
		
		
		\nextgroupplot[
			title = Sorghum,
			x tick table half = {rawData/enrichment_HSF_Sb_leaf_boxplot.tsv}{LaTeX.label}
		]
			
		% half violin and box plot
		\halfviolinbox[violin color half = sorghum]{rawData/enrichment_HSF_Sb_leaf_boxplot.tsv}{rawData/enrichment_HSF_Sb_leaf_violin.tsv}
			
		% add sample size
		\samplesizehalf[violin color half = sorghum]{rawData/enrichment_HSF_Sb_leaf_boxplot.tsv}{id}{n}
		
		% add pvalues
		\signifall*{rawData/enrichment_HSF_Sb_leaf_pvalues.tsv}	

	\end{hgroupplot}
	
	\node[anchor = south west] (legend) at (group c1r1.above north west) {\textbf{HSF/S1Fa-like} transcripition factor binding site: $-$};
	\coordinate (legend symbol) at ($(legend.south east) + (.25em, .2em)$);
	\node[anchor = base west] at ($(legend.base east) + (.6em, 0)$) {$+$};
	
	\fill[thin, draw = black, /pgfplots/violin color half = arabidopsis, fill = viocolleft, fill opacity = .5, x = .25em, y = .1em, rotate = 90, shift = {(legend symbol)}] plot[domain = 0:6] (\x,{4*1/exp(((\x-3)^2)/2)}) -- cycle;
	\fill[thin, draw = black, /pgfplots/violin color half = arabidopsis, fill = viocolright, fill opacity = .5, x = .25em, y = -.1em, rotate = 90, shift = {(legend symbol)}, yshift = -.1em] plot[domain = 0:6] (\x,{4*1/exp(((\x-3)^2)/2)}) -- cycle;
	
	
	%%% enrichment by HSF binding site (protoplasts)
	\coordinate (enrHSFproto) at (\textwidth - \twocolumnwidth, 0 |- enrHSFleaf);
	
	\protosymbol{enrHSFproto}; 
	
	\begin{hgroupplot}[%
		ylabel = promoter strength,
		ymin = -9.5,
		ymax = 8,
		ytick = {-10, -8, ..., 10},
		group position = {anchor = above north west, at = {(enrHSFproto)}, xshift = \plotylabelwidth, yshift = -1.25\baselineskip},
		group/every plot/.append style = {x grids = false}
	]{\twocolumnwidth}{3}{GC content}
	
			
		\nextgroupplot[
			title = Arabidopsis,
			x tick table half = {rawData/enrichment_HSF_At_proto_boxplot.tsv}{LaTeX.label}
		]
			
		% half violin and box plot
		\halfviolinbox[violin color half = arabidopsis]{rawData/enrichment_HSF_At_proto_boxplot.tsv}{rawData/enrichment_HSF_At_proto_violin.tsv}
			
		% add sample size
		\samplesizehalf[violin color half = arabidopsis]{rawData/enrichment_HSF_At_proto_boxplot.tsv}{id}{n}
		
		% add pvalues
		\signifall*{rawData/enrichment_HSF_At_proto_pvalues.tsv}	
		
		
		\nextgroupplot[
			title = Maize,
			x tick table half = {rawData/enrichment_HSF_Zm_proto_boxplot.tsv}{LaTeX.label}
		]
			
		% half violin and box plot
		\halfviolinbox[violin color half = maize]{rawData/enrichment_HSF_Zm_proto_boxplot.tsv}{rawData/enrichment_HSF_Zm_proto_violin.tsv}
			
		% add sample size
		\samplesizehalf[violin color half = maize]{rawData/enrichment_HSF_Zm_proto_boxplot.tsv}{id}{n}
		
		% add pvalues
		\signifall*{rawData/enrichment_HSF_Zm_proto_pvalues.tsv}
		
		
		\nextgroupplot[
			title = Sorghum,
			x tick table half = {rawData/enrichment_HSF_Sb_proto_boxplot.tsv}{LaTeX.label}
		]
			
		% half violin and box plot
		\halfviolinbox[violin color half = sorghum]{rawData/enrichment_HSF_Sb_proto_boxplot.tsv}{rawData/enrichment_HSF_Sb_proto_violin.tsv}
			
		% add sample size
		\samplesizehalf[violin color half = sorghum]{rawData/enrichment_HSF_Sb_proto_boxplot.tsv}{id}{n}
		
		% add pvalues
		\signifall*{rawData/enrichment_HSF_Sb_proto_pvalues.tsv}	

	\end{hgroupplot}
	
	\node[anchor = south west] (legend) at (group c1r1.above north west) {\textbf{HSF/S1Fa-like} transcripition factor binding site: $-$};
	\coordinate (legend symbol) at ($(legend.south east) + (.25em, .2em)$);
	\node[anchor = base west] at ($(legend.base east) + (.6em, 0)$) {$+$};
	
	\fill[thin, draw = black, /pgfplots/violin color half = arabidopsis, fill = viocolleft, fill opacity = .5, x = .25em, y = .1em, rotate = 90, shift = {(legend symbol)}] plot[domain = 0:6] (\x,{4*1/exp(((\x-3)^2)/2)}) -- cycle;
	\fill[thin, draw = black, /pgfplots/violin color half = arabidopsis, fill = viocolright, fill opacity = .5, x = .25em, y = -.1em, rotate = 90, shift = {(legend symbol)}, yshift = -.1em] plot[domain = 0:6] (\x,{4*1/exp(((\x-3)^2)/2)}) -- cycle;
	
	
	%%% subfigure labels
	\subfiglabel[yshift = .5\columnsep]{enrTCPleaf}
	\subfiglabel[yshift = .5\columnsep]{enrTCPproto}
	\subfiglabel[yshift = .5\columnsep]{enrHSFleaf}
	\subfiglabel[yshift = .5\columnsep]{enrHSFproto}

\end{tikzpicture}%
			\caption{%
				\textbf{Transcription factor binding sites contribute to promoter strength in an assay system-dependent manner.}\titleend
				\subfigrange{A}{D} Violin plots of promoter strength for libraries without enhancer in tobacco leaves \parensubfig[\subfigref{A}]{C} or maize protoplasts \parensubfig[\subfigref{B}]{D}. Promoters were grouped by GC content and split into promoters without (left half, darker color) or with (right half, lighter color) a binding site for TCP \parensubfig[\subfigref{A}]{B} or HSF \parensubfig[\subfigref{C}]{D} transcription factors. Violin plots are as defined in \autoref{fig:overview}, except only one half is shown.
			}%
			\label{fig:TFs}
		\end{sfig}
		
		\begin{sfig}
			%\tikzset{external/export next = false}

\begin{tikzpicture}

	\begin{pgfinterruptboundingbox} % somehow these plots add about 25pt of whitespace to their right ... no idea why

		%%% TCP position
		\coordinate (TCPpos) at (0, 0);
		
		\begin{vgroupplot}[%
			width = \twocolumnwidth - \plotylabelwidth,
			xlabel = {TCP binding site position (rel. to TSS)},
			xlabel style = {name = plot xlabel},
			xmin = 0,
			xmax = 170,
			ymin = 0,
			ymax = 115,
			xtick = {6, 26, ..., 166},
			xticklabel = {$\pgfmathparse{\tick < 166 ? \tick - 166 : \tick -165}\pgfmathprintnumber[print sign]{\pgfmathresult}$},
			ytick = {0, 20, ..., 140}, 
			group position = {at = {(TCPpos)}, anchor = north west, xshift = \plotylabelwidth},
			ybar = 0pt,
			legend pos = north west,
			legend columns = 3
		]{\twocolumnwidth}{3}{promoters}
		
				\nextgroupplot[
					bar width = 1
				]
	
				\addplot [arabidopsis, fill, fill opacity = .5] table [x = pos, y = At] {rawData/TCP_position.tsv};
				
				\addlegendimage{maize, fill, fill opacity = .5}
				\addlegendimage{sorghum, fill, fill opacity = .5}
				
				\legend{Arabidopsis\phantom{A}, Maize\phantom{A}, Sorghum}
				
				
				\nextgroupplot[
					bar width = 1
				]
				
				\addplot [maize, fill, fill opacity = .5] table [x = pos, y = Zm] {rawData/TCP_position.tsv};
				
				
				\nextgroupplot[
					bar width = 1
				]
				
				\addplot [sorghum, fill, fill opacity = .5] table [x = pos, y = Sb] {rawData/TCP_position.tsv};
		
		\end{vgroupplot}
		
		
		%%% HSF position
		\coordinate (HSFpos) at (TCPpos -| \textwidth - \twocolumnwidth, 0);
		
		\begin{vgroupplot}[%
			width = \twocolumnwidth - \plotylabelwidth,
			xlabel = {HSF binding site position (rel. to TSS)},
			xlabel style = {name = plot xlabel},
			xmin = 0,
			xmax = 170,
			ymin = 0,
			ymax = 13,
			xtick = {6, 26, ..., 166},
			xticklabel = {$\pgfmathparse{\tick < 166 ? \tick - 166 : \tick -165}\pgfmathprintnumber[print sign]{\pgfmathresult}$},
			ytick = {0, 2, ..., 14}, 
			group position = {at = {(HSFpos)}, anchor = north west, xshift = \plotylabelwidth},
			ybar = 0pt,
			legend pos = north west,
			legend columns = 3
		]{\twocolumnwidth}{3}{promoters}
		
				\nextgroupplot[
					bar width = 1
				]
	
				\addplot [arabidopsis, fill, fill opacity = .5] table [x = pos, y = At] {rawData/HSF_position.tsv};
				
				\addlegendimage{maize, fill, fill opacity = .5}
				\addlegendimage{sorghum, fill, fill opacity = .5}
				
				\legend{Arabidopsis\phantom{A}, Maize\phantom{A}, Sorghum}
				
				
				\nextgroupplot[
					bar width = 1
				]
				
				\addplot [maize, fill, fill opacity = .5] table [x = pos, y = Zm] {rawData/HSF_position.tsv};
				
				
				\nextgroupplot[
					bar width = 1
				]
				
				\addplot [sorghum, fill, fill opacity = .5] table [x = pos, y = Sb] {rawData/HSF_position.tsv};
		
		\end{vgroupplot}
		
		
		%%% enrichment by TCP position (tobacco)
		\coordinate[yshift = -\columnsep] (enrTCPposLeaf) at (TCPpos |- plot xlabel.south);
		
		\leafsymbol{enrTCPposLeaf};
		
		\begin{hgroupplot}[%
			ylabel = promoter strength,
			ymin = -8,
			ymax = 7,
			ytick = {-10, -8, ..., 10},
			group position = {anchor = above north west, at = {(enrTCPposLeaf)}, xshift = \plotylabelwidth},
			group/every plot/.append style = {x grids = false}
		]{\twocolumnwidth}{3}{TCP binding site position (rel. to TATA-box)}
	
				
			\nextgroupplot[
				title = Arabidopsis,
				x tick table = {rawData/enrichment_TCPpos_At_leaf_boxplot.tsv}{LaTeX.label}
			]
				
			% violin and box plot
			\violinbox[violin color = arabidopsis]{rawData/enrichment_TCPpos_At_leaf_boxplot.tsv}{rawData/enrichment_TCPpos_At_leaf_violin.tsv}
				
			% add sample size
			\samplesize{rawData/enrichment_TCPpos_At_leaf_boxplot.tsv}{id}{n}
			
			% add significance level
			\signif*{rawData/enrichment_TCPpos_At_leaf_pvalues.tsv}{1}{2}
			
			
			\nextgroupplot[
				title = Maize,
				x tick table = {rawData/enrichment_TCPpos_Zm_leaf_boxplot.tsv}{LaTeX.label}
			]
			
			% violin and box plot
			\violinbox[violin color = maize]{rawData/enrichment_TCPpos_Zm_leaf_boxplot.tsv}{rawData/enrichment_TCPpos_Zm_leaf_violin.tsv}
				
			% add sample size
			\samplesize{rawData/enrichment_TCPpos_Zm_leaf_boxplot.tsv}{id}{n}
			
			% add significance level
			\signif*{rawData/enrichment_TCPpos_Zm_leaf_pvalues.tsv}{1}{2}
			
			
			\nextgroupplot[
				title = Sorghum,
				x tick table = {rawData/enrichment_TCPpos_Sb_leaf_boxplot.tsv}{LaTeX.label}
			]
				
			% violin and box plot
			\violinbox[violin color = sorghum]{rawData/enrichment_TCPpos_Sb_leaf_boxplot.tsv}{rawData/enrichment_TCPpos_Sb_leaf_violin.tsv}
			
			% add sample size
			\samplesize{rawData/enrichment_TCPpos_Sb_leaf_boxplot.tsv}{id}{n}
			
			% add significance level
			\signif*{rawData/enrichment_TCPpos_Sb_leaf_pvalues.tsv}{1}{2}
				
		\end{hgroupplot}
		
		
		%%% enrichment by HSF position (protoplasts)
	
		\coordinate (enrHSFposLeaf) at (HSFpos |- enrTCPposLeaf);
		
		\protosymbol{enrHSFposLeaf};
		
		\begin{hgroupplot}[%
			ylabel = promoter strength,
			ymin = -8,
			ymax = 7,
			ytick = {-10, -8, ..., 10},
			group position = {anchor = above north west, at = {(enrHSFposLeaf)}, xshift = \plotylabelwidth},
			group/every plot/.append style = {x grids = false}
		]{\twocolumnwidth}{3}{HSF binding site position (rel. to TATA-box)}
	
				
			\nextgroupplot[
				title = Arabidopsis,
				x tick table = {rawData/enrichment_HSFpos_At_proto_boxplot.tsv}{LaTeX.label}
			]
				
			% violin and box plot
			\violinbox[violin color = arabidopsis]{rawData/enrichment_HSFpos_At_proto_boxplot.tsv}{rawData/enrichment_HSFpos_At_proto_violin.tsv}
				
			% add sample size
			\samplesize{rawData/enrichment_HSFpos_At_proto_boxplot.tsv}{id}{n}
			
			% add significance level
			\signif*{rawData/enrichment_HSFpos_At_proto_pvalues.tsv}{1}{2}
			
			
			\nextgroupplot[
				title = Maize,
				x tick table = {rawData/enrichment_HSFpos_Zm_proto_boxplot.tsv}{LaTeX.label}
			]
			
			% violin and box plot
			\violinbox[violin color = maize]{rawData/enrichment_HSFpos_Zm_proto_boxplot.tsv}{rawData/enrichment_HSFpos_Zm_proto_violin.tsv}
				
			% add sample size
			\samplesize{rawData/enrichment_HSFpos_Zm_proto_boxplot.tsv}{id}{n}
			
			% add significance level
			\signif*{rawData/enrichment_HSFpos_Zm_proto_pvalues.tsv}{1}{2}
			
			
			\nextgroupplot[
				title = Sorghum,
				x tick table = {rawData/enrichment_HSFpos_Sb_proto_boxplot.tsv}{LaTeX.label}
			]
				
			% violin and box plot
			\violinbox[violin color = sorghum]{rawData/enrichment_HSFpos_Sb_proto_boxplot.tsv}{rawData/enrichment_HSFpos_Sb_proto_violin.tsv}
			
			% add sample size
			\samplesize{rawData/enrichment_HSFpos_Sb_proto_boxplot.tsv}{id}{n}
			
			% add significance level
			\signif*{rawData/enrichment_HSFpos_Sb_proto_pvalues.tsv}{1}{2}
				
		\end{hgroupplot}
		
		
		%%% subfigure labels
		\subfiglabel[yshift = .5\columnsep]{TCPpos}
		\subfiglabel[yshift = .5\columnsep]{HSFpos}
		\subfiglabel[yshift = .5\columnsep]{enrTCPposLeaf}
		\subfiglabel[yshift = .5\columnsep]{enrHSFposLeaf}
	
	\end{pgfinterruptboundingbox}
	
	% manually add the correct bounding box
	\path (TCPpos) ++(.2pt, .5\columnsep) rectangle (\textwidth + .2pt, 0 |- plot xlabel.south);

\end{tikzpicture}%
			\caption{%
				\textbf{Transcription factor binding sites are more active upstream of the TATA-box.}\titleend
				\subfigref{A}\subfigref{B} Histograms showing the number of promoters with a TCP \parensubfig{A} or HSF \parensubfig{B} transcription factor binding site at the indicated position.\nextentry
				\subfigrange{C}{F} Violin plots (as defined in \autoref{fig:overview}) of promoter strength for libraries without enhancer in tobacco leaves \parensubfig[\subfigref{C}]{E} or maize protoplasts \parensubfig[\subfigref{D}]{F}. Promoters were grouped by the position of their TCP \parensubfig[\subfigref{C}]{D}, or HSF \parensubfig[\subfigref{E}]{F} transcription factor binding site relative to the TATA-box.
			}%
			\label{sfig:TFpos}%
		\end{sfig}
		
		\begin{sfig}
			%\tikzset{external/export next = false}

\begin{tikzpicture}

	%%% enhancer responsiveness by type (leaf)
	\coordinate (enhTypeLeaf) at (0, 0);
	
	\leafsymbol{enhTypeLeaf};
	
	\begin{hgroupplot}[%
		ylabel = enhancer responsiveness,
		ymin = -4,
		ymax = 11,
		ytick = {-10, -8, ..., 10},
		group position = {anchor = above north west, at = {(enhTypeLeaf)}, xshift = \plotylabelwidth},
		group/every plot/.append style = {x grids = false},
		zero line,
	]{\twocolumnwidth}{3}{promoter type}

		\nextgroupplot[
			title = Arabidopsis,
			x tick table = {rawData/enh-effect_type_At_leaf_boxplot.tsv}{LaTeX.label}
		]
			
		% violin and box plot
		\violinbox[violin color = arabidopsis]{rawData/enh-effect_type_At_leaf_boxplot.tsv}{rawData/enh-effect_type_At_leaf_violin.tsv}
			
		% add sample size
		\samplesize{rawData/enh-effect_type_At_leaf_boxplot.tsv}{id}{n}
		
		% add significance level
		\signif*{rawData/enh-effect_type_At_leaf_pvalues.tsv}{1}{2}
		
		
		\nextgroupplot[
			title = Maize,
			x tick table = {rawData/enh-effect_type_Zm_leaf_boxplot.tsv}{LaTeX.label}
		]
		
		% violin and box plot
		\violinbox[violin color = maize]{rawData/enh-effect_type_Zm_leaf_boxplot.tsv}{rawData/enh-effect_type_Zm_leaf_violin.tsv}
			
		% add sample size
		\samplesize{rawData/enh-effect_type_Zm_leaf_boxplot.tsv}{id}{n}
		
		% add significance level
		\signif*{rawData/enh-effect_type_Zm_leaf_pvalues.tsv}{1}{2}
		
		
		\nextgroupplot[
			title = Sorghum,
			x tick table = {rawData/enh-effect_type_Sb_leaf_boxplot.tsv}{LaTeX.label}
		]
			
		% violin and box plot
		\violinbox[violin color = sorghum]{rawData/enh-effect_type_Sb_leaf_boxplot.tsv}{rawData/enh-effect_type_Sb_leaf_violin.tsv}
		
		% add sample size
		\samplesize{rawData/enh-effect_type_Sb_leaf_boxplot.tsv}{id}{n}
		
		% add significance level
		\signif*{rawData/enh-effect_type_Sb_leaf_pvalues.tsv}{1}{2}
			
	\end{hgroupplot}
	
	
	%%% enhancer responsiveness by type (proto)
	\coordinate (enhTypeProto) at (enhTypeLeaf -| \textwidth - \twocolumnwidth, 0);
	
	\protosymbol{enhTypeProto};
	
	\begin{hgroupplot}[%
		ylabel = enhancer responsiveness,
		ymin = -4,
		ymax = 9.25,
		ytick = {-10, -8, ..., 10},
		group position = {anchor = above north west, at = {(enhTypeProto)}, xshift = \plotylabelwidth},
		group/every plot/.append style = {x grids = false},
		zero line,
	]{\twocolumnwidth}{3}{promoter type}

		\nextgroupplot[
			title = Arabidopsis,
			x tick table = {rawData/enh-effect_type_At_proto_boxplot.tsv}{LaTeX.label}
		]
			
		% violin and box plot
		\violinbox[violin color = arabidopsis]{rawData/enh-effect_type_At_proto_boxplot.tsv}{rawData/enh-effect_type_At_proto_violin.tsv}
			
		% add sample size
		\samplesize{rawData/enh-effect_type_At_proto_boxplot.tsv}{id}{n}
		
		% add significance level
		\signif*{rawData/enh-effect_type_At_proto_pvalues.tsv}{1}{2}
		
		
		\nextgroupplot[
			title = Maize,
			x tick table = {rawData/enh-effect_type_Zm_proto_boxplot.tsv}{LaTeX.label}
		]
		
		% violin and box plot
		\violinbox[violin color = maize]{rawData/enh-effect_type_Zm_proto_boxplot.tsv}{rawData/enh-effect_type_Zm_proto_violin.tsv}
			
		% add sample size
		\samplesize{rawData/enh-effect_type_Zm_proto_boxplot.tsv}{id}{n}
		
		% add significance level
		\signif*{rawData/enh-effect_type_Zm_proto_pvalues.tsv}{1}{2}
		
		
		\nextgroupplot[
			title = Sorghum,
			x tick table = {rawData/enh-effect_type_Sb_proto_boxplot.tsv}{LaTeX.label}
		]
			
		% violin and box plot
		\violinbox[violin color = sorghum]{rawData/enh-effect_type_Sb_proto_boxplot.tsv}{rawData/enh-effect_type_Sb_proto_violin.tsv}
		
		% add sample size
		\samplesize{rawData/enh-effect_type_Sb_proto_boxplot.tsv}{id}{n}
		
		% add significance level
		\signif*{rawData/enh-effect_type_Sb_proto_pvalues.tsv}{1}{2}
			
	\end{hgroupplot}
	
	
	%%% subfigure labels
	\subfiglabel[yshift = .5\columnsep]{enhTypeLeaf}
	\subfiglabel[yshift = .5\columnsep]{enhTypeProto}
\end{tikzpicture}%
			\caption{%
				\textbf{Promoters of miRNA genes are more responsive to the 35S enhancer than those associated with protein-coding genes.}\titleend
				\subfigref{A}\subfigref{B} Violin plots (as defined in \autoref{fig:overview}) of enhancer responsiveness (promoter strength\textsuperscript{with enhancer} $-$ promoter strength\textsuperscript{without enhancer}) in tobacco leaves \parensubfig{A} or maize protoplasts \parensubfig{B}. Promoters associated with miRNA or protein-coding genes are compared.
			}%
			\label{sfig:enhEffectSpecificity}%
		\end{sfig}
		
		\begin{sfig}
			%\tikzset{external/export next = false}

\begin{tikzpicture}

	%%% enh-effect by TCP (tobacco)
	\coordinate (enhTCPLeaf) at (0, 0);
	
	\leafsymbol{enhTCPLeaf}; 
	
	\begin{hgroupplot}[%
		ylabel = enhancer responsiveness,
		ymin = -3.75,
		ymax = 10.5,
		ytick = {-10, -8, ..., 10},
		group position = {anchor = above north west, at = {(enhTCPLeaf)}, xshift = \plotylabelwidth, yshift = -1.25\baselineskip},
		group/every plot/.append style = {x grids = false},
		zero line,
	]{\twocolumnwidth}{3}{GC content}
	
			
		\nextgroupplot[
			title = Arabidopsis,
			x tick table half = {rawData/enh-effect_TCP_At_leaf_boxplot.tsv}{LaTeX.label}
		]
			
		% half violin and box plot
		\halfviolinbox[violin color half = arabidopsis]{rawData/enh-effect_TCP_At_leaf_boxplot.tsv}{rawData/enh-effect_TCP_At_leaf_violin.tsv}
			
		% add sample size
		\samplesizehalf[violin color half = arabidopsis]{rawData/enh-effect_TCP_At_leaf_boxplot.tsv}{id}{n}
		
		% add pvalues
		\signifall*{rawData/enh-effect_TCP_At_leaf_pvalues.tsv}	
		
		
		\nextgroupplot[
			title = Maize,
			x tick table half = {rawData/enh-effect_TCP_Zm_leaf_boxplot.tsv}{LaTeX.label}
		]
			
		% half violin and box plot
		\halfviolinbox[violin color half = maize]{rawData/enh-effect_TCP_Zm_leaf_boxplot.tsv}{rawData/enh-effect_TCP_Zm_leaf_violin.tsv}
			
		% add sample size
		\samplesizehalf[violin color half = maize]{rawData/enh-effect_TCP_Zm_leaf_boxplot.tsv}{id}{n}
		
		% add pvalues
		\signifall*{rawData/enh-effect_TCP_Zm_leaf_pvalues.tsv}
		
		
		\nextgroupplot[
			title = Sorghum,
			x tick table half = {rawData/enh-effect_TCP_Sb_leaf_boxplot.tsv}{LaTeX.label}
		]
			
		% half violin and box plot
		\halfviolinbox[violin color half = sorghum]{rawData/enh-effect_TCP_Sb_leaf_boxplot.tsv}{rawData/enh-effect_TCP_Sb_leaf_violin.tsv}
			
		% add sample size
		\samplesizehalf[violin color half = sorghum]{rawData/enh-effect_TCP_Sb_leaf_boxplot.tsv}{id}{n}
		
		% add pvalues
		\signifall*{rawData/enh-effect_TCP_Sb_leaf_pvalues.tsv}	

	\end{hgroupplot}
	
	\node[anchor = south west] (legend) at (group c1r1.above north west) {\textbf{TCP} transcripition factor binding site: $-$};
	\coordinate (legend symbol) at ($(legend.south east) + (.25em, .2em)$);
	\node[anchor = base west] at ($(legend.base east) + (.6em, 0)$) {$+$};
	
	\fill[thin, draw = black, /pgfplots/violin color half = arabidopsis, fill = viocolleft, fill opacity = .5, x = .25em, y = .1em, rotate = 90, shift = {(legend symbol)}] plot[domain = 0:6] (\x,{4*1/exp(((\x-3)^2)/2)}) -- cycle;
	\fill[thin, draw = black, /pgfplots/violin color half = arabidopsis, fill = viocolright, fill opacity = .5, x = .25em, y = -.1em, rotate = 90, shift = {(legend symbol)}, yshift = -.1em] plot[domain = 0:6] (\x,{4*1/exp(((\x-3)^2)/2)}) -- cycle;
	
	
	%%% enh-effect by TCP (protoplasts)
	\coordinate (enhTCPProto) at (\textwidth - \twocolumnwidth, 0 |- enhTCPLeaf);
	
	\protosymbol{enhTCPProto}; 
	
	\begin{hgroupplot}[%
		ylabel = enhancer responsiveness,
		ymin = -3.75,
		ymax = 10.5,
		ytick = {-10, -8, ..., 10},
		group position = {anchor = above north west, at = {(enhTCPProto)}, xshift = \plotylabelwidth, yshift = -1.25\baselineskip},
		group/every plot/.append style = {x grids = false},
		zero line,
	]{\twocolumnwidth}{3}{GC content}
	
			
		\nextgroupplot[
			title = Arabidopsis,
			x tick table half = {rawData/enh-effect_TCP_At_proto_boxplot.tsv}{LaTeX.label}
		]
			
		% half violin and box plot
		\halfviolinbox[violin color half = arabidopsis]{rawData/enh-effect_TCP_At_proto_boxplot.tsv}{rawData/enh-effect_TCP_At_proto_violin.tsv}
			
		% add sample size
		\samplesizehalf[violin color half = arabidopsis]{rawData/enh-effect_TCP_At_proto_boxplot.tsv}{id}{n}
		
		% add pvalues
		\signifall*{rawData/enh-effect_TCP_At_proto_pvalues.tsv}	
		
		
		\nextgroupplot[
			title = Maize,
			x tick table half = {rawData/enh-effect_TCP_Zm_proto_boxplot.tsv}{LaTeX.label}
		]
			
		% half violin and box plot
		\halfviolinbox[violin color half = maize]{rawData/enh-effect_TCP_Zm_proto_boxplot.tsv}{rawData/enh-effect_TCP_Zm_proto_violin.tsv}
			
		% add sample size
		\samplesizehalf[violin color half = maize]{rawData/enh-effect_TCP_Zm_proto_boxplot.tsv}{id}{n}
		
		% add pvalues
		\signifall*{rawData/enh-effect_TCP_Zm_proto_pvalues.tsv}
		
		
		\nextgroupplot[
			title = Sorghum,
			x tick table half = {rawData/enh-effect_TCP_Sb_proto_boxplot.tsv}{LaTeX.label}
		]
			
		% half violin and box plot
		\halfviolinbox[violin color half = sorghum]{rawData/enh-effect_TCP_Sb_proto_boxplot.tsv}{rawData/enh-effect_TCP_Sb_proto_violin.tsv}
			
		% add sample size
		\samplesizehalf[violin color half = sorghum]{rawData/enh-effect_TCP_Sb_proto_boxplot.tsv}{id}{n}
		
		% add pvalues
		\signifall*{rawData/enh-effect_TCP_Sb_proto_pvalues.tsv}	

	\end{hgroupplot}
	
	\node[anchor = south west] (legend) at (group c1r1.above north west) {\textbf{TCP} transcripition factor binding site: $-$};
	\coordinate (legend symbol) at ($(legend.south east) + (.25em, .2em)$);
	\node[anchor = base west] at ($(legend.base east) + (.6em, 0)$) {$+$};
	
	\fill[thin, draw = black, /pgfplots/violin color half = arabidopsis, fill = viocolleft, fill opacity = .5, x = .25em, y = .1em, rotate = 90, shift = {(legend symbol)}] plot[domain = 0:6] (\x,{4*1/exp(((\x-3)^2)/2)}) -- cycle;
	\fill[thin, draw = black, /pgfplots/violin color half = arabidopsis, fill = viocolright, fill opacity = .5, x = .25em, y = -.1em, rotate = 90, shift = {(legend symbol)}, yshift = -.1em] plot[domain = 0:6] (\x,{4*1/exp(((\x-3)^2)/2)}) -- cycle;
	

	%%% enh-effect by WRKY (tobacco)
	\coordinate[yshift = -\columnsep] (enhWRKYLeaf) at (enhTCPLeaf |- plot xlabel.south);
	
	\leafsymbol{enhWRKYLeaf}; 
	
	\begin{hgroupplot}[%
		ylabel = enhancer responsiveness,
		ymin = -3.75,
		ymax = 10.5,
		ytick = {-10, -8, ..., 10},
		group position = {anchor = above north west, at = {(enhWRKYLeaf)}, xshift = \plotylabelwidth, yshift = -1.25\baselineskip},
		group/every plot/.append style = {x grids = false},
		zero line,
	]{\twocolumnwidth}{3}{GC content}
	
			
		\nextgroupplot[
			title = Arabidopsis,
			x tick table half = {rawData/enh-effect_WRKY_At_leaf_boxplot.tsv}{LaTeX.label}
		]
			
		% half violin and box plot
		\halfviolinbox[violin color half = arabidopsis]{rawData/enh-effect_WRKY_At_leaf_boxplot.tsv}{rawData/enh-effect_WRKY_At_leaf_violin.tsv}
			
		% add sample size
		\samplesizehalf[violin color half = arabidopsis]{rawData/enh-effect_WRKY_At_leaf_boxplot.tsv}{id}{n}
		
		% add pvalues
		\signifall*{rawData/enh-effect_WRKY_At_leaf_pvalues.tsv}	
		
		
		\nextgroupplot[
			title = Maize,
			x tick table half = {rawData/enh-effect_WRKY_Zm_leaf_boxplot.tsv}{LaTeX.label}
		]
			
		% half violin and box plot
		\halfviolinbox[violin color half = maize]{rawData/enh-effect_WRKY_Zm_leaf_boxplot.tsv}{rawData/enh-effect_WRKY_Zm_leaf_violin.tsv}
			
		% add sample size
		\samplesizehalf[violin color half = maize]{rawData/enh-effect_WRKY_Zm_leaf_boxplot.tsv}{id}{n}
		
		% add pvalues
		\signifall*{rawData/enh-effect_WRKY_Zm_leaf_pvalues.tsv}
		
		
		\nextgroupplot[
			title = Sorghum,
			x tick table half = {rawData/enh-effect_WRKY_Sb_leaf_boxplot.tsv}{LaTeX.label}
		]
			
		% half violin and box plot
		\halfviolinbox[violin color half = sorghum]{rawData/enh-effect_WRKY_Sb_leaf_boxplot.tsv}{rawData/enh-effect_WRKY_Sb_leaf_violin.tsv}
			
		% add sample size
		\samplesizehalf[violin color half = sorghum]{rawData/enh-effect_WRKY_Sb_leaf_boxplot.tsv}{id}{n}
		
		% add pvalues
		\signifall*{rawData/enh-effect_WRKY_Sb_leaf_pvalues.tsv}	

	\end{hgroupplot}
	
	\node[anchor = south west] (legend) at (group c1r1.above north west) {\textbf{WRKY/C3H} transcripition factor binding site: $-$};
	\coordinate (legend symbol) at ($(legend.south east) + (.25em, .2em)$);
	\node[anchor = base west] at ($(legend.base east) + (.6em, 0)$) {$+$};
	
	\fill[thin, draw = black, /pgfplots/violin color half = arabidopsis, fill = viocolleft, fill opacity = .5, x = .25em, y = .1em, rotate = 90, shift = {(legend symbol)}] plot[domain = 0:6] (\x,{4*1/exp(((\x-3)^2)/2)}) -- cycle;
	\fill[thin, draw = black, /pgfplots/violin color half = arabidopsis, fill = viocolright, fill opacity = .5, x = .25em, y = -.1em, rotate = 90, shift = {(legend symbol)}, yshift = -.1em] plot[domain = 0:6] (\x,{4*1/exp(((\x-3)^2)/2)}) -- cycle;
	
	
	%%% enh-effect by WRKY binding site (protoplasts)
	\coordinate (enhWRKYProto) at (\textwidth - \twocolumnwidth, 0 |- enhWRKYLeaf);
	
	\protosymbol{enhWRKYProto}; 
	
	\begin{hgroupplot}[%
		ylabel = enhancer responsiveness,
		ymin = -3.75,
		ymax = 10.5,
		ytick = {-10, -8, ..., 10},
		group position = {anchor = above north west, at = {(enhWRKYProto)}, xshift = \plotylabelwidth, yshift = -1.25\baselineskip},
		group/every plot/.append style = {x grids = false},
		zero line,
	]{\twocolumnwidth}{3}{GC content}
	
			
		\nextgroupplot[
			title = Arabidopsis,
			x tick table half = {rawData/enh-effect_WRKY_At_proto_boxplot.tsv}{LaTeX.label}
		]
			
		% half violin and box plot
		\halfviolinbox[violin color half = arabidopsis]{rawData/enh-effect_WRKY_At_proto_boxplot.tsv}{rawData/enh-effect_WRKY_At_proto_violin.tsv}
			
		% add sample size
		\samplesizehalf[violin color half = arabidopsis]{rawData/enh-effect_WRKY_At_proto_boxplot.tsv}{id}{n}
		
		% add pvalues
		\signifall*{rawData/enh-effect_WRKY_At_proto_pvalues.tsv}	
		
		
		\nextgroupplot[
			title = Maize,
			x tick table half = {rawData/enh-effect_WRKY_Zm_proto_boxplot.tsv}{LaTeX.label}
		]
			
		% half violin and box plot
		\halfviolinbox[violin color half = maize]{rawData/enh-effect_WRKY_Zm_proto_boxplot.tsv}{rawData/enh-effect_WRKY_Zm_proto_violin.tsv}
			
		% add sample size
		\samplesizehalf[violin color half = maize]{rawData/enh-effect_WRKY_Zm_proto_boxplot.tsv}{id}{n}
		
		% add pvalues
		\signifall*{rawData/enh-effect_WRKY_Zm_proto_pvalues.tsv}
		
		
		\nextgroupplot[
			title = Sorghum,
			x tick table half = {rawData/enh-effect_WRKY_Sb_proto_boxplot.tsv}{LaTeX.label}
		]
			
		% half violin and box plot
		\halfviolinbox[violin color half = sorghum]{rawData/enh-effect_WRKY_Sb_proto_boxplot.tsv}{rawData/enh-effect_WRKY_Sb_proto_violin.tsv}
			
		% add sample size
		\samplesizehalf[violin color half = sorghum]{rawData/enh-effect_WRKY_Sb_proto_boxplot.tsv}{id}{n}
		
		% add pvalues
		\signifall*{rawData/enh-effect_WRKY_Sb_proto_pvalues.tsv}	

	\end{hgroupplot}
	
	\node[anchor = south west] (legend) at (group c1r1.above north west) {\textbf{WRKY/C3H} transcripition factor binding site: $-$};
	\coordinate (legend symbol) at ($(legend.south east) + (.25em, .2em)$);
	\node[anchor = base west] at ($(legend.base east) + (.6em, 0)$) {$+$};
	
	\fill[thin, draw = black, /pgfplots/violin color half = arabidopsis, fill = viocolleft, fill opacity = .5, x = .25em, y = .1em, rotate = 90, shift = {(legend symbol)}] plot[domain = 0:6] (\x,{4*1/exp(((\x-3)^2)/2)}) -- cycle;
	\fill[thin, draw = black, /pgfplots/violin color half = arabidopsis, fill = viocolright, fill opacity = .5, x = .25em, y = -.1em, rotate = 90, shift = {(legend symbol)}, yshift = -.1em] plot[domain = 0:6] (\x,{4*1/exp(((\x-3)^2)/2)}) -- cycle;


	%%% enh-effect by B3 binding site (tobacco)
	\coordinate[yshift = -\columnsep] (enhB3Leaf) at (enhTCPLeaf |- plot xlabel.south);
	
	\leafsymbol{enhB3Leaf}; 
	
	\begin{hgroupplot}[%
		ylabel = enhancer responsiveness,
		ymin = -3.75,
		ymax = 10.5,
		ytick = {-10, -8, ..., 10},
		group position = {anchor = above north west, at = {(enhB3Leaf)}, xshift = \plotylabelwidth, yshift = -1.25\baselineskip},
		group/every plot/.append style = {x grids = false},
		zero line,
	]{\twocolumnwidth}{3}{GC content}
	
			
		\nextgroupplot[
			title = Arabidopsis,
			x tick table half = {rawData/enh-effect_B3_At_leaf_boxplot.tsv}{LaTeX.label}
		]
			
		% half violin and box plot
		\halfviolinbox[violin color half = arabidopsis]{rawData/enh-effect_B3_At_leaf_boxplot.tsv}{rawData/enh-effect_B3_At_leaf_violin.tsv}
			
		% add sample size
		\samplesizehalf[violin color half = arabidopsis]{rawData/enh-effect_B3_At_leaf_boxplot.tsv}{id}{n}
		
		% add pvalues
		\signifall*{rawData/enh-effect_B3_At_leaf_pvalues.tsv}	
		
		
		\nextgroupplot[
			title = Maize,
			x tick table half = {rawData/enh-effect_B3_Zm_leaf_boxplot.tsv}{LaTeX.label}
		]
			
		% half violin and box plot
		\halfviolinbox[violin color half = maize]{rawData/enh-effect_B3_Zm_leaf_boxplot.tsv}{rawData/enh-effect_B3_Zm_leaf_violin.tsv}
			
		% add sample size
		\samplesizehalf[violin color half = maize]{rawData/enh-effect_B3_Zm_leaf_boxplot.tsv}{id}{n}
		
		% add pvalues
		\signifall*{rawData/enh-effect_B3_Zm_leaf_pvalues.tsv}
		
		
		\nextgroupplot[
			title = Sorghum,
			x tick table half = {rawData/enh-effect_B3_Sb_leaf_boxplot.tsv}{LaTeX.label}
		]
			
		% half violin and box plot
		\halfviolinbox[violin color half = sorghum]{rawData/enh-effect_B3_Sb_leaf_boxplot.tsv}{rawData/enh-effect_B3_Sb_leaf_violin.tsv}
			
		% add sample size
		\samplesizehalf[violin color half = sorghum]{rawData/enh-effect_B3_Sb_leaf_boxplot.tsv}{id}{n}
		
		% add pvalues
		\signifall*{rawData/enh-effect_B3_Sb_leaf_pvalues.tsv}	

	\end{hgroupplot}
	
	\node[anchor = south west] (legend) at (group c1r1.above north west) {\textbf{B3} transcripition factor binding site: $-$};
	\coordinate (legend symbol) at ($(legend.south east) + (.25em, .2em)$);
	\node[anchor = base west] at ($(legend.base east) + (.6em, 0)$) {$+$};
	
	\fill[thin, draw = black, /pgfplots/violin color half = arabidopsis, fill = viocolleft, fill opacity = .5, x = .25em, y = .1em, rotate = 90, shift = {(legend symbol)}] plot[domain = 0:6] (\x,{4*1/exp(((\x-3)^2)/2)}) -- cycle;
	\fill[thin, draw = black, /pgfplots/violin color half = arabidopsis, fill = viocolright, fill opacity = .5, x = .25em, y = -.1em, rotate = 90, shift = {(legend symbol)}, yshift = -.1em] plot[domain = 0:6] (\x,{4*1/exp(((\x-3)^2)/2)}) -- cycle;
	
	
	%%% enh-effect by B3 binding site (protoplasts)
	\coordinate (enhB3Proto) at (\textwidth - \twocolumnwidth, 0 |- enhB3Leaf);
	
	\protosymbol{enhB3Proto}; 
	
	\begin{hgroupplot}[%
		ylabel = enhancer responsiveness,
		ymin = -3.75,
		ymax = 10.5,
		ytick = {-10, -8, ..., 10},
		group position = {anchor = above north west, at = {(enhB3Proto)}, xshift = \plotylabelwidth, yshift = -1.25\baselineskip},
		group/every plot/.append style = {x grids = false},
		zero line,
	]{\twocolumnwidth}{3}{GC content}
	
			
		\nextgroupplot[
			title = Arabidopsis,
			x tick table half = {rawData/enh-effect_B3_At_proto_boxplot.tsv}{LaTeX.label}
		]
			
		% half violin and box plot
		\halfviolinbox[violin color half = arabidopsis]{rawData/enh-effect_B3_At_proto_boxplot.tsv}{rawData/enh-effect_B3_At_proto_violin.tsv}
			
		% add sample size
		\samplesizehalf[violin color half = arabidopsis]{rawData/enh-effect_B3_At_proto_boxplot.tsv}{id}{n}
		
		% add pvalues
		\signifall*{rawData/enh-effect_B3_At_proto_pvalues.tsv}	
		
		
		\nextgroupplot[
			title = Maize,
			x tick table half = {rawData/enh-effect_B3_Zm_proto_boxplot.tsv}{LaTeX.label}
		]
			
		% half violin and box plot
		\halfviolinbox[violin color half = maize]{rawData/enh-effect_B3_Zm_proto_boxplot.tsv}{rawData/enh-effect_B3_Zm_proto_violin.tsv}
			
		% add sample size
		\samplesizehalf[violin color half = maize]{rawData/enh-effect_B3_Zm_proto_boxplot.tsv}{id}{n}
		
		% add pvalues
		\signifall*{rawData/enh-effect_B3_Zm_proto_pvalues.tsv}
		
		
		\nextgroupplot[
			title = Sorghum,
			x tick table half = {rawData/enh-effect_B3_Sb_proto_boxplot.tsv}{LaTeX.label}
		]
			
		% half violin and box plot
		\halfviolinbox[violin color half = sorghum]{rawData/enh-effect_B3_Sb_proto_boxplot.tsv}{rawData/enh-effect_B3_Sb_proto_violin.tsv}
			
		% add sample size
		\samplesizehalf[violin color half = sorghum]{rawData/enh-effect_B3_Sb_proto_boxplot.tsv}{id}{n}
		
		% add pvalues
		\signifall*{rawData/enh-effect_B3_Sb_proto_pvalues.tsv}	

	\end{hgroupplot}
	
	\node[anchor = south west] (legend) at (group c1r1.above north west) {\textbf{B3} transcripition factor binding site: $-$};
	\coordinate (legend symbol) at ($(legend.south east) + (.25em, .2em)$);
	\node[anchor = base west] at ($(legend.base east) + (.6em, 0)$) {$+$};
	
	\fill[thin, draw = black, /pgfplots/violin color half = arabidopsis, fill = viocolleft, fill opacity = .5, x = .25em, y = .1em, rotate = 90, shift = {(legend symbol)}] plot[domain = 0:6] (\x,{4*1/exp(((\x-3)^2)/2)}) -- cycle;
	\fill[thin, draw = black, /pgfplots/violin color half = arabidopsis, fill = viocolright, fill opacity = .5, x = .25em, y = -.1em, rotate = 90, shift = {(legend symbol)}, yshift = -.1em] plot[domain = 0:6] (\x,{4*1/exp(((\x-3)^2)/2)}) -- cycle;
	
	
	%%% subfigure labels
	\subfiglabel[yshift = .5\columnsep]{enhTCPLeaf}
	\subfiglabel[yshift = .5\columnsep]{enhTCPProto}
	\subfiglabel[yshift = .5\columnsep]{enhWRKYLeaf}
	\subfiglabel[yshift = .5\columnsep]{enhWRKYProto}
	\subfiglabel[yshift = .5\columnsep]{enhB3Leaf}
	\subfiglabel[yshift = .5\columnsep]{enhB3Proto}

\end{tikzpicture}%
			\caption{%
				\textbf{Promoter-proximal transcription factor binding sites influence enhancer responsiveness.}\titleend
				\subfigrange{A}{F} Violin plots of enhancer responsiveness in tobacco leaves \parensubfig[\subfigref{A}\subfigref{C}]{E} or maize protoplasts \parensubfig[\subfigref{B}\subfigref{D}]{F}. Promoters were grouped by GC content and split into promoters without (left half, darker color) or with (right half, lighter color) a TCP \parensubfig[\subfigref{A}]{B}, WRKY \parensubfig[\subfigref{C}]{D}, or B3 \parensubfig[\subfigref{E}]{F} transcription factor binding site. Violin plots are as defined in \autoref{fig:overview}, except only one half is shown.
			}%
			\label{sfig:enhEffectTFs}%
		\end{sfig}
		
		\begin{sfig}
			%\tikzset{external/export next = false}

\newcounter{plotct}


\begin{tikzpicture}	

	%%% plant enhancers
	
	\coordinate (plantEnh) at (0, 0);
	
	\protosymbol{plantEnh}; 
	
	\begin{axis}[
		at = {(plantEnh)},
		xshift = \plotylabelwidth,
		anchor = north west,
		width = \threecolumnwidth - \plotylabelwidth,
		boxplot/draw direction = y,
		x grids = false,
		ylabel = log\textsubscript{2}(enrichment),
		xlabel = enhancer,
		xtick = {1, 2, 3, 4, 5},
		xticklabels = {none, 35S, \textit{AB80}, \textit{Cab-1}, \textit{rbcS-E9}},
		ytick = {-5, ..., 5},
		typeset ticklabels with strut,
		zero line
	]
		
		\foreach \sample in {none, 35S, AB80, Cab-1, rbcS-E9} {
			
			\addplot [boxplot, black, fill = boxcol, mark = solido, mark options = black, boxstyle] table[y = \sample] {../data/misc/data_plant_enhancers_protoplasts.tsv};
		}
		
	\end{axis}
	
\end{tikzpicture}%
			\caption{%
				\textbf{Light-responsive plant enhancers are not active in maize protoplasts.}\titleend
				Constructs harboring no enhancer (none), a 35S enhancer, or one of three light-responsive plant enhancers (\textit{AB80}, \textit{Cab-1}, or \textit{rbcS-E9}) upstream of the 35S minimal promoter were subjected to STARR-seq in maize protoplasts generated from dark-grown plants (Jores et al., 2020). Each boxplot (center line, median; box limits, upper and lower quartiles; whiskers, 1.5 $\times$ interquartile range; points, outliers) denotes the enrichment of all recovered mRNA barcodes over the DNA input. Only one experiment was performed.
			}%
			\label{sfig:plantEnhancers}%
		\end{sfig}
		
		\begin{sfig}
			%\tikzset{external/export next = false}

\begin{tikzpicture}

	%%% promoter strength TF combinations (tobacco)
	\coordinate (TFcomboLeaf) at (0, 0);
	
	\leafsymbol{TFcomboLeaf}
	
	\begin{hgroupplot}[%
		ylabel = {rel. promoter strength},
		ymin = -2,
		ymax = 7,
		ytick = {-10, -8, ..., 10},
		xticklabel style = {name = xticklabel},
		zero line,
		group position = {anchor = above north west, at = {(TFcomboLeaf)}, xshift = \plotylabelwidth},
		group/every plot/.append style = {
			x grids = false,
			typeset ticklabels with strut 
		}
	]{\twocolumnwidth}{2}{}

			
		\nextgroupplot[
			width = 6.2 / 8.2 * \plotwidth,
			title style = {minimum width = 6.2 / 8.2 * \plotwidth},
			title = {two trancription factors},
			x tick table = {rawData/enrichment_TFcombo_double_leaf_boxplot.tsv}{TCP}
		]
		
			% boxplot
			\boxplots{%
				box color = leafCol,
				box shade,
				fill opacity = 0.5%
			}{rawData/enrichment_TFcombo_double_leaf_boxplot.tsv}{rawData/enrichment_TFcombo_double_leaf_boxplot_outliers.tsv}
			
			% add sample size
			\samplesize{rawData/enrichment_TFcombo_double_leaf_boxplot.tsv}{id}{n}
			
			% add significance
			\signifallsimple*{rawData/enrichment_TFcombo_double_leaf_boxplot.tsv}{id}{p.value}
				
			% save coordinates
			\pgfplotstablegetrowsof{rawData/enrichment_TFcombo_double_leaf_boxplot.tsv}
			\pgfplotsinvokeforeach{1, ..., \pgfplotsretval}{
				\coordinate (d#1) at (#1, 0);
			}
		
		
		\nextgroupplot[
			width = 10.2 / 8.2 * \plotwidth,
			title style = {minimum width = 10.2 / 8.2 * \plotwidth},
			title = {three transcription factors},
			x tick table = {rawData/enrichment_TFcombo_triple_leaf_boxplot.tsv}{TCP}
		]
		
			% boxplot
			\boxplots{%
				box color = leafCol,
				box shade,
				fill opacity = 0.5%
			}{rawData/enrichment_TFcombo_triple_leaf_boxplot.tsv}{rawData/enrichment_TFcombo_triple_leaf_boxplot_outliers.tsv}
			
			% add sample size
			\samplesize{rawData/enrichment_TFcombo_triple_leaf_boxplot.tsv}{id}{n}
			
			% add significance
			\signifallsimple*{rawData/enrichment_TFcombo_triple_leaf_boxplot.tsv}{id}{p.value}
				
			% save coordinates
			\pgfplotstablegetrowsof{rawData/enrichment_TFcombo_triple_leaf_boxplot.tsv}
			\pgfplotsinvokeforeach{1, ..., \pgfplotsretval}{
				\coordinate (t#1) at (#1, 0);
			}
			
	\end{hgroupplot}
	
	\node[anchor = base east, node font = \figsmall] (lTCP) at (group c1r1.west |- xticklabel.base) {\strut TCP};
	\node[anchor = north east, node font = \figsmall] (lNAC) at (lTCP.base east) {\strut NAC};
	\node[anchor = north east, node font = \figsmall] (lHSF) at (lNAC.base east) {\strut HSF};
	
	\pgfplotstableforeachcolumnelement{NAC}\of{rawData/enrichment_TFcombo_double_leaf_boxplot.tsv}\as\NAC{
		\pgfmathsetmacro{\i}{\pgfplotstablerow + 1}
		\node[anchor = base, node font = \figsmall] at (lNAC.base -| d\i) {\strut\NAC};
	}
	
	\pgfplotstableforeachcolumnelement{NAC}\of{rawData/enrichment_TFcombo_triple_leaf_boxplot.tsv}\as\NAC{
		\pgfmathsetmacro{\i}{\pgfplotstablerow + 1}
		\node[anchor = base, node font = \figsmall] at (lNAC.base -| t\i) {\strut\NAC};
	}
	
	\pgfplotstableforeachcolumnelement{HSF}\of{rawData/enrichment_TFcombo_double_leaf_boxplot.tsv}\as\HSF{
		\pgfmathsetmacro{\i}{\pgfplotstablerow + 1}
		\node[anchor = base, node font = \figsmall] at (lHSF.base -| d\i) {\strut\HSF};
	}
	
	\pgfplotstableforeachcolumnelement{HSF}\of{rawData/enrichment_TFcombo_triple_leaf_boxplot.tsv}\as\HSF{
		\pgfmathsetmacro{\i}{\pgfplotstablerow + 1}
		\node[anchor = base, node font = \figsmall] at (lHSF.base -| t\i) {\strut\HSF};
	}
	
	
	%%% promoter strength TF combinations (protoplasts)
	\coordinate (TFcomboProto) at (TFcomboLeaf -| \textwidth - \twocolumnwidth, 0);
	
	\protosymbol{TFcomboProto}
	
	\begin{hgroupplot}[%
		ylabel = {rel. promoter strength},
		ymin = -2,
		ymax = 5,
		ytick = {-10, -9, ..., 10},
		xticklabel style = {name = xticklabel},
		zero line,
		group position = {anchor = above north west, at = {(TFcomboProto)}, xshift = \plotylabelwidth},
		group/every plot/.append style = {
			x grids = false,
			typeset ticklabels with strut 
		}
	]{\twocolumnwidth}{2}{}

			
		\nextgroupplot[
			width = 6.2 / 8.2 * \plotwidth,
			title style = {minimum width = 6.2 / 8.2 * \plotwidth},
			title = {two trancription factors},
			x tick table = {rawData/enrichment_TFcombo_double_proto_boxplot.tsv}{TCP}
		]
		
			% boxplot
			\boxplots{%
				box color = protoCol,
				box shade,
				fill opacity = 0.5%
			}{rawData/enrichment_TFcombo_double_proto_boxplot.tsv}{rawData/enrichment_TFcombo_double_proto_boxplot_outliers.tsv}
			
			% add sample size
			\samplesize{rawData/enrichment_TFcombo_double_proto_boxplot.tsv}{id}{n}
			
			% add significance
			\signifallsimple*{rawData/enrichment_TFcombo_double_proto_boxplot.tsv}{id}{p.value}
				
			% save coordinates
			\pgfplotstablegetrowsof{rawData/enrichment_TFcombo_double_proto_boxplot.tsv}
			\pgfplotsinvokeforeach{1, ..., \pgfplotsretval}{
				\coordinate (d#1) at (#1, 0);
			}
		
		
		\nextgroupplot[
			width = 10.2 / 8.2 * \plotwidth,
			title style = {minimum width = 10.2 / 8.2 * \plotwidth},
			title = {three transcription factors},
			x tick table = {rawData/enrichment_TFcombo_triple_proto_boxplot.tsv}{TCP}
		]
		
			% boxplot
			\boxplots{%
				box color = protoCol,
				box shade,
				fill opacity = 0.5%
			}{rawData/enrichment_TFcombo_triple_proto_boxplot.tsv}{rawData/enrichment_TFcombo_triple_proto_boxplot_outliers.tsv}
			
			% add sample size
			\samplesize{rawData/enrichment_TFcombo_triple_proto_boxplot.tsv}{id}{n}
			
			% add significance
			\signifallsimple*{rawData/enrichment_TFcombo_triple_proto_boxplot.tsv}{id}{p.value}
				
			% save coordinates
			\pgfplotstablegetrowsof{rawData/enrichment_TFcombo_triple_proto_boxplot.tsv}
			\pgfplotsinvokeforeach{1, ..., \pgfplotsretval}{
				\coordinate (t#1) at (#1, 0);
			}
			
	\end{hgroupplot}
	
	\node[anchor = base east, node font = \figsmall] (lTCP) at (group c1r1.west |- xticklabel.base) {\strut TCP};
	\node[anchor = north east, node font = \figsmall] (lNAC) at (lTCP.base east) {\strut NAC};
	\node[anchor = north east, node font = \figsmall] (lHSF) at (lNAC.base east) {\strut HSF};
	
	\pgfplotstableforeachcolumnelement{NAC}\of{rawData/enrichment_TFcombo_double_proto_boxplot.tsv}\as\xticklabel{
		\pgfmathsetmacro{\i}{\pgfplotstablerow + 1}
		\node[anchor = base, node font = \figsmall] at (lNAC.base -| d\i) {\strut\xticklabel};
	}
	
	\pgfplotstableforeachcolumnelement{NAC}\of{rawData/enrichment_TFcombo_triple_proto_boxplot.tsv}\as\xticklabel{
		\pgfmathsetmacro{\i}{\pgfplotstablerow + 1}
		\node[anchor = base, node font = \figsmall] at (lNAC.base -| t\i) {\strut\xticklabel};
	}
	
	\pgfplotstableforeachcolumnelement{HSF}\of{rawData/enrichment_TFcombo_double_proto_boxplot.tsv}\as\xticklabel{
		\pgfmathsetmacro{\i}{\pgfplotstablerow + 1}
		\node[anchor = base, node font = \figsmall] at (lHSF.base -| d\i) {\strut\xticklabel};
	}
	
	\pgfplotstableforeachcolumnelement{HSF}\of{rawData/enrichment_TFcombo_triple_proto_boxplot.tsv}\as\xticklabel{
		\pgfmathsetmacro{\i}{\pgfplotstablerow + 1}
		\node[anchor = base, node font = \figsmall] at (lHSF.base -| t\i) {\strut\xticklabel};
	}
	
		
	%%% subfigure labels
	\subfiglabel[yshift = .5\columnsep]{TFcomboLeaf}
	\subfiglabel[yshift = .5\columnsep]{TFcomboProto}
	
\end{tikzpicture}%
			\caption{%
				\textbf{Transcription factor binding sites affect promoter strength additively.}\titleend
				\subfigref{A}\subfigref{B} Boxplots (as defined in \autoref{fig:TATA}) of promoter strength for libraries without enhancer in tobacco leaves \parensubfig{A} or maize protoplasts \parensubfig{B} for synthetic promoters with the indicated numbers of binding sites for TCP, NAC, and HSF transcription factors. The corresponding promoter without any transcription factor binding site was set to 0 (horizontal black line).
			}%
			\label{sfig:TFcombos}%
		\end{sfig}
		
%		\begin{stab}
%			\caption{caption}%
%			\input{SuppTable1}%
%		\end{stab}
		
	\fi
	
	%%% Supp end

\end{document}